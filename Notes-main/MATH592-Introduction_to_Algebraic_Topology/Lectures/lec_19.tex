\chapter{Homology}
\hyperref[def:homology-class]{Homology} is a general way of associating a sequence of algebraic objects, such as \hyperref[def:Abelian-group]{Abelian groups}
or modules, with other  mathematical objects such as topological spaces. In this section, we'll develop this notion rigorously, and make some connection
with \hyperref[def:fundamental-group]{fundamental group} to see why this is much more powerful.


\lecture{19}{18 Feb. 10:00}{Simplex and Homology}
\section{Motivation for Homology}
Informally, the higher \hyperref[def:homotopy]{homotopy} groups is defined as
\[
	\pi _{n} (X, x_0)\colon I^n_\ast \to (X, x_0),\quad \partial I^n \mapsto x_0.
\]

\begin{figure}[H]
	\centering
	\incfig{higher-homotopy-group-dim2}
	\label{fig:higher-homotopy-group-dim2}
\end{figure}


We see that it's extremely hard to compute higher \hyperref[def:fundamental-group]{fundamental group}. Hence instead,
we will study the higher dimensional structure of \(X\) via \emph{homology}.

\begin{itemize}
	\item \textbf{Cons.}
	      \begin{itemize}
		      \item The definition is more opaque at first encounter.
	      \end{itemize}
	\item \textbf{Pros.}
	      \begin{itemize}
		      \item Lots of computational tools
		      \item Functional
		      \item \hyperref[def:Abelian-group]{Abelian Groups}
		            \begin{remark}
			            More like \(\pi _n\) for \(n>1\).
		            \end{remark}
		      \item No basepoints
		      \item Can compute using \hyperref[def:CW-subcomplex]{CW} structure.
		      \item Good properties. For example, \(H_{n} = 0\) if \(n > \mathrm{dim} X\)
	      \end{itemize}
\end{itemize}

\section{Simplicial Homology}
\subsection{\(\Delta\)-Simplex}
This is a stricter version of a \hyperref[def:CW-Complex]{CW complex} which allows us to decompose our spaces into \hyperref[def:cell]{cells}.
In terms of how things fit together, we have the following diagram.
\begin{figure}[H]
	\centering
	\incfig{simplicial-homology-venn-diagram}
	\label{fig:simplicial-homology-venn-diagram}
\end{figure}

Now we try to give the definition.
\begin{definition}[Simplex]\label{def:simplex}
	We see that
	\begin{itemize}
		\item \emph{\(0\)-simplex}. A point.
		\item \emph{\(1\)-simplex}. Interval.
		\item \emph{\(2\)-simplex}. Triangle.
		\item \emph{\(3\)-simplex}. Tetrahedron.
		\item \emph{\(n\)-simplex}. The convex hull of \((n+1)\)-points position in \(\mathbb{\MakeUppercase{r}} ^n\).
	\end{itemize}
	\begin{figure}[H]
		\centering
		\incfig{def:simplex}
		\label{fig:def:simplex}
	\end{figure}
\end{definition}
\begin{remark}
	We see that
	\begin{itemize}
		\item The top of which is the \(2\)-disk and remember \hyperref[def:cell]{cell} structure (edges and vertices) and remember orientation (ordering on vertices).
		\item The top of which is the \(3\)-disk and \hyperref[def:cell]{cells} and the orientation.
		\item We can view \hyperref[def:simplex]{simplices} as both \emph{combinatorial} and \emph{topological} objects.
	\end{itemize}
\end{remark}

An alternative definition can be made.
\begin{definition}[Standard simplex]\label{def:standard-simplex}
	We say that an \(n\)-dimensional \emph{standard simplex}, denoted by \(\Delta^n\) is
	\[
		\Delta^n = \left\{(t_0, \ldots , t_{n}) \in \mathbb{\MakeUppercase{r}}^{n+1} \mid t_i \geq 0,\ \sum\limits_{i}t_{i}  = 1 \right\}.
	\]
	We'll call such a simplex as \emph{standard \(n\)-simplex}.
	\begin{figure}[H]
		\centering
		\incfig{def:standard-simplex}
		\label{fig:def:standard-simplex}
	\end{figure}
\end{definition}
\begin{remark}
	In our definition, the \hyperref[def:standard-simplex]{standard simplices} will implicitly come with a choice of \underline{ordering of the vertices} as
	\[
		\Delta ^n = [v_0, v_1, \ldots , v_n ]
	\]
	such that the convex hull of these points is taken with this ordering.
\end{remark}