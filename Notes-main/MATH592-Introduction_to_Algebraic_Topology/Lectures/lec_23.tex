\lecture{23}{07 Mar. 10:00}{Singular Homology}
\section{Singular Homology}
As we noted before, we can give a different structure of \hyperref[def:chain-complex]{chain complex}, which shall
induces a different \hyperref[def:homology-group]{homology group} compare to \hyperref[def:simplicial-homology-group]{simplicial homology group}.

We now see one abstract way to define \(\sigma \), which will give us so-called \hyperref[def:singular-homology-group]{singular homology group}.

\begin{definition}[Singular simplex]\label{def:singular-simplex}
	A \emph{singular \(n\)-simplex} in a space \(X\) is a continuous map
	\[
		\sigma \colon \Delta^n \to X.
	\]
\end{definition}

\begin{definition}[Singular chain complex]\label{def:singular-chain-complex}
	The \hyperref[def:chain-complex]{chain complex} defined with \hyperref[def:singular-chain-group]{singular chain group} and
	\hyperref[def:singular-boundary-map]{singular boundary map} defined as follows is called \emph{singular chain complex}.
	\begin{definition}[Singular chain group]\label{def:singular-chain-group}
		Let \(C_n(X)\) be the \hyperref[def:free-group]{free group} on \hyperref[def:singular-simplex]{singular \(n\)-simplices} in \(X\), which we
		call it the \emph{singular \(n\)-chain group}.
	\end{definition}
	\begin{definition}[Singular boundary map]\label{def:singular-boundary-map}
		With \(C_\ast\) being the \hyperref[def:singular-chain-group]{singular chain group}, we defined so-called \emph{singular boundary map} \(\partial _n\) as
		\[
			\begin{split}
				\partial_n \colon C_n(X) & \to C_{n - 1}(X)                                                                       \\
				\sigma              & \mapsto \sum_{i = 1}^n (-1)^i \at{\sigma}{[v_0, \ldots, \widehat{v}_i, \ldots, v_n]}{}.
			\end{split}
		\]
	\end{definition}
\end{definition}

\begin{definition}[Singular homology group]\label{def:singular-homology-group}
	The \emph{singular homology groups} are the \hyperref[def:homology-group]{homology groups} of this \hyperref[def:singular-chain-complex]{singular chain complex}
	given as
	\[
		H_n(X) = \quotient{\ker \partial_n}{\im \partial_{n + 1}}.
	\]
\end{definition}
\begin{remark}
	We now see that from the definition of \hyperref[def:homology-group]{homology group}, we can put different structure on which.
	But the idea is the same, namely we are taking \(H_{n} (X)\) being
	\[
		H_n(X) \coloneqq \quotient{\ker \partial _n}{\im \partial _{n+1}},
	\]
	where the difference is what structure we put on \(X\) which induces different \hyperref[def:chain-complex]{chain complex} \((C_{\ast} (X), \partial _\ast)\).
	In this case, we have \hyperref[def:singular-homology-group]{singular homology group} since we are considering
	\hyperref[def:singular-chain-complex]{singular chain complex}, while we can also have
	\hyperref[def:simplicial-homology-group]{simplicial homology group}.
\end{remark}

Since the generating sets for \(C_{n} (X)\) when considering \hyperref[def:singular-chain-complex]{singular chain complex} are almost
always hugely uncountable from its definition, it's almost impossible to compute with these. However, it does give us a definition
that does not depend on any other structure than the topology of \(X\), making it useful for \underline{developing theory}.

\begin{note}
	The heuristic is that, we interpret a \hyperref[def:chain-group]{chain} \(\sigma_1 \pm \sigma_2 \pm \cdots \pm \sigma_k\) as a
	map from a \hyperref[def:delta-complex]{\(\Delta\)-complex} to \(X\).

	\begin{eg}
		For example, with \(\sigma_1 + \sigma_2\) as below,
		\begin{figure}[H]
			\centering
			\incfig{note:chain-as-a-map}
			\label{fig:note:chain-as-a-map}
		\end{figure}
		where we've glued \([v_1, v_2]\) of \(\sigma_1\) to \([v_0, v_2]\) of \(\sigma_2\) if \(\at{\sigma_1}{[v_1, v_2]}{}\) and
		\(\sigma_{[v_0, v_2]}\) are the same \hyperref[def:singular-chain-group]{singular \(n\)-chain} with opposite signs.
	\end{eg}
\end{note}

With what we have defined, we now have some \emph{goals}.
\begin{itemize}
	\item \hyperref[def:singular-homology-group]{Singular homology} is a \hyperref[def:homotopy]{homotopy} invariant. (\autoref{thm:functoriality-is-homotopy-invariant})
	\item \hyperref[def:singular-homology-group]{Singular} and \hyperref[def:simplicial-homology-group]{simplicial} \hyperref[def:homology-group]{homology groups} are isomorphic. (\autoref{thm:singular-homology-agrees-with-simplicial-homology})
\end{itemize}

\begin{exercise}
	Check that if \(X\) has \hyperref[def:path]{path} components \(\{X_\alpha\}\) then
	\[
		H_n(X) \cong \bigoplus_\alpha H_n(X_\alpha).
	\]
\end{exercise}
\begin{exercise}
	If \(X = \{\ast\}\), then
	\[
		H_n(X) = \begin{dcases}
			\mathbb{\MakeUppercase{z}}, & \text{ if } n=0 ;     \\
			0,                          & \text{ if } n\geq 1 .
		\end{dcases}
	\]
\end{exercise}
\begin{exercise}
	If \(X\) is \hyperref[def:path]{path}-connected, then \(H_0(X) \cong \mathbb{\MakeUppercase{z}}\).
\end{exercise}

\section{Functoriality and Homotopy Invariance}
\begin{definition}[Induced map on singular chains]\label{def:induced-map-on-singular-chain}
	For a given continuous map \(f \colon X \to Y\), we can consider the \emph{map \(f_{\#}\) induced by \hyperref[def:singular-chain-group]{singular chains}} as
	\[
		\begin{split}
			f_{\#} \colon C_n(X)           & \to C_n(Y)                                      \\
			[\sigma \colon \Delta^n \to X] & \mapsto [f \circ \sigma \colon \Delta^n \to Y].
		\end{split}
	\]
\end{definition}
\begin{note}
	Note that we're considering \hyperref[def:singular-chain-group]{singular chain groups} specifically in this case.
\end{note}
\begin{remark}
	We see that the functoriality doesn't depend on any kind of \hyperref[def:delta-complex]{\(\Delta \)-complex} structure.
\end{remark}

\begin{definition}[Chain map]\label{def:chain-map}
	Given two \hyperref[def:chain-complex]{chain complexes} \((C_\ast, \partial_\ast)\) and \((D_\ast, \delta_\ast)\), a
	\emph{chain map} between them is a collection of group homomorphisms \(f_n \colon C_n \to D_n\) such that the following diagram commutes.
	\[
		\begin{tikzcd}
			\ldots & {C_{n+1}} & {C_n} & {C_{n-1}} & \ldots \\
			\ldots & {D_{n+1}} & {D_n} & {D_{n-1}} & \ldots
			\arrow["{\partial_{n+2}}", from=1-1, to=1-2]
			\arrow["{\delta_{n+2}}", from=2-1, to=2-2]
			\arrow["{\delta_{n+1}}", from=2-2, to=2-3]
			\arrow["{\delta_{n}}", from=2-3, to=2-4]
			\arrow["{\partial_{n-1}}", from=1-4, to=1-5]
			\arrow["{\partial_{n}}", from=1-3, to=1-4]
			\arrow["{\partial_{n+1}}", from=1-2, to=1-3]
			\arrow["{\delta_{n-1}}", from=2-4, to=2-5]
			\arrow["{f_{n-1}}", from=1-4, to=2-4]
			\arrow["{f_{n}}", from=1-3, to=2-3]
			\arrow["{f_{n+1}}", from=1-2, to=2-2]
		\end{tikzcd}
	\]
	i.e. we have that \(\delta_n \circ f_n = f_{n - 1} \circ \partial_n\).
\end{definition}

\begin{exercise}
	We have that \(f_{\#} \partial = \partial f_{\#}\). In other words, \(f_{\#}\) is a \hyperref[def:chain-map]{chain map}.
	Thus, by the homework \(f_{\#}\) induces a group homomorphism on the \hyperref[def:homology-group]{homology groups}. We write
	this as \(f_\ast \colon H_n(X) \to H_n(Y)\) for all \(n\).
\end{exercise}
\begin{exercise}
	We have \underline{functoriality}, i.e. \((f \circ g)_\ast = f_\ast \circ g_\ast\) and \((\identity_X)_\ast = \identity_{H_n(X)}\).
\end{exercise}

\begin{theorem}[Homology group defines a functor]\label{thm:homology-group-defines-a-functor}
	The \hyperref[def:homology-group]{\(n\)-th homology group} \(H_n \colon X \mapsto H_n(X)\) gives a \hyperref[def:functor]{functor} from \(\underline{\mathrm{Top}}\)
	to \(\underline{\mathrm{Ab} }\).
\end{theorem}
\begin{proof}
	This follows from the two exercises above.
\end{proof}

\begin{theorem}[Functoriality is homotopy invariant]\label{thm:functoriality-is-homotopy-invariant}
	If \(f, g\colon X \to Y\) are \hyperref[def:homotopic]{homotopic}, then they will induce the same map on \hyperref[def:homology-group]{homology}
	\[
		f_\ast = g_\ast \colon H_n(X) \to H_n(Y).
	\]
\end{theorem}
The proof of \autoref{thm:functoriality-is-homotopy-invariant} can be found \hyperref[pf:functoriality-is-homotopy-invariant]{here}.

\begin{exercise}
	\autoref{thm:homology-group-defines-a-functor} and \autoref{thm:functoriality-is-homotopy-invariant} imply that \(H_{n} \) is a \hyperref[def:homotopy]{homotopy} invariant.
\end{exercise}
\todo[inline]{I'm not sure whether the above discussion holds only for \hyperref[def:singular-homology-group]{singular homology group} or
	can be extended to \hyperref[def:homology-group]{general homology group}. I link them to \hyperref[def:homology-group]{general homology group} anyway.}