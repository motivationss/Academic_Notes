\lecture{9}{29 Sep. 12:30}{Analysis of HITS}
\section{Analysis of HITS}
Though the \hyperref[algo:HITS-algorithm]{HITS algorithm} is clear and simple enough, but we can indeed get a closed form solution
to both the authority and the hub scores of the algorithm.
\begin{itemize}
	\item Authority update:
	      \[
		      \mathtt{a_k(\mathnormal{i})}  = \sum\limits_{j\in \mathcal{V}\colon j\to i\in \mathcal{E}}\mathtt{h_{k-1}(\mathnormal{j})}
		      =\sum\limits_{j\in\mathcal{V}\colon A_{ji} = 1}\mathtt{h_{k-1}(\mathnormal{j})}
		      =\sum\limits_{j\in\mathcal{V}}A_{ji}\mathtt{h_{k-1}(\mathnormal{j})}
		      = \left(A^{\top} \mathtt{h_{k-1}}\right)_i.
	      \]
	      We see that
	      \[
		      \mathtt{a_k} = A^{\top}\mathtt{h_{k-1}}.
	      \]
	\item Hub update:
	      \[
		      \mathtt{h_k(\mathnormal{j})} = \sum\limits_{j\in\mathcal{V}\colon i\to j\in\mathcal{E}} \mathtt{a_k(\mathnormal{j})}
		      = \sum\limits_{j\in\mathcal{V}} \mathtt{a_k(\mathnormal{j})}
		      = \sum\limits_{j\in\mathcal{V} }A_{ij}\mathtt{a_k(\mathnormal{j})}
		      = \left(A\mathtt{a_k}\right)_i.
	      \]
	      We see that
	      \[
		      \mathtt{h_{k}} = A \mathtt{a_{k}}.
	      \]
	      Furthermore,
	      \[
		      \mathtt{h_k} = A\mathtt{a_k}\implies \mathtt{h_k}= A\mathtt{a_k} = A(A^{\top}\mathtt{h_{k-1}}) = A A^{\top} \mathtt{h_{k-1}}.
	      \]
\end{itemize}
If we ignore the normalization for now, then
\[
	\mathtt{h_k} = A A^{\top}\mathtt{h_{k-1}} = (A A^{\top})A A^{\top} h_{k-2} = \cdots = (A A^{\top})^k \mathtt{h_0}.
\]
Hence, we need to study \(A A^{\top}\). Firstly, from \autoref{lma:lec8-2}, we see that \(A A^{\top} \) is \hyperref[def:positive-semi-definite]{positive semi-definite}
and \hyperref[def:symmetric-matrix]{symmetric}. Furthermore, from \autoref{thm:spectral-theorem}, since \(A A^{\top} \) is symmetric, all \hyperref[def:eigenvalue]{eigenvalues} are real. Then
from \autoref{lma:lec8-1}, we see that all \hyperref[def:eigenvalue]{eigenvalue} of \(A A^{\top} \) are positive, i.e.,
\[
	0 \leq \lambda_1 \leq \lambda_2 \leq \cdots \leq \lambda_{\left\vert \mathcal{\MakeUppercase{v}} \right\vert } .
\]

From the \autoref{thm:spectral-theorem}, let the \hyperref[def:eigenvector]{eigenvectors} of \(A A^{\top} \) be \(u_1, \ldots , u_{\left\vert \mathcal{\MakeUppercase{v}} \right\vert } \), then
\[
	\mathtt{h_0} = \sum\limits_{i=1}^{\left\vert \mathcal{\MakeUppercase{v}} \right\vert } \alpha_i u_{i}
\]
where \(\alpha_{i}\) are unique since \(\{u_{i}\} \) form a basis, and
\[
	\alpha_{i} = u_{i}^{\top}\mathtt{h_0} = \sum\limits_{j=1}^{\left\vert \mathcal{\MakeUppercase{v}} \right\vert } \alpha_{j}u_{i}^{\top}u_{j}.
\]
Moreover, we have
\[
	\begin{split}
		\mathtt{h_1} = (A A^{\top})\mathtt{h_0} &= \sum\limits_{i=1}^{\left\vert \mathcal{\MakeUppercase{v}} \right\vert } \alpha_{i} \lambda_{i}u_{i};\\
		&\vdots\\
		\mathtt{h_k} = (A A^{\top})^k\mathtt{h_0} &= \sum\limits_{i=1}^{\left\vert \mathcal{\MakeUppercase{v}} \right\vert } \alpha_{i} \lambda_{i}^k u_{i}.\\
	\end{split}
\]
With the above formula, we now consider
\[
	0 \leq \lambda_1 \leq \lambda_2 \leq \cdots \leq \underbrace{\lambda_{\left\vert \mathcal{\MakeUppercase{v}} \right\vert}}_{\text{largest}}.
\]
\begin{enumerate}
	\item[Case i.] Unique largest \hyperref[def:eigenvalue]{eigenvalue}:
		\(\lambda_{\left\vert \mathcal{\MakeUppercase{v}} \right\vert } > \lambda_{\left\vert \mathcal{\MakeUppercase{v}} \right\vert -1} \geq \cdots \geq \lambda_{1}\geq 0\),
		then the term \(\lambda^k\)	will dominate, namely
		\[
			\mathtt{h_k} = \sum\limits_{i=1}^{\left\vert \mathcal{\MakeUppercase{v}} \right\vert } \alpha_{i}\lambda_{i}^k u_{i}.
		\]
		Dividing both sides by \(\lambda_{\left\vert \mathcal{\MakeUppercase{v}} \right\vert }^k\), we have
		\[
			\frac{\mathtt{h_{k}}}{\lambda_{\left\vert \mathcal{\MakeUppercase{v}} \right\vert }^k}
			= \sum\limits_{i=1}^{\left\vert \mathcal{\MakeUppercase{v}} \right\vert  - 1}\alpha_{i}
			\left(\frac{\lambda_{i}}{\lambda_{\left\vert \mathcal{\MakeUppercase{v}} \right\vert }}\right)^k u_{i}
			+ \alpha_{\left\vert \mathcal{\MakeUppercase{v}} \right\vert }u_{\left\vert \mathcal{\MakeUppercase{v}} \right\vert }.
		\]
		Then since \(\lambda_{i}/\lambda_{\left\vert \mathcal{\MakeUppercase{v}} \right\vert }< 1\) for \(1\leq i\leq \left\vert \mathcal{\MakeUppercase{v}} \right\vert -1\),
		hence when \(k\to \infty\), all terms in the sum will converge to \(0\), leaving us with
		\[
			\lim_{k \to \infty} \frac{\mathtt{h_k}}{\lambda_{\left\vert \mathcal{\MakeUppercase{v}} \right\vert }^k}= \alpha_{\left\vert \mathcal{\MakeUppercase{v}} \right\vert } u_{\left\vert \mathcal{\MakeUppercase{v}} \right\vert }.
		\]

		Since \(u_{\left\vert \mathcal{\MakeUppercase{v}} \right\vert }\) is a non-zero vector, we consider the following two cases:
		\begin{enumerate}
			\item All terms are non-negative with one positive.
			      Firstly, suppose \(\mathtt{h_0}= x>0\) such that \(u_{\left\vert \mathcal{\MakeUppercase{v}} \right\vert }^{\top}x = \alpha_{\left\vert \mathcal{\MakeUppercase{v}} \right\vert } >0\).
			      Then we have
			      \[
				      \lim\limits_{h \to \infty} \frac{\left(A A^{\top}\right)^k x}{\lambda_{\left\vert \mathcal{\MakeUppercase{v}} \right\vert }^k} = \alpha_{\left\vert \mathcal{\MakeUppercase{v}} \right\vert }u_{\left\vert \mathcal{\MakeUppercase{v}} \right\vert }.
			      \]
			      We see that the left-hand side is greater or equal to \(0\), and stays in positive for every \(k\implies \) the limit is also positive. Now, since
			      \(\alpha_{\left\vert \mathcal{\MakeUppercase{v}} \right\vert }>0\), we can conclude that \(u_{\left\vert \mathcal{\MakeUppercase{v}} \right\vert }\geq 0\).
			      In other words, \(\mathtt{h_0}\) can be any vector with all coefficients positive and then
			      \(\alpha_{\left\vert \mathcal{\MakeUppercase{v}} \right\vert } = u^{\top}_{\left\vert \mathcal{\MakeUppercase{v}} \right\vert }\mathtt{h_0}>0\).
			\item At least one term is positive.
			      \(\mathtt{h_0} = x > 0\) such that \(u_{\left\vert \mathcal{\MakeUppercase{v}} \right\vert }^{\top}x>0\).
		\end{enumerate}

		\begin{remark}
			To rank nodes, all we need is the vector \(u_{\left\vert \mathcal{\MakeUppercase{v}} \right\vert }\).
			And the limiting value of \(u_{\left\vert \mathcal{\MakeUppercase{v}} \right\vert }\) is called the \textbf{hub score}.
		\end{remark}
	\item[Case ii.] Non-unique largest \hyperref[def:eigenvalue]{eigenvalues}: For discussion, we assume that the largest \hyperref[def:eigenvalue]{eigenvalue} has algebraic multiplicity of \(2\). Namely,
		\[
			\lambda_{\left\vert \mathcal{\MakeUppercase{v}} \right\vert } = \lambda_{\left\vert \mathcal{\MakeUppercase{v}} \right\vert -1} \geq \cdots \geq \lambda_{1}\geq 0.
		\]
		Again, from the same discussion, we see that
		\[
			\mathtt{h_{0}} = \sum\limits_{i=1}^{\left\vert \mathcal{\MakeUppercase{v}} \right\vert } \alpha_{i}u_{i}.
		\]
		Furthermore, we see that
		\[
			\frac{\left(A A^{\top}\right)^k h_0}{\lambda_{\left\vert \mathcal{\MakeUppercase{v}} \right\vert }^k}
			= \sum\limits_{i=1}^{\left\vert \mathcal{\MakeUppercase{v}} \right\vert - 2} \alpha_{i}\left(\frac{\lambda_{i}}{\lambda_{\left\vert \mathcal{\MakeUppercase{v}} \right\vert }}\right)^k u_{i}
			+ \alpha_{\left\vert \mathcal{\MakeUppercase{v}} \right\vert -1}u_{\left\vert \mathcal{\MakeUppercase{v}} \right\vert -1}+\alpha_{\left\vert \mathcal{\MakeUppercase{v}} \right\vert } u_{\left\vert \mathcal{\MakeUppercase{v}} \right\vert }.
		\]
		Since \(\lambda_{i}/\lambda_{\left\vert \mathcal{\MakeUppercase{v}} \right\vert }< 1\) for \(1\leq i\leq V-2\), hence when \(k\to \infty \), all terms in the sum will converge to \(0\), leaving us with
		\[
			\lim_{k \to \infty} \frac{(A A^{\top})^{k}\mathtt{h_0}}{\lambda_V^k} = \alpha_{\left\vert \mathcal{\MakeUppercase{v}} \right\vert -1}u_{\left\vert \mathcal{\MakeUppercase{v}} \right\vert -1}
			+ \alpha_{\left\vert \mathcal{\MakeUppercase{v}} \right\vert } u_{\left\vert \mathcal{\MakeUppercase{v}} \right\vert }.
		\]
		\begin{remark}
			It still converges but it may not be \(u_{\left\vert \mathcal{\MakeUppercase{v}} \right\vert }\).
		\end{remark}
\end{enumerate}



