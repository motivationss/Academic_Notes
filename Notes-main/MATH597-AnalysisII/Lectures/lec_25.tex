\lecture{25}{14 Mar. 11:00}{Banach Spaces}
\begin{proposition}
	Suppose \(\mu (X)<\infty \), then for every \(0<p<q\leq \infty \), \(L^q\subseteq L^p\).
\end{proposition}
\begin{proof}
	Suppose \(q<\infty \), then
	\[
		\int \left\vert f \right\vert ^p \leq \left(\int \left(\left\vert f \right\vert ^p\right)^{q / p}\right)^{p / q}\left(\int 1^{q/(q-p)}\right)^{1 - p / q}
		= \left(\int \left\vert f \right\vert ^q\right)^{p / q}\mu (x)^{1 - p / q}< \infty
	\]
	where we split \(\int \left\vert f \right\vert ^p\) into \(\int \left\vert f \right\vert ^p\cdot 1\). From \hyperref[thm:Holder-inequality]{Hölder's inequality}
	with \(q / p > 1\), we have
	\[
		\left\lVert f\right\rVert _p \leq \left\lVert f\right\rVert _q \mu (X)^{1 / p - 1 / q}< \infty .
	\]
	The case that \(q = \infty \) is left as an exercise.\todo{DIY}
\end{proof}

\begin{proposition}
	If \(0<p<q\leq \infty \), then \(\ell ^p\subseteq \ell ^q\).
\end{proposition}
\begin{proof}
	We consider two cases.
	\begin{itemize}
		\item When \(q = \infty \), we have
		      \[
			      \left\lVert a\right\rVert ^p_\infty = \left(\sup _i \left\vert a_{i}  \right\vert \right)^p = \sup _i \left\vert a_{i}  \right\vert ^p \leq \sum\limits_{i=1}^{\infty} \left\vert a_{i}  \right\vert ^p.
		      \]
		      Thus, \(\left\lVert a\right\rVert _\infty \leq \left\lVert a\right\rVert _p\).
		\item When \(q<\infty \), we see that
		      \[
			      \sum\limits_{i=1}^{\infty}\left\vert a_{i}  \right\vert  ^q
			      = \sum\limits_{i=1}^{\infty} \left\vert a_{i}  \right\vert ^p \cdot \left\vert a_{i}  \right\vert ^{q - p}
			      \leq \left\lVert a\right\rVert _\infty ^{q-p}\sum\limits_{i=1}^{\infty} \left\vert a_{i}  \right\vert ^p
			      \leq \left\lVert a\right\rVert _p^{q-p} \cdot \left\lVert a\right\rVert _p^p = \left\lVert a\right\rVert ^q_p.
		      \]
		      Therefore,
		      \[
			      \left\lVert a\right\rVert _q\leq \left\lVert a\right\rVert _p.
		      \]
	\end{itemize}
\end{proof}

\begin{proposition}
	For all \(0<p<q<r\leq \infty \), \(L^p \cap L^{r} \subseteq L^q \).
\end{proposition}
\begin{proof}
	\todo{DIY}
\end{proof}

\section{Banach Spaces}
Let's start with a definition.
\begin{definition}[Cauchy sequence]\label{def:Cauchy-sequence}
	Let \(Y, \rho \) be a \hyperref[def:metric]{metric} space. We call \(x_{n} \) a \emph{Cauchy sequence} if for every \(\epsilon >0\), there exists an
	\(N\in \mathbb{\MakeUppercase{n}} \) such that for all \(n, m\geq N\), \(\rho (x_{n} , x_{m} )< \epsilon \).
\end{definition}
\begin{note}
	Convergent sequence are \hyperref[def:Cauchy-sequence]{Cauchy}.
\end{note}

\begin{definition}[Complete]\label{def:complete}
	A \hyperref[def:metric]{metric} space \((Y, \rho )\) is called \emph{complete} if every \hyperref[def:Cauchy-sequence]{Cauchy sequence} in \(Y\) converges.
\end{definition}

\begin{eg}
	We first see some examples.
	\begin{enumerate}[(1)]
		\item We see that \(\mathbb{\MakeUppercase{q}} \) with \(\rho (x, y) = \left\vert x - y \right\vert \) is \textbf{not} \hyperref[def:complete]{complete}, but \(\mathbb{\MakeUppercase{r}} \) with
		      the same \hyperref[def:metric]{metric} is \hyperref[def:complete]{complete}.
		\item \(C([0, 1])\) with \(\rho (f, g) = \left\lVert f-g\right\rVert _\infty \) is \hyperref[def:complete]{complete}, but with \(\rho (f, g) = \int \left\vert f-g \right\vert \) is not.
	\end{enumerate}
\end{eg}

\begin{definition}[Banach space]\label{def:Banach-space}
	A \emph{Banach space} is a \hyperref[def:complete]{complete} \hyperref[def:norm]{normed} vector space.
\end{definition}
\begin{remark}
	Namely, a vector space equipped with a \hyperref[def:norm]{norm} whose \hyperref[def:metric]{metric} \hyperref[induced-metric]{induced by the norm} is
	\hyperref[def:complete]{complete}.
\end{remark}

\begin{theorem}\label{thm:lec-25}
	Let \((V, \left\lVert \cdot\right\rVert )\) be a \hyperref[def:norm]{normed} space. Then,
	\[
		\text{\(V\) is \hyperref[def:complete]{complete}} \iff \text{every absolutely convergent series is convergent}.
	\]
	i.e., if \(\sum\limits_{i=1}^{\infty} \left\lVert v_{i} \right\rVert<\infty\), then \(\left\{\sum\limits_{i=1}^{N} v_{i} \right\}_{N\in \mathbb{\MakeUppercase{n}} }\)
	converges to some \(s\in V\).
\end{theorem}
Before we prove \autoref{thm:lec-25}, we first see one of the result based on this theorem.\footnote{The proof can be found in \hyperref[pf:thm:lec25]{here}.}

\begin{theorem}[Riesz-Fischer theorem]\label{thm:Riesz-Fischer-theorem}
	For every \(1\leq p\leq \infty \), we have \(L^p(X, \mathcal{\MakeUppercase{a}} , \mu )\) is \hyperref[def:complete]{complete}, hence a
	\hyperref[def:Banach-space]{Banach space}.
\end{theorem}
\begin{proof}
	We prove this in two cases.
	\begin{itemize}
		\item We first prove this for \(1\leq p<\infty \). Suppose \(f_{n} \in L^p\) and \(\sum\limits_{n=1}^{\infty} \left\lVert f_{n} \right\rVert_{p}<\infty \).

		      We need to show that there is an \(F\in L^p\) such that \(\left\lVert \sum\limits_{n=1}^{N} f_{n} -F\right\rVert_{p}\to 0\) as \(N\to \infty \). i.e., we need to show
		      the following.
		      \begin{enumerate}
			      \item \(\sum\limits_{n=1}^{\infty} f_{n}(x)\) is convergent \hyperref[def:mu-almost-everywhere]{a.e.} In fact, we can show this by showing the following.
			            \begin{claim}
				            We have
				            \[
					            \int \sum\limits_{n=1}^{\infty} \left\vert f_{n} (x) \right\vert < \infty .
				            \]
			            \end{claim}
			            \begin{explanation}
				            Let \(G(x) = \sum\limits_{n=1}^{\infty} \left\vert f_{n} (x) \right\vert = \sup _N \sum\limits_{n=1}^{N} \left\vert f_{n} (x) \right\vert  \), \(G\colon X\to [0, \infty ]\). Also, let
				            \(G_{N} (x)= \sum\limits_{n=1}^{N} \left\vert f_{n} (x) \right\vert \). Then, we have
				            \[
					            0\leq G_{1} \leq G_2 \leq \ldots \leq G,
				            \]
				            and \(G_{N} \to G \). Furthermore,
				            \[
					            0\leq G^p_1 \leq G^p_2 \leq \ldots \leq G^p,
				            \]
				            and \(G^p_N \to G^p\). From \hyperref[thm:MCT]{monotone convergence theorem},
				            \[
					            \int G^p = \lim\limits_{N \to \infty} \int G^p_N.
				            \]
				            From \hyperref[thm:Minkowski-inequality]{Minkowski inequality}, we further have
				            \[
					            \left\lVert G_N\right\rVert _p \leq \sum\limits_{n=1}^{N} \left\lVert f_{n} \right\rVert _p \leq \sum\limits_{n=1}^{\infty} \left\lVert f_{n} \right\rVert _p \coloneqq B < \infty.
				            \]
				            Thus,
				            \[
					            \int G(x)^p = \lim\limits_{N \to \infty} \int G^p_N = \lim\limits_{N \to \infty} \left\lVert G_{N} \right\rVert ^p_p \leq B^{p} < \infty .
				            \]
				            We see that \(G\) is finite \hyperref[def:mu-almost-everywhere]{a.e.} as desired. This implies that \(\sum\limits_{n=1}^{\infty }\left\vert f_{n} (x) \right\vert  < \infty \)
				            \hyperref[def:mu-almost-everywhere]{a.e.}, so \(\sum\limits_{n=1}^{\infty }f_{n} (x) \) converges \hyperref[def:mu-almost-everywhere]{a.e.} Now, we simply let
				            \[
					            F(x) = \begin{dcases}
						            \sum\limits_{n=1}^{\infty} f_{n} (x), & \text{ if it converges}  ; \\
						            0,                                    & \text{ otherwise}.
					            \end{dcases}
				            \]
			            \end{explanation}
			      \item \(F\in L^p\), where \(F(x)\coloneqq \sum\limits_{n=1}^{\infty}f_{n} (x) \) \hyperref[def:mu-almost-everywhere]{a.e.} and say is zero elsewhere.
			            \begin{claim}
				            \(F\) defined in this way is indeed in \(L^p\).
			            \end{claim}
			            \begin{explanation}
				            This is clear since
				            \[
					            \left\vert F(x) \right\vert \leq G(x) \implies \int \left\vert F \right\vert ^p \leq \int G^p < \infty ,
				            \]
				            hence \(F\in L^p\).
			            \end{explanation}
			      \item We then show the last condition we need to check.
			            \begin{claim}
				            \(\left\lVert \sum\limits_{n=1}^{N} f_{n} -F\right\rVert_p \to 0\) as \(N\to \infty \).
			            \end{claim}
			            \begin{explanation}
				            We now see that
				            \[
					            \left\vert \sum\limits_{n=1}^{N} f_{n} (x) - F(x) \right\vert ^p \leq \left(\sum\limits_{n=1}^{\infty} \left\vert f_{n} (x) \right\vert + \left\vert F(x) \right\vert \right)^p \leq (2G(x))^p.
				            \]
				            Since \(2G\in L^p\), so \(2G^p\in L^1\). Thus, by \hyperref[thm:dominated-convergence-theorem]{dominated convergence theorem}, we have
				            \[
					            \lim\limits_{N \to \infty} \int \left\vert \sum\limits_{n=1}^{N} f_{n} (x)- F(x) \right\vert ^p \,\mathrm{d} x = 0.
				            \]
				            This implies
				            \[
					            \left\lVert \sum\limits_{n=1}^{N} f_{n} - F\right\rVert _p \to 0
				            \]
				            as \(N\to \infty \).
			            \end{explanation}
		      \end{enumerate}
		\item Now assume \(1\leq p\leq \infty \). \todo{DIY}
	\end{itemize}
\end{proof}