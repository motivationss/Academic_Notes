\chapter{Introduction to Linear Programming}
\lecture{1}{30 Aug. 08:00}{Introduction}
\section{General Linear Programming Problem}
Let's start with the definition of a \hyperref[def:general-linear-programming-problem]{linear programming problem}.
\begin{definition}[General linear programming problem]\label{def:general-linear-programming-problem}
	A \emph{general linear programming problem} is to either minimize or maximize an \hyperref[def:objective-function]{objective function} with the
	\hyperref[def:constraints]{constraints} defined as follows.

	\begin{definition}[Objective function]\label{def:objective-function}
		An \emph{objective function} is in the form of
		\[
			c_1 x_1 + c_2 x_2 + \ldots +c_n x_n,
		\]
		where \(x_{i}\) are our variables, \(i = 1, \ldots, n\).
	\end{definition}

	\begin{definition}[Constraints]\label{def:constraints}
		The \emph{constraints} are the combination of \hyperref[def:structured-constraints]{structured constraints} and also the \hyperref[def:signed-constraints]{signed constraints}.
		\begin{definition}[Structured constraints]\label{def:structured-constraints}
			The so-called \emph{structured constraints} are in the form of
			\[
				\begin{alignedat}{4}
					&a_{11} x_1 + &&\ldots + &&a_{1n}x_n &&\gtreqqless b_1\\
					&a_{21} x_1 + &&\ldots + &&a_{2n}x_n &&\gtreqqless b_2\\
					&\vdots &&\ddots &&\vdots && \\
					&a_{n1} x_1 + &&\ldots + &&a_{nn}x_n &&\gtreqqless b_n\\
				\end{alignedat}.
			\]
		\end{definition}
		\begin{definition}[Signed constraints]\label{def:signed-constraints}
			The constraints
			\[
				x_1 \gtreqqless 0, x_2\gtreqqless 0, \ldots ,x_n\gtreqqless 0
			\]
			is called the \emph{signed constraints}.
		\end{definition}
	\end{definition}
\end{definition}

Given a \hyperref[def:general-linear-programming-problem]{general linear programming problem}, we have the following definitions.
\begin{definition}[Solution of a general linear programming problem]\label{def:solution-of-a-general-linear-programming-problem}
	We called an assignment of values to variable \(x\) as a \emph{solution}.

	\begin{definition}[Feasible solution]\label{def:feasible-solution}
		If this \hyperref[def:solution-of-a-general-linear-programming-problem]{solution} satisfies the \hyperref[def:constraints]{linear constraints}, we say that
		this \hyperref[def:solution-of-a-general-linear-programming-problem]{solution} is a \emph{feasible solution}.
	\end{definition}

	\begin{definition}[Feasible region]\label{def:feasible-region}
		The set of \hyperref[def:feasible-solution]{feasible solutions} is called \emph{feasible region}.
	\end{definition}

	\begin{definition}[Optimal solution]\label{def:optimal-solution}
		And a \hyperref[def:solution-of-a-general-linear-programming-problem]{solution} is an \emph{optimal solution} if there is no \hyperref[def:feasible-solution]{feasible solution}
		with better objective value.
	\end{definition}
\end{definition}

\begin{remark}
	A \hyperref[def:feasible-region]{feasible region} is a polyhedron.
\end{remark}


\begin{notation}
	We often referred \(\gtreqqless \) to either \(\geq , \leq\) or \(=\).
\end{notation}

We will denote
\[
	c= \begin{pmatrix}
		c_1    \\
		\vdots \\
		c_n    \\
	\end{pmatrix},\quad
	x = \begin{pmatrix}
		x_1    \\
		\vdots \\
		x_n    \\
	\end{pmatrix}, \quad
	A = \begin{pmatrix}
		a_{11} & \ldots & a_{1n} \\
		\vdots & \ddots & \vdots \\
		a_{n1} & \ldots & a_{nn} \\
	\end{pmatrix},\quad
	b = \begin{pmatrix}
		b_1    \\
		\vdots \\
		b_n    \\
	\end{pmatrix}.
\]

It's convenient to only consider linear programming in the \hyperref[def:standard-form]{standard form} defined as follows.

\begin{definition}[Standard form]\label{def:standard-form}
	The \emph{standard form} linear programming problem has the form of
	\begin{align*}
		\min~ & c^{\top}x \\
		      & Ax = b    \\
		      & x\geq 0
	\end{align*}
	with the condition that rows of \(A\) are linear independent, which means that no redundant equations  and the system is consistent.
\end{definition}

\begin{remark}[Compactness of the solution space]
	Notice that we only consider finitely many of \hyperref[def:constraints]{constrains}, since the property that the \hyperref[def:objective-function]{objective function}
	can attain its extremum only on a compact set, which requires finite dimensional vector space.
\end{remark}

\begin{remark}[Convert to standard form]
	Surprisingly, every \hyperref[def:general-linear-programming-problem]{general linear programming problem} can be converted to \hyperref[def:standard-form]{standard form},
	we now see how is this done.
\end{remark}
\begin{explanation}
	Given a \hyperref[def:general-linear-programming-problem]{general linear programming problem} with our notation, we have the following conversion.
	\begin{itemize}
		\item Sign:
		      \begin{itemize}
			      \item If \(x_{j}\leq 0 \implies x_j \to -x_j^-\), where \(x_{j}^- \geq 0\).
			      \item If \(x_{j}\) is unrestricted \(\implies x_j \to x_j^+ - x_{j}^-\), where \(x_j^{\pm} \geq 0\).
		      \end{itemize}
		\item Constraints:
		      \begin{itemize}
			      \item \(\sum\limits_{j=1}^{n} a_{ij} x_{j} \leq b \implies \sum\limits_{j=1}^{n} a_{ij} x_{j} + s_i = b_{i}\), where \(s_i \geq 0\).
			            \begin{definition}[Slack variable]\label{def:slack-variable}
				            This \(s_i\) sometimes called \emph{slack variable}.
			            \end{definition}
			      \item \(\sum\limits_{j=1}^{n} a_{ij} x_{j} \geq b \implies \sum\limits_{j=1}^{n} a_{ij} x_{j} - s_i = b_{i}\), where \(s_i \geq 0\).
			            \begin{definition}[Surplus variable]\label{def:surplus-variable}
				            This \(s_i\) sometimes called \emph{surplus variable}.
			            \end{definition}
		      \end{itemize}
		\item Maximize: \(\max \sum c_{j}x_{j} \implies -\min -\sum c_{j}x_{j}\).
	\end{itemize}
\end{explanation}