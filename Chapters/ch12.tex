\chapter{Connectivity}

\begin{definition}[connectedness]
    A metric space \(X\) is called \textbf{connected} if \(X\) cannot be represented as 
    the union of two disjoint nonempty open subsets. Thus \(X\) is connected if and only if 
    \[
        \nexists \: O_1, O_2 \subseteq X, \: \text{open, nonempty, with } O_1 \cap O_2 = \emptyset 
        \: \text{and } O_1 \cup O_2 = X     
    \]
    A subset \(M\) of \(X\) is called \textbf{connected} in \(X\) if \(M\) is connected 
    with respect to the metric induced from \(X\). 
\end{definition}

% \begin{example}
%     \begin{enumerate}[label=(\alph*)]
%         \item The empty set and any one-element set are connected. 
%         \item The set of the natural numbers \(\mathbb{N}\) is not connected. 
%         \item The set of rational numbers \(\mathbb{Q}\) is not connected in \(\mathbb{R}\).
%     \end{enumerate}
% \end{example}

\begin{proposition}
    For any metric space \(X\), the followings are equivalent: 
    \begin{itemize}
        \item \(X\) is connected. 
        \item \(X\) is the only nonempty subset of \(X\) which is both open and closed. 
    \end{itemize}
\end{proposition}

\begin{theorem} [Connectivity in \(\mathbb{R}\)]
    A subset of \(\mathbb{R}\) is connected if and only if it is an interval. 
\end{theorem}

\section{The Generalized Intermediate Value Theorem}

\begin{theorem}
    Let \(X\) and \(Y\) be metric spaces and \(f \colon X \to Y\) continuous. If \(X\)
    is connected, then so is \(f\left(X\right)\). That is, continuous images of connected
    sets are connected. 
\end{theorem}

\begin{corollary}
    Continuous images of intervals are connected. 
\end{corollary}

\begin{theorem}[Generalized Intermediate Value Theorem]
    Let \(X\) be a connected metric space and \(f \colon X \to \mathbb{R}\) continuous. 
    Then \(f\left(X\right)\) is an interval. In particular, \(f\) takes on every value 
    between any two given function values. 
\end{theorem}

\section{Path Connectivity}

\begin{definition}
    Let \(\alpha, \beta \in \mathbb{R}\) with \(\alpha < \beta\). A continuous function 
    \(w \colon [\alpha, \beta ] \to X\) is called a \textbf{continuous path} connecting 
    \textbf{end points} \(w\left(\alpha\right)\) and \(w\left(\beta\right)\). 
\end{definition}


\begin{definition}
    A metric space \(X\) is called \textbf{path connected} if, for each pair \(\left(x, y\right) \in X \times X\), 
    there is a continuous path in \(X\) connecting \(x\) and \(y\). A subset of a metric space 
    is called \textbf{path connected} if it is a path connected metric space with respect 
    to the induced metric. 
\end{definition}

\begin{proposition}
    Any path connected space is connected. 
\end{proposition}

Let \(E\) be a normed vector space and \(a, b \in E\). The linear structure of \(E\) allows us 
to consider 'straight' paths in \(E\): 
\[
    v \colon [0, 1] \to E, \:\:\: t \mapsto (1-t)a + tb  
\]
we denote the image of the path \(v\) by \([a, b] \)

A subset \(X\) of \(E\) is called \textbf{convex} if, for each pair 
\(\left(a, b\right) \in X \times X\), [\(a, b\)] is contained in \(X\). 

\begin{remark}
    Let \(E\) be a normed vector space. 
    \begin{enumerate}[label=(\alph*)]
        \item Every convex subset of \(E\) is path connected and connected. 
        \item For all \(a \in E\) and \(r > 0\), the balls \(\mathbb{B}_E\left(a, r\right)\) are convex. 
        \item A subset of \(\mathbb{R}\) is convex if and only if it is an interval. 
    \end{enumerate}
\end{remark}


\begin{theorem}
    Let \(X\) be a nonempty, open and connected subset of a normed vector space. 
    Then any pair of points of \(X\) can be connected by a polygonal path in \(X\). 
\end{theorem}

\begin{corollary}
    An open subset of a normed vector space is connected if and only if it is path connected. 
\end{corollary}








