\part{Continuous Functions}

\chapter{Continuity}

\section{Elementary Properties and Examples}

Let \(f \colon X \to Y \) be a function between metric spaces (\(X, d_X\)) and 
(\(Y, d_Y\)). Then \(f\) is \textbf{continuous} at \(x_0 \in X\) if, for each 
neighborhood \(V\) of \(f\left(x_0\right)\) in \(Y\), there is a neighborhood
\(U\) of \(x_0\) in \(X\) such that \(f\left(U\right) \subseteq V\). 


Hence to prove the continuity of \(f\) at \(x_0\), one supposes that an arbitrary 
neighborhood \(V\) of \(f\left(x_0\right)\) is given and then shows that there is 
a neighborhood \(U\) of \(x_0\) such that \(f\left(U\right) \subseteq V\), that is,
\(f\left(x\right) \in V\) for all \(x \in U\). 


The function \(f \colon X \to Y\) is \textbf{continuous} if it is continuous at each point
of \(X\). We say \(f\) is \textbf{discontinuous at} \(x_0\) if \(f\) is not continuous
at \(x_0\). \(f\) is \textbf{discontinuous} if it is discontinuous at (at least) one point
of \(X\). The set of all continuous functions from \(X\) to \(Y\) is denoted 
\(C \left(X, Y\right)\), a subset of \(Y^X\). 

\begin{proposition}
    A function \(f \colon X \to Y\) is continuous at \(x_0 \in X\) if and only if, for
    each \(\epsilon > 0\), there is some \(\delta \coloneqq \delta\left(x_0, \epsilon\right) > 0\)
    with the property that 
    \begin{center}
        \(d\left(f\left(x_0\right), f\left(x\right)\right) < \epsilon\) for all 
        \(x\in X\) such that \(d\left(x_0, x\right) < \delta\)
    \end{center}
\end{proposition}

\begin{corollary}
    Let \(E\) and \(F\) be normed vector spaces and \(X \subseteq E\). Then \(f \colon X \to F\)
    is continuous at \(x_0 \in X\) if and only if, for each \(\epsilon > 0\), there is 
    some \(\delta \coloneqq \delta\left(x_0, \epsilon\right) > 0\) satisfying that 
    \[
        \lVert f(x) - f(x_0) \rVert_F < \epsilon \:\:\:\:\: \text{for all } x \in X 
        \:\:\:\: \text{such that } \lVert x - x_0 \rVert_E < \delta 
    \]
\end{corollary}

\begin{eg}
    In the following examples, \(X\) and \(Y\) are metric spaces. 
    \begin{enumerate}[label=(\alph*)]
        \item The square root function \(\mathbb{R}^+ \to \mathbb{R}^+\), \(x \to \sqrt{x}\)
        is continuous. 
        \item The floor function \(\lfloor \cdot \rfloor \colon \mathbb{R} \to \mathbb{R}\),
        \(x \to \lfloor x \rfloor \coloneqq \text{max}\{k\in \mathbb{Z}; k \leq x\}\) is 
        continuous at \(x_0 \in \mathbb{R} \backslash \mathbb{Z} \) and discontinuous at 
        \(x_0 \in \mathbb{Z}\). 
        
        \item The \textbf{Dirichlet function } \(f \colon \mathbb{R} \to \mathbb{R}\) defined
        by 
        \[
           f(x) \coloneqq \begin{cases}
                1, & x \in \mathbb{Q}, \\ 
                0, & x \in \mathbb{R} \backslash \mathbb{Q}, 
           \end{cases}  
        \]
        is nowhere continuous, that is, it is discontinuous at every \(x_0 \in \mathbb{R}\). 

        \item Suppose that \(f \colon X \to \mathbb{R}\) is continuous at \(x_0 \in X\)
        and \(f\left(x_0\right) > 0\). Then there is a neighborhood \(U\) of \(x_0\) such 
        that \(f\left(x\right) > 0\) for all \(x \in U\). 
        \item A function \(f \colon X \to Y\) is \textbf{Lipschitz continuous} with 
        \textbf{Lipschitz constant } \(\alpha > 0\) if 
        \[
           d(f(x), f(y)) \leq \alpha d(x, y), \:\:\:\:\:\:\: x,y \in X
        \]  
        Every Lipschitz continuous function is continuous. 
        \item Any constant function \(X \to Y\), \(x \mapsto y_0\) is Lipschitz continuous. 
        \item The identity function \(\text{id} \colon X \to X\), \(x \mapsto x\) 
        is Lipschitz continuous.
        \item If \(E_1, \ldots, E_m\) are normed vector spaces, then 
        \(E \coloneqq E_1 \times \cdots \times E_m\) is a normed vector space with 
        respect to the product norm \(\lVert \cdot \rVert_\infty\). The canonical projections
        \[
           \text{pr}_k \colon E \to E_k, \:\:\:\:\: x = (x_1,\ldots,x_m) \mapsto x_k, 
           \:\:\:\: 1 \leq k \leq m, 
        \]
        are Lipschitz continuous. In particular, the projections \(\text{pr}_k \colon 
        \mathbb{K}^m \to \mathbb{K}\) are Lipschitz continuous. 
        \item Let \(E\) be a normed vector space. Then the norm function
        \[
           \lVert \cdot \rVert \colon E \to \mathbb{R}, \:\:\:\:\: x \mapsto \lVert x \rVert
        \]
        is Lipschitz continuous. 
        \item If \(A \subseteq X\) and \(f \colon X \to Y\) is continuous at \(x_0 \in A\),
        then \(f|A \colon A \to Y\) is continuous at \(x_0\). Here \(A\) has the metric 
        induced from \(X\). 
        \item Let \(M \subseteq X\) be a nonempty subset of \(X\). For each \(x \in X\), 
        \[
           d(x, M) \coloneqq \text{inf}_{m \in M} d(x,m)
        \]
        is called the \textbf{distance} from \(x\) to \(M\). The \textbf{distance function}
        \[
            d(\cdot, M) \colon X \to \mathbb{R}, \:\:\:\: x \mapsto d(x, M)   
        \]
        is Lipschitz continuous. 
        \item For any inner product space (\(E, \left(\cdot|\cdot\right)\)), the scalar
        product \(\left(\cdot|\cdot\right) \colon E \times E \to \mathbb{K}\) is continuous. 
        \item Let \(E\) and \(F\) be normed vector spaces and \(X \subseteq E\). Then the 
        continuity of \(f \colon X \to F\) at \(x_0 \in X\) is independent of the choice of 
        equivalent norms on \(E\) and on \(F\).  

        \item A function \(f\) between metric spaces \(X\) and \(Y\) is \textbf{isometric}
        (or an \textbf{isometry}) if \(d\left(f\left(x\right), f\left(x{}'\right)\right)
        = d\left(x, x{}'\right)\) for all \(x, x{}' \in X\), that is, if f 'preserves 
        distances'. Such a function is Lipschitz continuous and is a bijection from \(X\)
        to its image \(f\left(X\right)\). If \(E\) and \(F\) are normed vector spaces 
        and \(T \colon E \to F\) is linear, then \(T\) is isometric if and only if 
        \(\lVert Tx \rVert = \lVert x \rVert\) for all \(x \in E\). If, in addition, 
        \(T\) is surjective then \(T\) is an \textbf{isometric isomorphism} from \(E\)
        to \(F\), and \(T^{-1}\) is also isometric. 
    \end{enumerate}
\end{eg}

\section{Sequential Continuity}

\begin{definition}[Sequential Continuity]\label{def: sequential_conti}
    A function \(f \colon X \to Y\) between metric spaces \(X\) and \(Y\) is called 
    \textbf{sequentially continuous} at \(x \in X\), if, for every sequences (\(x_k\))
    in \(X\) such that \(\lim_{x_k} = x\), we have \(\lim_{f\left(x_k\right)} = f\left(x\right)\). 
\end{definition}

\begin{theorem}[sequence criterion]
    Let \(X, Y\) be metric spaces. Then a function \(f \colon X \to Y\) is continuous 
    at \(x\) if and only if it is sequentially continuous at \(x\). 
\end{theorem}

Let \(f \colon X \to Y\) be a continuous function between metric spaces. Then for any 
convergent sequence (\(x_k\)) in \(X\) we have 
\[
    \lim f(x_k) = f(\lim x_k)  
\]
That's why we say 'continuous functions respect the taking of limits'. 

\section{Addition and Multiplication of Continuous Functions}
\begin{proposition}
    Suppose that \(X\) is a metric space, \(F\) is a normed vector space, and 
    \[
        f \colon \text{dom}(f) \subseteq X \to F, \:\:\:\:\: 
        g \colon \text{dom}(g) \subseteq X \to F 
    \]
    are continuous at \(x_0 \in \text{dom}\left(f\right) \cap \text{dom}\left(g\right)\). 

    \begin{itemize}
        \item \(f + g\) and \(\lambda f\) are continuous at \(x_0\). 
        \item If \(F = \mathbb{K}\), then \(f \cdot g\) is continuous at \(x_0\). 
        \item If \(F = \mathbb{K}\) and \(g\left(x_0\right) \neq 0\), then 
        \(f / g\) is continuous at \(x_0\). 
    \end{itemize}
\end{proposition}

\begin{corollary}
    \begin{enumerate}[label=(\roman*)]
        \item Rational functions are continuous. 
        \item Polynomials in \(n\) variables are continuous (on \(\mathbb{K}^m\)). 
        \item \(C\left(X, F\right)\) is a subspace of \(F^X\), the \textbf{vector space
        of continuous functions} from \(X\) to \(F\). 
    \end{enumerate}
\end{corollary}

\begin{theorem}[continuity of compositions]
    Let \(X, Y\) and \(Z\) be metric spaces. Suppose that \(f \colon X \to Y\) is 
    continuous at \(x \in X\), and \(g \colon Y \to Z\) is continuous at 
    \(f\left(x\right) \in Y\). Then the composition \(g \circ f \colon X \to Z\)
    is continuous at \(x\). 
\end{theorem}

\begin{eg}
    Let \(X\) be a metric space and \(E\) be a normed vector space. \\ 
   
    Let \(f \colon X \to E\) be continuous at \(x_0\). Then the \textbf{norm of } \(f\),
    \[
        \lVert f \rVert \colon X \to \mathbb{R}, \:\:\:\:\:\: x \mapsto \lVert f(x) \rVert ,
    \]
    is continuous at \(x_0\). 
\end{eg}

\section{One-Sided Continuity}

Let \(X\) be a subset of \(\mathbb{R}\) and \(x_0 \in X\). For \(\delta > 0\), the set
\(X \cap (\ x_0 - \delta, x_0 ]\  \) (or \(X \cap [\ x_0, x_0 + \delta )\ \) ) is called 
a \textbf{left} (or \textbf{right}) \(\delta\)-\textbf{neighborhood} of \(x_0\). 

Let \(Y\) be a metric space. Then \(f \colon X \to Y\) is \textbf{left} (or \textbf{right})
\textbf{continuous} at \(x_0\), if, for each neighborhood \(V\) of \(f\left(x_0\right)\)
in \(Y\), there is some \(\delta > 0\) such that \(f\left(X \cap (\ x_0 - \delta, x_0  ]\ \right)
\subseteq V\) (or \(f\left(X \cap [\ x_0, x_0 + \delta  )\ \right) \subseteq V\)). 

