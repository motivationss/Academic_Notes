
\chapter{Rings, Fields and Polynomials}

\section{Rings}

\begin{definition}[Ring]\label{def: ring}
    A triple (\(R, +, \cdot\)) consisting of a nonempty set \(R\) and operations, \textbf{addition}
    \(+\) and \textbf{multiplication} \(\cdot\), is called a \textbf{ring} if 
    \begin{itemize}
        \item (\(R, +\)) is an Abelian group
        \item Multiplication is associative
        \item The \textbf{distributive law} holds:
        \[
            (a + b) \cdot c = a \cdot c + b \cdot, \:\: c \cdot (a + b) = c \cdot a + c \cdot b, \:\:\: a,b,c \in R  
        \]
    \end{itemize}
\end{definition}

\begin{note}
    A ring is called \textbf{commutative} if multiplication is commutative. \\
    If there is an identity element with respect to multiplication, then it is written as 
    \(1_R\) or simply 1, and is called the \textbf{unity} (or \textbf{multiplicative identity})
    of \(R\), and we say (\(R, +, \cdot\)) is a \textbf{ring with unity}. \\
    When the addition and multiplication operations are clear from context, we write simply
    \(R\) instead of (\(R, +, \cdot\)).
\end{note}

\begin{eg}
    \begin{enumerate}[label=(\alph*)]
        \item The \textbf{trivial ring} has exactly one element 0 and is itself denoted by 0.
        A ring with more than one element is \textbf{nontrivial}. If \(R\) is a ring with unity,
        then it follows from \(1_R \cdot a = a\) for each \(a \in R\), that \(R\) is trivial 
        if and only if \(1_R == 0_R\).
        \item Suppose \(R\) is a ring and \(S\) is a nonempty subset of \(R\) that satisfies the
        following:
        \begin{itemize}
            \item \(S\) is a subgroup of (\(R, +\)). 
            \item \(S \cdot S \subseteq S\)
        \end{itemize}
        Then \(S\) itself is a ring, a \textbf{subring} of \(R\), and \(R\) is called an \textbf{overring}
        of \(S\). If \(R\) is commutative then so is \(S\), but the converse is not true in general.
        \item Intersections of subrings are subrings.
    \end{enumerate}
\end{eg}

\begin{definition}[Ring Homomorphism]\label{def: ring_homomorphism}
    Let \(R\) and \(R{}'\) be rings. A (\textbf{ring}) \textbf{homomorphism} is a function
    \(\varphi \colon R \rightarrow R{}'\) which is compatible with the ring operations, that is,
    \begin{equation}
        \label{eqn: ring_homo}
        \varphi(a + b) = \varphi(a) + \varphi(b), \:\:\:\:\: \varphi(ab) = \varphi(a)\varphi(b), \:\: a,b \in R
    \end{equation}
\end{definition}

\begin{note}
    If, in addition, \(\varphi\) is bijective, then \(\varphi\) is called a (\textbf{ring}) \textbf{isomorphism}
    and \(R\) and \(R{}'\) are \textbf{isomorphic}.\\
    A homomorphism \(\varphi\) from \(R\) to itself is a (\textbf{ring}) \textbf{endomorphism}. If \(\varphi\)
    is an isomorphism, then it is a (\textbf{ring}) \textbf{automorphism}.
\end{note}

\begin{eg}
    \begin{enumerate}[label=(\alph*)]
        \item A ring homomorphism \(\varphi \colon R \rightarrow R{}'\) is, in particular, a
        group homomorphism from (\(R, +\)) to (\(R{}', +\)). The \textbf{kernel}, ker(\(\varphi\)),
        of \(\varphi\) is defined to be the kernel of this group homomorphism, that is,
        \[
          \text{ker}(\varphi) = \{ a \in R ; \varphi(a) = 0 \} = \varphi^{-1}(0)  
        \]
        \item The \textbf{zero function} \(R \rightarrow R{}'\), \(a \mapsto 0_{R{}'}\) is a
        homomorphism with ker(\(\varphi\)) = \(R\). 
        \item Let \(R\) and \(R{}'\) be rings with unity and \(\varphi \colon R \rightarrow R{}'\)
        a homomorphism. As (b) shows, it does not necessarily follow that \(\varphi\left(1_R\right) = 1_{R{}'}\). 
        This can be seen as a consequence of the fact that, with respect to multiplication, a ring 
        is not a group.
    \end{enumerate}
\end{eg}

\section{Consequence of Ring Definitions}

\begin{definition}[The Binomial Theorem]\label{def: binomial_theorem}
    Let \(a\) and \(b\) be two commuting elements (\(ab = ba\)) of a ring \(R\) with unity.
    Then, for all \(n \in \mathbb{N}\),
    \[
        (a + b)^b = \sum^n_{k=0} \binom{n}{k}a^kb^{n-k}  
    \]
\end{definition}

\begin{lemma}
    For \(m \in \mathbb{N}\) with \(m \geq 2\), an element \(\alpha = \left(\alpha_1\ldots\alpha_m\right)
    \in \mathbb{N}^m\) is called a \textbf{multi-index}. The \textbf{length} \(\left | \alpha \right |\)
    of a multi-index \(\alpha \in \mathbb{N}^m\) is defined by 
    \[
        \left |  \alpha   \right |  \coloneqq \sum^m_{j=1}\alpha_j
    \]
    Set also 
    \[
        \alpha! \coloneqq \prod^m_{j=1} (\alpha_j)!,    
    \]
    and define the \textbf{natural} (\textbf{partial}) \textbf{order} on \(\mathbb{N}^m\) by
    \[
        \alpha \leq \beta \rightleftarrows (\alpha_j \leq \beta_j, \:\: 1 \leq j \leq m).   
    \]
    for \(a = \left(a_1, \ldots, a_m\right) \in R^m \) and \(\alpha = \left(\alpha_1,\ldots,\alpha_m\right)\in \mathbb{N}^m\)

\end{lemma}

\begin{definition}[The Multinomial Theorem]\label{def: multinomial_theorem}
    Let \(R\) be a commutative ring with unity. Then for all \(m \geq 2\),
    \[
      (\sum_{j=1}^m \alpha)^k = \sum_{ \left | \alpha \right | = k } \frac{k!}{\alpha!} a^{\alpha},
      \:\:\: a = (a_1, \ldots, a_m) \in R^m, \:\: k \in \mathbb{N}.  
    \]
\end{definition}

\section{Field}

\begin{definition}[Field]\label{def: field}
    \(K\) is a \textbf{field} when the following are satisfied:
    \begin{itemize}
        \item \(K\) is a commutative ring with unity.
        \item 0 \(\neq\) 1
        \item \(K^{\times} \coloneqq K \backslash \{0\} \) is an Abelian group with respect to multiplication.
    \end{itemize}
\end{definition}

\begin{note}
    The Abelian group \(K^{\times} \coloneqq \left(K^{\times}, \cdot\right)\) is called the 
    \textbf{multiplicative group} of \(K\).
\end{note}

\begin{remark}
    Let \(K\) be a field.

    \begin{enumerate}[label=(\alph*)]
        \item For all \(a \in K^{\times}\), \(\left(a^{-1}\right){-1} = a\)
        \item A field has no zero divisors
        \item Let \(a \in K^{\times}\) and \(a \in K^{\times}\) and \(b \in K\). Then there 
        is an unique \(x \in K\) with \(ax = b\), namely the \textbf{quotient} 
        \(\frac{b}{a} \coloneqq b/a \coloneqq ba^{-1}\)
        \item Let \(K{}'\) be a field and \(\varphi \colon K \rightarrow K{}'\) a 
        homomorphism with \(\varphi \neq 0\). Then 
        \[
            \varphi(1_K) = 1_{K{}'} \:\:\: \text{   and   } \:\:\: \varphi(a^{-1}) = \varphi(a)^{-1}, \:\: a \in K^\times  
        \]
    \end{enumerate}
\end{remark}

\section{Ordered Field}

\begin{definition}[Ordered Ring]\label{def: ordered_ring}
    A ring \(R\) with an ordered \(\leq\) is called an \textbf{ordered ring} if the following holds:
    \begin{itemize}
        \item (\(R, \leq\)) is totally ordered. 
        \item \(x < y \Rightarrow  x + z < y + z, z \in R\)
        \item \(x, y > 0 \Rightarrow  xy>0\)
    \end{itemize}
\end{definition}

\begin{note}
    This leads to a series of basic arithmetic rules. \par 
    We may define absolute value function from \(K \mapsto K\). 
\end{note}

\begin{proposition}
    Let \(K\) be an ordered field and \(x, y, a, \epsilon \in K \) with \(\epsilon > 0\). 
    \begin{enumerate}[label=(\roman*)]
        \item \(x = |x|\text{sign}\left(x\right)\), \(|x| = x\text{sign}\left(x\right)\)
        \item \(|x| = |- x|, x \leq |x|\)
        \item \(|xy| = |x||y|\)
        \item \(|x| \geq 0\) and (\(|x| = 0 \Leftrightarrow x = 0\))
        \item \(|x - a| < \epsilon \leftrightarrow a - \epsilon < x < a + \epsilon \)
        \item \(|x + y| \leq |x| + |y|\) (\textbf{triangular inequality})
    \end{enumerate}
\end{proposition}

\begin{corollary}[reversed triangular inequality]\label{cor: reversed_tri_inequality}
    In any ordered field \(K\) we have
    \[
        |x - y| \geq \left | \left | x \right | - \left | y \right | \right |, \:\:\:\:\: x,y \in K.  
    \]
\end{corollary}


\section{Formal Power Series}

\begin{definition}[formal power series]\label{def: formal_power_series}
    Let \(R\) be a nontrivial ring with unity. On the set \(R^{\mathbb{N}} = \text{Funct}\left(\mathbb{N}, R\right)\)
    define addition by 
    \[
        (p + q)_n \coloneqq p_n + q_n, \:\:\:\:\:\:\: n \in \mathbb{N},    
    \]
    and multiplication by \textbf{convolution},
    \[
        (pq)_n \coloneqq (p \cdot q)_n \coloneqq \sum_{j=0}^n p_j q_{n-1} = p_0q_n + p_1q_{n-1}+ \cdots + p_nq_0
    \]
    for \(n \in \mathbb{N}\). Here \(p_n\) denotes the value of \(p \in R^\mathbb{N}\) at \(n \in \mathbb{N}\) and is 
    called the \(n^{\text{th}}\) \textbf{coefficient} of \(p\). In this situation an element \(p \in R^\mathbb{N}\) is 
    called a \textbf{formal power series over} \(R\), and we set \(R \left[  X \right] \coloneqq \left(R^\mathbb{N}, +, \cdot\right) \)
    
\end{definition}

\begin{proposition}
    \(R \left[ X \right] \) is a ring with unity, the \textbf{formal power series ring over} \(R\). 
    If \(R\) is commutative, then so is \(R \left[X\right]\)
\end{proposition}

\section{Polynomials}

\begin{definition}[Polynomial]\label{def: polynomials}
    A \textbf{polynomial over} \(R\) is a formal power seres \(p \in R \left[X\right]\)
    such that \( \{n; p_n \neq 0 \} \) is finite, in other words, \(p_n = 0\) "almost everywhere". 
\end{definition}