\chapter{Compactness}
We see that continuous images of open sets may not be open, and continuous images of closed 
sets may not be closed. This chapter investigates certain properties that are preserved by 
continuity. 

\section{Covers}

In the following, \(X \coloneqq \left(X, d\right)\) is a metric space. 

A family of sets \(\{ A_\alpha \subseteq X; \alpha \in \mathcal{A} \}\) is called a 
\textbf{cover} of the subset \(K \subseteq X\) if \(K \subseteq \bigcup_\alpha A_\alpha\). 
A cover is called \textbf{open} if each \(A_\alpha\) is open in \(X\). A subset \(K \subseteq X\)
is called \textbf{compact} if every open cover of \(K\) has a finite subfamily which is also 
a cover of \(K\). In other words, \(K \subseteq X\) is compact if every open cover of \(K\)
has a \textbf{finite subcover}. 

\begin{eg}
    \begin{enumerate}[label=(\alph*)]
        \item Let (\(x_k\)) be a convergent sequence in \(X\) with limit \(a\). Then the 
        set \(K \coloneqq \{a\} \cup \{x_k ; k \in \mathbb{N}\} \) is compact. 
        \item The state of (a) is false, in general, if the limit \(a\) is not included in \(K\). 
        \item The set of natural numbers \(\mathbb{N}\) is not compact in \(\mathbb{R}\). 
    \end{enumerate}
\end{eg}

\begin{proposition}
    Any compact set \(K \subseteq X\) is closed and bounded in \(X\). 
\end{proposition}

\section{A Characterization of Compact Set}

% \begin{definition}
%     A subset \(K\) of \(X\) is \textbf{totally bounded} if, for each \(r > 0\), there 
%     are \(m \in \mathbb{N}\) and \(x_0,\ldots,x_m \in K\) such that \(K \subseteq \bigcup_{k=0}^m \mathbb{B}\left(x_k, r\right)\). 
% \end{definition}

\begin{definition}
    A subset \(K\) of \(X\) is \textbf{totally bounded} if, for each \(r > 0\), there are 
    \(m \in \mathbb{N}\) and \(x_0, \ldots, x_m \in K\) such that \(K \subseteq \bigcup_{k=0}^m \mathbb{B}\left(x_k,r\right) \). 
\end{definition}


\begin{theorem}
    A subset \(K \subseteq X\) is compact if and only if every sequence in \(K\) has 
    a cluster point \(K\). 
\end{theorem}


\section{Sequential Compactness}

\begin{definition}
    A subset \(K \subseteq X\) is \textbf{sequentially compact} if every sequence in \(K\)
    has a subsequence which converges to an element of \(K\). 
\end{definition}

\begin{theorem}
    A subset of a metric space is compact if and only if it is sequentially compact. 
\end{theorem}

\begin{theorem}[Heine-Borel]
    A subset of \(\mathbb{K}^n\) is compact if and only if it is closed and bounded. 
    In particular, an interval is compact if and only if it is closed and bounded. 
\end{theorem}

\section{Continuous Functions on Compact Spaces}

\begin{theorem}
    Let \(X\) and \(Y\) be metric spaces and \(f \colon X \to Y\) continuous. If \(X\)
    is compact, then \(f\left(X\right)\) is compact. That is, continuous images of compact
    sets are compact. 
\end{theorem}

\begin{corollary}
    Let \(X\) and \(Y\) be metric spaces and \(f \colon X \to Y\). If \(X\) is compact, 
    then \(f \left(X\right)\) is bounded. 
\end{corollary}

\begin{theorem}[extreme value theorem]
    Let \(X\) be a compact metric space and \(f \colon X \to \mathbb{R}\) continuous. 
    Then there are \(x_0, x_1 \in X\) such that 
    \[
        f (x_0) = \text{min}_{x \in X} f(x) \:\:\:\: \text{and} \:\:\:\: f(x_1) = \text{max}_{x\in X}f(x)  
    \]
\end{theorem}


\section{Total Boundedness}

\begin{theorem}
    A subset of a metric space is compact if and only if it is complete and totally bounded. 
\end{theorem}

\section{Uniform continuity}

\begin{definition}
    Let \(X\) and \(Y\) be metric spaces and \(f \colon X \to Y\). Then \(f\) is called 
    \textbf{uniformly continuous} if, for each \(\epsilon > 0\), there is some 
    \(\delta \left(\epsilon\right) > 0\) such that 
    \[
        d(f(x), f(y)) < \epsilon \:\:\:\: \text{for all } x,y \in X \: \text{such that } d(x, y) < \delta(\epsilon). 
    \]
\end{definition}

\begin{eg}
    Lipschitz continuous functions are uniformly continuous. 
\end{eg}

\begin{theorem}
    Suppose that \(X\) and \(Y\) are metric spaces with \(X\) compact. If \(f \colon X \to Y\)
    is continuous, then \(f\) is uniformly continuous. That is, continuous functions on 
    compact sets are uniformly continuous. 
\end{theorem}

\section{Compactness in General Topological Spaces}

\begin{remark}
    \begin{enumerate}[label=(\alph*)]
        \item Let \(X = \left(X, \tau\right)\) be a topological space. Then \(X\) is 
    \textbf{compact} if \(X\) is a Hausdorff space and every open cover of \(X\) has a finite
    subcover. The space \(X\) is \textbf{sequentially compact} if it is a Hausdorff space 
    and every sequence has a convergent subsequence. A subset \(Y \subseteq X\) is \textbf{compact}
    (or \textbf{sequentially compact}) if the topological subspace (\(Y, \tau_Y\)) is compact
    (or sequentially compact). 
        
        \item Any compact subset \(K\) of a Hausdorff space \(X\) is closed. For each \(x_0 \in K^c\)
        there are disjoint open sets \(U\) and \(V\) in \(X\) such that \(K \subseteq U\) and 
        \(x_0 \in V\). In other words, a compact subset of a Hausdorff space and a point, not 
        in that subset, can be separated by open neighborhoods. 
        
        \item Any closed subset of a compact space is compact. 
        \item Let \(X\) be compact and \(Y\) Hausdorff. Then the image of any continuous 
        function \(f \colon X \to Y\) is compact. 
        \item In general topological spaces, compactness and sequential compactness are 
        distinct concepts. That is, a compact space need not to be sequentially compact, and 
        a sequentially compact space need not be compact. 
        \item Uniform continuity is undefined in general topological spaces since the definition 
        above makes essential use of the metric.  
    \end{enumerate}
\end{remark}


