\chapter{The Fundamentals of Topology}

\section{Open Sets}
\begin{definition}
    Let \(X \coloneqq \left(X, d\right)\) be a metric space. An element \(a\) of a subset
    \(A\) of \(X\) is called an \textbf{interior point} of \(A\) if there is a neighborhood
    \(U\) of \(a\) such that \(U \subseteq A\). The set \(A\) is called \textbf{open} 
    if every point of \(A\) is an interior point.    
\end{definition}

\begin{eg}
    The open ball \(\mathbb{B}\left(a, r\right)\) is open. 
\end{eg}

\begin{remark}
    \begin{enumerate}[label=(\alph*)]
        \item The concepts "interior point" and "open set" depend on the surrounding metric 
        space \(X\). It is sometimes useful to make this explicit by saying that '\(a\) is 
        an interior point of \(A\) with respect to \(X\)', or '\(A\) is open in \(X\)'. 
        
        \item If \(A\) is open with respect to a particular norm, it is open with respect 
        to all equivalent norms. 
        \item Every point in a metric space has an open neighborhood
    \end{enumerate}
\end{remark}

\begin{proposition}
    Let \(\tau \coloneqq \{O \subseteq X; \:\: O \:\: \text{is open}\} \) be a family of open sets. 
    \begin{enumerate}[label=(\roman*)]
        \item \(\emptyset, X \in \tau\)
        \item If \(O_\alpha \in \tau\) for all \(\alpha \in \mathcal{A}\), then \(\bigcup_\alpha O_\alpha \in \tau\). 
        That is, arbitrary unions of open sets are open. 
        \item If \(O_0, \ldots, O_n \in \tau\), then \(\bigcap^n_{k=0}O_k \in \tau\). That is,
        finite intersections of open sets are open. 
    \end{enumerate}
\end{proposition}

Let \(M\) be a set and \(\tau \subseteq \mathcal{P}\left(M\right)\), a set of subset satisfying
(i)-(iii). Then \(\tau\) is called a \textbf{topology} on \(M\), and the elements of \(\tau\)
are called the \textbf{open sets} with respect to \(\tau\). The pair (\(M, \tau\)) is called 
a \textbf{topological space}. 

\begin{remark}
    \begin{enumerate}[label=(\alph*)]
        \item Let \(\tau \in \mathcal{P}\left(X\right)\) be the family of open sets. 
        Then \(\tau\) is called the \textbf{topology on} \(X\) \textbf{induced from the metric} \(d\). 
        If \(X\) is a normed vector space with metric induced from the norm, then \(\tau\) is called
        the \textbf{norm topology}. 
        \item  Let (\(X, \lVert \cdot \rVert\)) be a normed vector space, and \(\lVert \cdot \rVert_1\)
        a norm on \(X\) which is equivalent to \(\lVert \cdot \rVert\). Let \(\tau_{\lVert \cdot \rVert}\)
        and \(\tau_{\lVert \cdot \rVert_1}\) be the norm topologies induced from (\(X, \lVert \cdot \rVert\))
        and (\(X, \lVert \cdot \rVert_1\)). Then they coincide, that is, equivalent norms induce the 
        same topology on \(X\). 
    \end{enumerate}
\end{remark}

\section{Closed Sets}

A subset \(A\) of the metric space \(X\) is called \textbf{closed} in \(X\) if \(A^c\) is open 
in \(X\). 

\begin{proposition}
    \begin{enumerate}[label=(\roman*)]
        \item \(\emptyset\) and \(X\) are closed. 
        \item Arbitrary intersections of closed sets are closed. 
        \item Finite unions of closed sets are closed. 
    \end{enumerate}
\end{proposition}

\begin{remark}
    \begin{itemize}
        \item Infinite intersections of open sets need not be open. 
        \item Infinite unions of closed sets need not be closed. 
    \end{itemize}
\end{remark}

\begin{definition}
    Let \(A \subseteq X\) and \(x \in X\). We call \(x\) an \textbf{accumulation point}
    of \(A\) if every neighborhood of \(x\) in \(X\) has a nonempty intersection with \(A\). 
    The element \(x \in X\) is called a \textbf{limit point} of \(A\) if every neighborhood
    of \(x\) in \(X\) contains a point of \(A\) other than \(x\). We set 
    \[
        \overline{A} \coloneqq \{ x \in X; \:\: x \:\:\: \text{is an accumulation point of } A \}.   
    \]
    Clearly any element of \(A\) and any limit point of \(A\) is an accumulation point of \(A\). 
\end{definition}

\begin{proposition}
    Let \(A\) be a subset of a metric space \(X\). 
    \begin{enumerate}[label=(\roman*)]
        \item \(A \subseteq \overline{A}\)
        \item \(A = \overline{A} \Leftrightarrow A\) is closed. 
    \end{enumerate}
\end{proposition}

\begin{proposition}
    An element \(x\) of \(X\) is a limit point of \(A\) if and only if there is a sequence 
    (\(x_k\)) in \(A \backslash \{x\}\) which converges to \(x\). 
\end{proposition}

\begin{corollary}
    An element \(x\) of \(X\) is an accumulation point of \(A\) if and only if there is 
    a sequence (\(x_k\)) in \(A\) such that \(x_k \to x\). 
\end{corollary}


\begin{proposition}
    For \(A \subseteq X\), the followings are equivalent:
    \begin{itemize}
        \item \(A\) is closed. 
        \item \(A\) contains all its limit points. 
        \item Every sequence in \(A\) which converges in \(X\), has its limits in \(A\). 
    \end{itemize}
\end{proposition}


\section{The Closure and Interior of a Set}

\begin{definition}
    Let \(A\) be a subset of a metric space \(X\). Define the \textbf{closure of } \(A\) by 
    \[
       \text{cl}(A) \coloneqq \text{cl}_X(A) \coloneqq \bigcap_{B \in M} B   
    \]
    with 
    \[
        M \coloneqq \{B \subseteq X; B \supseteq  A \:\:\: \text{and } B \:\:\: \text{is closed in } X \}.   
    \]. 
\end{definition}
\begin{remark}
    The closure of \(A\) is the smallest closed set which contains \(A\). Any closed set which 
    contains \(A\), also contains cl(\(A\)). 
\end{remark}

\begin{proposition}
    Let \(A\) be a subset of a metric space \(X\). Then \(\overline{A} = \text{cl}\left(A\right)\). 
\end{proposition}

\begin{corollary}
    Let \(A\) and \(B\) be subsets of \(X\). 
    \begin{enumerate}[label=(\roman*)]
        \item \(A \subseteq B \implies \overline{A} \subseteq \overline{B}\). 
        \item \(\overline{\left(\overline{A}\right)} = \overline{A}\). 
        \item \(\overline{A \cup B} = \overline{A} \cup \overline{B}\). 
    \end{enumerate}
\end{corollary}

\begin{definition}[Interior of a Set]
    \[
      \text{int}(A) \coloneqq \text{int}_X (A) \coloneqq \bigcup \{O \subseteq A;O  \text{
        is open in X
      }\}  
    \]
    int(\(A\)) is the largest open subset of \(A\). 
\end{definition}

\begin{definition}
    \[
        \mathring{A} \coloneqq \{ a \in A ; a \text{ is an interior point of } A\}.   
    \]
\end{definition}

\begin{proposition}
    Let \(A\) be a subset of a metric space \(X\). Then \(\mathring{A} = \text{int}\left(A\right)\). 
\end{proposition}

\begin{corollary}
    Let \(A\) and \(B\) be subsets of \(X\). 
    \begin{enumerate}[label=(\roman*)]
        \item \(A \subseteq B \Rightarrow \mathring{A} \subseteq \mathring{B}\). 
        \item \(\left(\mathring{A}\right)^\circ = \mathring{A}\). 
        \item \(A\) is open \(\Leftrightarrow A = \mathring{A}\). 
    \end{enumerate}
\end{corollary}

\section{The Hausdorff Condition}

\begin{definition}
    For a subset \(A\) of a metric space \(X\), the (topological) \textbf{boundary of}
    \(A\) is defined by \(\partial A \coloneqq \overline{A}\backslash \mathring{A}\). 
\end{definition}

\begin{proposition}
    Let \(A\) be a subset of \(X\). 
    \begin{itemize}
        \item \(\partial A \) is closed. 
        \item \(x\) is in \(\partial A\) if and only if every neighborhood of \(x\) has 
        nonempty intersection with both \(A\) and \(A^c\). 
    \end{itemize}
\end{proposition}

The following proposition shows that, in metric spaces, any two distinct points have disjoint 
neighborhoods. 
\begin{proposition}[Hausdorff Condition]
    Let \(x, y \in X\) be such that \(x \neq y\). Then there are a neighborhood \(U\)
    of \(x\) and a neighborhood \(V\) of \(y\) such that \(U \cap V = \emptyset\). 
\end{proposition}

\begin{corollary}
    Any one element subset of a metric space is closed. 
\end{corollary}

\section{A Characterization of Continuous Functions}

\begin{theorem}
    Let \(f \colon X \to Y \) be a function between metric spaces \(X\) and \(Y\). Then the followings
    are equivalent:
    \begin{enumerate}[label=(\roman*)]
        \item \(f\) is continuous. 
        \item \(f^{-1}\left(O\right)\) is open in \(X\) for each open set \(O\) in \(Y\). 
        \item \(f^{-1}\left(A\right)\) is closed in \(X\) for each closed set \(A\) in \(Y\). 
    \end{enumerate}
\end{theorem}

\begin{remark}
    According to this theorem, a function is continuous if and only if the preimage of any 
    open set is open, if and only if the preimage of any closed set is closed. We denote 
    the topology of a metric space \(X\) by \(\tau_X\), that is, 
    \[
        \tau_X \coloneqq \{ O \subseteq X ; O \text{ is open in } X \}  
    \]
    Then, 
    \[
        f \colon X \to Y \text{ is continuous } \Longleftrightarrow f^{-1} \colon \tau_Y \to \tau_X  
    \]
\end{remark}

\section{Continuous Extensions}

Let \(X\) and \(Y\) be metric spaces. Suppose that \(D \subseteq X\), \(f \colon D \to Y\) is 
continuous and \(a \in X\) is a limit point of \(D\). If \(D\) is not closed, then \(a\) may 
not be in \(D\) and so \(f\) is not defined at \(a\). In this section we consider whether 
\(f\left(a\right)\) can be defined so that \(f\) is continuous on \(D \cup \{a\}\). If such 
extensions exist, then, for any sequence (\(x_n\)) in \(D\) which converges to \(a\), 
\(\left(f\left(x_n\right)\right)\) converges to \(f\left(a\right)\). 

\begin{remark}
    \begin{enumerate}[label=(\alph*)]
        \item The followings are equivalent:
        \begin{itemize}
            \item \(\lim_{x\to a}f\left(x\right) = y\). 
            \item For each neighborhood \(V\) of \(y\) in \(Y\), there is a neighborhood \(U\)
            of \(a\) in \(X\) such that \(f\left(U \cap D\right) \subseteq V\). 
        \end{itemize}
        \item If \(a \in D\) is a limit point of \(D\), then 
        \[
            \lim_{x\to a} f(x) = f(a) \Longleftrightarrow f \text{ is continuous at } a.  
        \]
    \end{enumerate}
\end{remark}

\section{Relative Topology}

Let \(X\) be a metric space and \(Y\) a subset of \(X\). Then \(Y\) is itself a metric space
with respect to the metric \(d_Y \coloneqq d|Y \times Y\) induced from \(X\), and so 'open 
in (\(Y, d_Y\))' and 'closed in (\(Y, d_Y\))' are well-defined concepts. 

\begin{proposition}
    Let \(X\) be a metric space and \(M \subseteq Y \subseteq X\). Then \(M\) is open (or closed)
    in \(Y\) if and only if \(M\) is open (or closed) in (\(Y, d_Y\)). 
\end{proposition}

\begin{corollary}
    If \(M \subseteq Y \subseteq X\), then \(M\) is open in \(Y\) if and only if \(Y\backslash M\)
    is closed in \(Y\). 
\end{corollary}




