\chapter{Diagonalization}
\lecture{7}{22 Sep. 12:30}{Spectral Theorem}
Our goal is to do network partition, and to do so, we now develop a power tool to decompose a matrix. Later, we'll see that by decomposing a
special matrix called \hyperref[def:Laplacian]{Laplacian} which characterize a network, we'll get a corresponding decomposition on the network
in an insightful way.

\section{Eigenvalue and Eigenvector}
We now want to study a very useful theorem called \hyperref[thm:spectral-theorem]{spectral theorem}. Before this, we need to study \hyperref[def:eigenvalue]{eigenvalue}.

\begin{definition*}
	For a matrix \(A\), if \(v\) is a nonzero vector such that we can write
	\[
		Av = \lambda v,
	\]
	then we call \(v\) the \hyperref[def:eigenvector]{eigenvector} while \(\lambda\) the \hyperref[def:eigenvalue]{eigenvalue}.
	\begin{definition}[Eigenvalue]\label{def:eigenvalue}
		The \emph{eigenvalue} is the scalar multiple \(\lambda \) as described above.
	\end{definition}

	\begin{definition}[Eigenvector]\label{def:eigenvector}
		The \emph{eigenvector} is the nonzero vector \(v\) as described above.
	\end{definition}
\end{definition*}

\begin{remark}
	Indeed, \hyperref[def:eigenvector]{eigenvector} and \hyperref[def:eigenvalue]{eigenvalue} is defined in a more general sense, i.e., for linear transformation \(T\) such that
	\[
		T(u) = \lambda u.
	\]
	This is a generalization since a linear transformation can come from an infinite-dimensional vector space, which don't have a vector representation in this case.
\end{remark}

\begin{note}[Left eigenvector]
	The notion of \hyperref[def:eigenvector]{eigenvector} is indeed also called a \emph{right \hyperref[def:eigenvector]{eigenvector}}. As for the
	\emph{left \hyperref[def:eigenvector]{eigenvector}}, it's the similar definition but with transpose. Rigorously, for a matrix \(A\), the
	\emph{left \hyperref[def:eigenvector]{eigenvector}} is the nonzero vector \(v\) which satisfies
	\[
		v^{\top} A = \lambda v^{\top}.
	\]
	But for simplicity, we do not make this into a definition.
\end{note}


\begin{note}[Characteristic equation]\label{note:characteristic-equation}
	To find \hyperref[def:eigenvalue]{eigenvalues} of a matrix \(A\), this is equivalent to solve \(Av = \lambda v\) where \(v\) is undetermined. We can indeed use
	the so-called \emph{characteristic equation}
	\[
		\det(\lambda I - A) = 0
	\]
	to help us solve for \(\lambda \).
\end{note}

\section{Spectral Theorem}
With some assumption on structural properties of a matrix, we can say something about its eigenvalue. Hence, we start with the so-called
\hyperref[def:symmetric-matrix]{symmetric matrices}.

\begin{definition}[Symmetric matrix]\label{def:symmetric-matrix}
	A \emph{symmetric matrix} is a square matrix such that
	\[
		A = A^{\top}.
	\]
\end{definition}

With this assumption, we have the following strong theorem which we'll heavily rely on.

\begin{theorem}[Spectral Theorem]\label{thm:spectral-theorem}
	If \(A\) is a \hyperref[def:symmetric-matrix]{symmetric matrix}, then all \hyperref[def:eigenvalue]{eigenvalues} are real. Furthermore, the left and right
	\hyperref[def:eigenvector]{eigenvectors} are the same, and they construct an orthonormal basis.
\end{theorem}
\begin{proof}
	We verify the second part of the theorem.  If we have \(A\) is \hyperref[def:symmetric-matrix]{symmetric}, then suppose \(\vec{x} \) is a
	right \hyperref[def:eigenvector]{eigenvector}, we have
	\[
		A \vec{x} = \lambda \vec{x}\Rightarrow
		(A \vec{x})^{\top} = \lambda \vec{x}^{\top}\Rightarrow
		\vec{x}^{\top} A^{\top} = \lambda\vec{x}^{\top}\Rightarrow
		\vec{x}^{\top} A = \lambda\vec{x}^{\top},
	\]
	where we see that \(\vec{x}\) is just left \hyperref[def:eigenvector]{eigenvector}, hence the left and right \hyperref[def:eigenvector]{eigenvectors} are the same.
\end{proof}