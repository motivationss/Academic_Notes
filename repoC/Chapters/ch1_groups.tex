

\part{Foundations}
\chapter{Groups and Homomorphism}\label{ch:group}
\section{Basics}
\begin{definition}[Group]\label{def:group}
	A pair (\(G\), \(\odot	\)) consisting of a nonempty set \(G\) and an operation \(\odot\)
	is called a \textbf{group} if the following holds:
	\begin{itemize}
		\item \(G\) is closed under the operation \(\odot\)
		\item \(\odot\) is associative
		\item \(\odot\) has an identity element \(e\)
		\item Each \(g \in G\) has an \textbf{inverse} \(h \in G\) such that \(g \odot h = h \odot g = e\)
	\end{itemize}
	
\end{definition}

\begin{definition}[Abelian group]\label{def:abelian-group}
	A group \(G, \odot\) is called \textbf{commutative} or \textbf{Abelian}
	if \(\odot\) is a commutative operation on \(G\).
\end{definition}

\begin{remark} 
	Let \(G = \left(G, \odot\right) \)
	\begin{enumerate}[label=(\alph*)]
		\item the identity element \(e\) is unique
		\item Each \(g \in G\) has a unique inverse which we denote by \(g^b\). In particular \(e^b = e\).
		\item For each \(g \in G\), we have \(\left(g^b\right)^b = g\). 
		\item For arbitrary group elements \(g\) and \(h\), \(\left(g \odot h\right)^b = h^b \odot g^b\)
	\end{enumerate}
\end{remark}

\begin{eg}
	\begin{enumerate}[label=(\alph*)]
		\item Let \(G \coloneqq \{e\}\) be a one element set. Then \( \{G, \odot\} \) is an Abelian group, the 
		\textbf{trivial group}, with the (only possible) operation \(e \odot e = e\). 
		\item Let \(X\) be a nonempty set, and \(S_X\) be the set of all bijections from \(X\) to itself. 
		Then \(S_X \coloneqq \left(S_X, \circ \right)\) is a group with identity element \(id_X\) when \(\circ\)
		denotes the composition of functions. Further, the inverse function \(f^{-1}\) is the inverse of \(f \in S_X\)
		in the group. When \(X\) is finite, the element of \(S_X\) are called permutations and \(S_X\) is called 
		the \textbf{permutation group} of \(X\).
		\item Let \(X\) be a nonempty set and \(G, \odot \) a group. With the induced operation \(\odot\), 
		\( \left(G^X, \odot\right)\) is a group. The inverse of \(f \in G^X\) is the function
		\[
			f^b \colon X 	\rightarrow G, \;\; x \mapsto \left(f\left(x\right)\right)^b
		\]
		\item Let \(G_1,\ldots,G_m\) be groups. Then \(G_1 \times \cdots \times G_m\) with the operation defined
		analogously to (d) is a group called the \textbf{direct product} of \(G_1,\ldots,G_m\). 
	\end{enumerate}
\end{eg}

\section{Subgroup}

\begin{definition}[Subgroup]\label{def:subgroup}
	Let \(G = \left(G, \odot\right)\) be a group and \(H\) a nonempty subset of \(G\), if
	\begin{itemize}
		\item \(H \odot H \subseteq H\)
		\item \(h^b \in H\) for all \(h \in H\)
	\end{itemize}
	then \(H \coloneqq \left(H, \odot\right)\) is itself a group and is called a \textbf{subgroup} of \(G\).
\end{definition}

\begin{remark}
	Here we use the same symbol \(\odot\) for the restriction of the operation to \(H\). Since
	\(H\) is nonempty, there is some \(h \in H\) and so, from the two axioms above,
	\(e = h^b \odot h\) is also in \(H\).
\end{remark}

\begin{eg} Let \(G = \left(G, \odot\right)\) be a group.
	\begin{enumerate}[label=(\alph*)]
		\item The trivial subgroup \(\{e\}\) and \(G\) itself are subgroups of \(G\), the smallest and
		largest subgroups with respect to inclusion
		\item If \(H_\alpha\), \(\alpha \in A\) are subgroups of \(G\), then \(\bigcap_\alpha H_\alpha\)
		is also a subgroup of \(G\). 
	\end{enumerate}
\end{eg}

\section{Cosets}
\begin{definition}[Coset]\label{def:coset}
	Let \(N\) be a subgroup of \(G\) and \(g \in G\). Then \(g \odot N\) is the \textbf{left coset}
	and \(N \odot g\) is the \textbf{right coset} of \(g \in G\) with respect to \(N\).
\end{definition}

\begin{remark}
	The definition of coset is related to the particular element.
\end{remark}

\begin{note}
	If we define 

	\begin{equation}
		\label{eqn: coset_equivalence}
		g \sim  h  \Leftrightarrow  g \in h \odot N
	\end{equation}
    Then \(\sim\) is an equivalence on \(G\).
	\begin{proof}
		\(\sim\) is reflexive because \(e \in N\)

		Let \(g \in h \odot N\) and \(h \in k \odot N\), then 
		\[
			g \in \left(k \odot N \right) \odot N = k \odot \left(N \odot N\right) = k \odot N
		\]

		Let \(g \in h \odot N\), then there is some \(n \in N\) with \(g = h \odot n\). Then it 
		follows that \(h = g \odot n^b \in N\).
	\end{proof}

	Here \ref{eqn: coset_equivalence} defines an equivalence relation on \(G\). For the equivalence
	classes \(\left [ \cdot  \right ]\) with respect to \(\sim\), we have
	\begin{equation}
		\left[g\right] = g \odot N, \:\: g \in G.
	\end{equation}
	For this reason, we denote \(G / \sim \) by \(G/N\), and call \(G/N\) the \textbf{set of left cosets}
	of \(G\) \textbf{modulo} \(N\).
	Particularly, we have subgroups \(N\) such that
	\begin{equation}
		\label{eqn: normal_subgroup}
		g \odot N = N \odot g, \:\:\: g \in G.
	\end{equation}
	Such a subgroup \ref{eqn: normal_subgroup} is called a \textbf{normal subgroup} of \(G\). We call
	\(g \odot N\) the \textbf{coset of} \(g\) \textbf{modulo} \(N\) since each left coset is a right coset
	and vice versa. 
	We have a well-defined operation on \(G/N\) where \(N\) is the normal subgroup of \(G\), induced from \(\odot\), such that
	\begin{equation}
		(G/N) \times (G/N) \rightarrow G/N, \:\:\: (g\odot N,h\odot N) \mapsto (g\odot h) \odot N
	\end{equation}
\end{note}

\begin{proposition}
	Let \(G\) be a group and \(N\) a normal subgroup of \(G\). Then \(G/N\) with the induced operation
	is a group, the \textbf{quotient group of} \(G\) \textbf{modulo} N.
\end{proposition}

\begin{proof}
	It is easy to check that the operation is associative. Since \(\left(e \odot N\right)
	\odot \left(g\odot N\right) = \left(e \odot g\right) \odot N = g \odot N \), the identity element
	of \(G/N\) is \(N = e \odot N\). Also 
	\[
		\left(g^b \odot N\right) \odot \left(g \odot N\right) = \left(g^b \odot g\right) \odot N = N	
	\]
\end{proof}

\begin{remark}
	\begin{enumerate}[label=(\alph*)]
		\item In notion of \ref{eqn: coset_equivalence}, \([e] = N\) is the identity element of
		\(G/N\) and \([g]^b = [g^b]\) is the inverse of \([g] \in G/N\). We also have
		\([g] \odot h = [g\odot h], \:\: g,h\in G\).
		\item Any subgroup \(N\) of an Abelian group \(G\) is normal and so \(G/N\) is a group. 
		Meanwhile, \(G/N\) is Abelian.
	\end{enumerate}
\end{remark}

\section{Homomorphisms}

\begin{definition}[Homomorphism]\label{def: homomorphism}
	Let \(G = \left(G, \odot\right)\) and \(G{}' = \left(G{}', \circledast\right)\) be groups\dots
	A function \(\varphi \colon G \rightarrow G{}'\) is called a (\textbf{group}) \textbf{homomorphism}
	if 
	\[
		\varphi ( g \odot h ) = \varphi(g) \circledast \varphi(h), \:\:\:\:\: g,h\in G
	\]
\end{definition}

\begin{definition}[Endomorphism]\label{def: endomorphism}
	A homomorphism from \(G\) to itself
\end{definition}

\begin{remark}
	\begin{enumerate}[label=(\alph*)]
		\item Let \(e\) and \(e{}'\) be the identity elements of \(G\) and \(G{}'\) respectively,
		and let \(\varphi \colon G \rightarrow G{}'\) be a homomorphism. Then
		\[
			\varphi(e) = e{}' \:\:\:\:\: \text{and} \:\:\: (\varphi(g))^b = \varphi(g^b), \:\:\: g \in G
		\]
		\begin{proof}
			\(e{}' \circledast \varphi\left(e\right) = \varphi\left(e\right) = \varphi\left(e \odot e\right)
			= \varphi\left(e\right) \circledast \varphi\left(e\right)\) \\
			\(e{}' = \varphi\left(e\right) = \varphi\left(g^b \odot g\right) = \varphi\left(g^b\right)
			\circledast \varphi\left(g\right)\)
		\end{proof}

		\item Let \(\varphi \colon G \rightarrow G{}'\) be a homomorphism. The \textbf{kernel} of 
		\(\varphi\), ker(\(\varphi\)), defined by 
		\[
			\text{ker}(\varphi) \coloneqq \varphi^{-1}(e{}') = \{g \in G \: ; \: \varphi(g)=e{}'\}	
		\]
		is a normal subgroup of \(G\). 
		\begin{proof}
			First, try to prove ker(\(\varphi\)) is a subgroup of \(G\). For all \(g, h \in G\),
			\begin{itemize}
				\item \(\varphi\left(g \odot h\right) = \varphi\left(g\right) \circledast
				\varphi \left(h\right) = e{}' \circledast e{}' = e{}' \)
				\item \(\varphi\left(g^b\right) = \left(\varphi\left(g\right)\right)^b 
				= \left(e{}'\right)^b = e{}'\)
			\end{itemize}
			Second, try to prove it is a normal subgroup. Let \(h \in g \odot \text{ker}\left(\varphi\right)\). 
			Then we there is some \(n \in G\) such that \(\varphi\left(n\right) = e{}'\) and 
			\(h = g \odot n\). For \(m \coloneqq g \odot n \odot g^b\), we have
			\[
				\varphi(m) = \varphi(g) \circledast	\varphi(n) \circledast \varphi(g^b)
				= \varphi(g) \circledast \varphi(g^b) = e{}' 
			\]
			and hence \(m \in \text{ker}\left(\varphi\right)\). Since \(m \odot g = g \odot m = h\),
			this implies that \(h \in \text{ker}\left(\varphi\right) \odot g\). So 
			\(\text{ker}\left(\varphi\right) \odot g \subseteq g \odot \text{ker}\left(\varphi\right)\).
			Similarly one can show \(g \odot \text{ker}\left(\varphi\right) \subset
			\text{ker}\left(\varphi\right) \odot g \).
		\end{proof}

		\item Let \(\varphi \colon G \rightarrow G{}'\) be a homomorphism and \(N \coloneqq\text{ker}\left(\varphi\right) \). Then
		\[
			g \odot N = \varphi^{-1}(\varphi(g)), \:\:\:\:\:       g \in G, 	
		\]
		and so
		\[
			g \sim h \Leftrightarrow \varphi(g) = \varphi(h), \:\:\:\:\: g,h \in G,	
		\]
		where \(\sim\) denotes the equivalence relation \ref{eqn: coset_equivalence}.

		\item A homomorphism is injective if and only if its kernel is trivial, that is,
		ker\(\left(\varphi\right) = \{e\}\)
		\item The image im(\(\varphi\)) of a homomorphism \(\varphi \colon G \rightarrow G{}'\) is a subgroup of \(G{}'\).
	\end{enumerate}
\end{remark}

\begin{eg}  
	\begin{enumerate}[label=(\alph*)]
		\item The constant function \(G \rightarrow G{}'\), \(g \mapsto e{}'\) is a homorphism,
		the \textbf{trivial} homomorphism.
		\item The identity function \(\text{id}_G \colon G \rightarrow G\) is an endomorphism.
		\item Compositions of homomorphisms (endomorphisms) are homomorphisms (endomorphisms).
		\item If \(\varphi \colon G \rightarrow G{}'\) is a bijective homomorphism, then so is
		\(\varphi^{-1} \colon G \rightarrow G{}'\)
	\end{enumerate}
\end{eg}

\begin{definition}[Isomorphism]\label{def: isomorphism}
	A homomorphism \(\varphi \colon G \rightarrow G{}'\) is called a (\textbf{group}) \textbf{isomorphism}
	from \(G\) to \(G{}'\) if \(\varphi\) is bijective. \\
	In this circumstance, we say that the groups \(G\) and \(G{}'\) are \textbf{isomorphic} and write
	\(G \cong G{}'\). 
\end{definition} 

\begin{definition}[Automorphism]\label{def: automorphism}
	An isomorphism from \(G\) to itself. 
\end{definition}



