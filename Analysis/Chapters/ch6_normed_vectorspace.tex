\chapter{Normed Vector Space}

\section{Norms}

\begin{definition}[Norm]\label{def: norm}
    Let \(E\) be a vector space over \(\mathbb{K}\). A function \(|| \cdot || \colon E \to
    \mathbb{R}^+\) is called a \textbf{norm} if the following hold:
    \begin{itemize}
        \item \( || x || = 0 \Leftrightarrow x = 0\). 
        \item \( || \lambda x|| = |\lambda| || x || \), \(x \in E\), \(\lambda \in \mathbb{K}\)
        (positive homogeneity)
        \item \( || x + y || \leq || x || + || y ||\), \(x, y \in E\) (triangle inequality).
        A pair (\(E, ||\cdot||\)) consisting of a vector space \(E\) and a norm
        \(||\cdot||\) is called a \textbf{normed vector space}. If the norm is clear from context,
        we write \(E\) instead of (\(E, || \cdot||\)).  
    \end{itemize}
\end{definition}

\begin{remark}
    Let \(E \coloneqq \left(E, ||\cdot||\right)\) be a normed vector space. 
    \begin{enumerate}[label=(\alph*)]
        \item The function 
        \[
            d \colon E \times E \to \mathbb{R}^+ , \:\:\:\:\: (x, y) \mapsto ||x-y||  
        \]
        is a metric on \(E\), the \textbf{metric induced from the norm}. Hence any 
        normed vector space is also a metric space. 
        \item The \textbf{reversed triangle inequality} holds for the norm:
        \[
            || x - y || \geq \left |  || x || - || y ||  \right |, \:\:\:\: x,y \in E  
        \]

        \item All statements from previous chapter also hold in normed vector space. 
    \end{enumerate}
\end{remark}

\subsection*{Balls}

For \(a \in E\) and \(r > 0\), we define the \textbf{open} and \textbf{closed balls} with
center at \(a\) and radius \(r\) by 
\[
   \mathbb{B}_E (a, r) \coloneqq \mathbb{B} (a, r) \coloneqq {x \in E; || x - a || < r}  
\]
and 
\[
  \bar{\mathbb{B}}_E (a, r) \coloneqq \bar{\mathbb{B}}_E \coloneqq {x\in E ; || x - a|| \leq r}.  
\]

These definitions agree with those for the metric space (\(E, d\)) when \(d\) is induced
from norm. We also write
\[
  \mathbb{B} \coloneqq \mathbb{B}(0, 1) = {x\in E; ||x|| < 1} \:\:\:\:\: \text{and}
  \:\:\:\:\: \bar{\mathbb{B}} \coloneqq \bar{\mathbb{B}}(0, 1) = {x \in E; ||x|| \leq 1}  
\]
for the \textbf{open} and \textbf{closed unit balls} in \(E\). We have
\[
    r\mathbb{B} = \mathbb{B}(0, r), \:\:\: r\bar{\mathbb{B}} = \bar{\mathbb{B}}(0, r)  
\]

\subsection*{Bounded Sets}

A subset \(X\) of \(E\) is called \textbf{bounded in} \(E\) (or \textbf{norm bounded}) if 
it is bounded in the induced metric space.

\begin{remark}
    Let \(E \coloneqq \left(E, ||\cdot||\right)\) be a normed vector space 
    \begin{enumerate}[label=(\alph*)]
        \item \(X \subseteq E\) is bounded if and only if there is some \(r > 0\) such
        that \(X \subseteq r\mathbb{B}\), that is, \(||x|| < r\) for all \(x\in X\). 
        \item If \(X\) and \(Y\) are nonempty bounded subsets of \(E\), then so are
        \(X \cup Y \), \(X + Y\) and \(\lambda X\) with \(\lambda \in \mathbb{K}\). 
    \end{enumerate}
\end{remark}

\begin{eg}
    \begin{enumerate}[label=(\alph*)]
        \item The absolute value \(|\cdot|\) is a norm on the vector space \(\mathbb{K}\). 
        \item Let \(F\) be a subspace of a normed vector space 
        \(E \coloneqq \left(E, \lVert \cdot \rVert \right)\). Then the restriction 
        \(\lVert \cdot \rVert_F \coloneqq  \lVert \cdot \rVert | F\) of 
        \(\lVert \cdot \rVert\) to \(F\) is a norm on \(F\). Thus 
        \(F \coloneqq \left(F, \lVert \cdot \rVert_F\right)\) is a normed vector space 
        with this \textbf{induced norm}. 
        \item Let \(E_j, \lVert \cdot \rVert_j\), \(1 \leq j \leq m\), be normed vector 
        space over \(\mathbb{K}\). Then

        \[
            \lVert x \rVert_\infty \coloneqq \displaystyle{\text{max}_{1\leq j\leq m}}
            \lVert x_j \rVert_j, \:\:\:\:\:\:\:\: x = (x_1,\ldots,x_m) \in E \coloneqq
            E_1 \times \cdots \times E_m   
        \]
        defines a norm, called the \textbf{product norm}, on the product vector space 
        \(E\). The metric on \(E\) induced from this norm coincides with the product metric
        from 5.2 example (d), where \(d_j\) is the metric induced on \(E_j\) from 
        \(\lVert \cdot \rVert_j\). 
        \item For \(m \in \mathbb{N}^\times, \mathbb{K}^m \) is a normed vector space
        with the \textbf{maximum norm}
        \[
            |x|_\infty \coloneqq  \text{max}_{1 \leq j \leq m} |x_j|, \:\:\:\:\:\:
            x = (x_1,\ldots, x_m) \in \mathbb{K}^m.   
        \] 
        In this case m = 1. 
    \end{enumerate}
\end{eg}

\section{The Space of Bounded Functions}

Let \(X\) be a nonempty set and (\(E, \lVert \cdot \rVert\)) a normed vector space. A function
\(u \in E^X\) is called \textbf{bounded} if the image of \(u\) in \(E\) is bounded. For 
\(u \in E^X\), define 
\[
    \lVert u \rVert_\infty \coloneqq \lVert u \rVert_{\infty,X} \coloneqq 
    \displaystyle{\text{sup}_{x\in X}} \lVert u(x) \rVert \in \mathbb{R}^+ \cup {\infty}
\]

\begin{remark}
    \begin{enumerate}[label=(\alph*)]
        \item For \(u \in E^X\), the followings are equivalent:
        \begin{enumerate}[label=(\roman*)]
            \item \(u\) is bounded. 
            \item \(u\left(X\right)\) is bounded in \(E\). 
            \item There is some \(r > 0\) such that \(\lVert u\left(x\right) \rVert \leq r\)
            for all \(x \in X\). 
            \item \(\lVert u \rVert_\infty < \infty\). 
        \end{enumerate}
        \item Clearly id \(\in \mathbb{K}^\mathbb{K}\) is not bounded, that is, 
        \(\lVert \text{id} \rVert_\infty = \infty\). 
    \end{enumerate}
    (b) shows that \(\lVert \cdot \rVert_\infty \) may not be a norm on the vector space
    \(E^X\) when \(E\) is not trivial. We therefore set
    \[
        B(X,E) \coloneqq {u \in E^X \: ; \: u \: \text{ is bounded}}  
    \]
    and call \(B\left(X, E\right)\) the \textbf{space of bounded functions} from \(X\) to \(E\). 
\end{remark}

\begin{proposition}
    \(B\left(X, E\right)\) is a subspace of \(E^X\) and \(\lVert \cdot \rVert_\infty \) is a norm,
    called the \textbf{supremum norm}, on \(B\left(X, E\right)\). 
\end{proposition}

\begin{remark}
    \begin{enumerate}[label=(\alph*)]
        \item If \(X\coloneqq \mathbb{N}\), then \(B\left(X, E\right)\) is the normed vector
        space of bounded sequences in \(E\). In the special case \(E\coloneqq \mathbb{K}\),
        \(B\left(\mathbb{N}, \mathbb{K}\right)\) is denoted by \(\ell_\infty\), that is, 
        \[
            \ell_\infty \coloneqq \ell\infty(\mathbb{K}) \coloneqq B(\mathbb{N}, \mathbb{K}) 
        \]
        is the normed \textbf{vector space of bounded sequences} with the supremum norm
        \[
            \lVert (x_n) \rVert_\infty = \displaystyle\text{sup}_{n\in \mathbb{N}} \lvert x_n \rvert
            , \:\:\:\:\: (x_n) \in \ell_\infty  
        \]
        \item If \(X = \{1,\ldots,m\}\) for some \(m \in \mathbb{N}^\times\), then
        \[
            B(X,E) = (E^m, \lVert \cdot \rVert_\infty)  
        \]
    \end{enumerate}
\end{remark}

\section{Inner Product Spaces}

\begin{definition}[Inner Product Space]\label{def: inner_product}
    Let \(E\) be a vector space over the filed \(\mathbb{K}\). A function
    \[
        (\cdot | \cdot) \colon E \times E \to \mathbb{K}, \:\:\:\: (x, y) \mapsto (x|y)  
    \]
    is called a \textbf{scalar product} or \textbf{inner product} on \(E\) if the following
    hold:
    \begin{itemize} 
        \item (\(x|y\)) = \(\overline{\left(y|x\right)}\), \(x, y\in E\)
        \item (\(\lambda x + \mu y | z\)) = \(\lambda\left(x|z\right) + \mu\left(y|z\right)\), 
        \(x, y, z \in E\), \(\lambda, \mu \in \mathbb{K}\). 
        \item (\(x|x\)) \(\geq\) 0, \(x \in E\), and \(\left(x|x\right) = 0 \Leftrightarrow 0\). 
    \end{itemize}
    A vector space \(E\) with a scalar product \(\left(\cdot|\cdot\right)\) is called 
    an \textbf{inner product space} and is written in \(\left(E, \left(\cdot|\cdot\right)\right)\). 
\end{definition}

\begin{remark}
    \begin{enumerate}[label=(\alph*)]
        \item In the real case \(\mathbb{K} = \mathbb{R}\), the first point can be written as
        \[
            (x|y) = (y|x), \:\:\:\: x, y \in E  
        \]
        In other words, the function is \textbf{symmetric} when \(E\) is a real vector space.
        In the case \(\mathbb{K} = \mathbb{C}\), the function is said to be \textbf{Hermitian}
        when the first point holds. 
        \item From the first two points it follows that
        \[
            (x|\lambda y + \mu z) = \bar{\lambda}(x|y) + \bar{\mu}(x|z), \:\:\:\:
            x,y,z \in E, \:\:\:\: \lambda,\mu \in \mathbb{K},     
        \]
        that is, for each fixed \(x \in E\), the function \(\left(x|\cdot\right)\colon E \to \mathbb{K}\)
        is \textbf{conjugate linear}. If \(\mathbb{K} = \mathbb{R}\), it is \textbf{bilinear}. 

        \item (\(x|0\)) = 0 for all \(x \in E\). 
    \end{enumerate}

\end{remark}

Let \(m \in \mathbb{N}^\times\). For \(x = \left(x_1,\ldots,x_m\right)\) and 
\(y = \left(y_1,\ldots,y_m\right)\) in \(\mathbb{K}^m\), define
\[
    (x|y) \coloneqq \sum^m_{j=1} x_j \bar{y}_j   
\]
to be the \textbf{Euclidean Inner product} on \(\mathbb{K}^m\). 

\subsection*{The Cauchy-Schwarz Inequality}

\begin{theorem}[Cauchy-Schwarz Inequality]\label{theorem: Cauchy-Schwarz}
    Let (\(E, \left(\cdot|\cdot\right)\)) be an inner product space. Then
    \[
        \lvert (x|y) \rvert^2  \leq (x|x)(y|y) \:\:\:\:\:\: x,y \in E   
    \]
    and the equality occurs if and only if \(x\) and \(y\) are linearly dependent.
\end{theorem}

\begin{theorem}
    Let (\(E, \left(\cdot|\cdot\right)\)) be an inner product space and 
    \[
        \lVert x \rVert \coloneqq \sqrt{(x|x)}, \:\:\:\:\:\: x\in E  
    \]
    Then \(\lVert \cdot \rVert\) is a norm on \(E\), the \textbf{norm induced from
    the scalar product} (\(\cdot | \cdot\)). 
    A norm which is induced from a scalar product is also called a \textbf{Hilbert norm}. 
\end{theorem}

\begin{corollary}
    Let (\(E, \left(\cdot|\cdot\right)\)) be an inner product space. Then
    \[
        \lvert (x|y) \rvert \leq \lVert x \rVert \lVert y\rVert \:\:\:\:\:\: x,y \in E
    \]
\end{corollary}

\section{Euclidean Spaces}

Convention: Unless otherwise stated, we consider \(\mathbb{K}^m\) to be endowed with 
the Euclidean inner product (\(\cdot|\cdot\)) and the induced norm
\[
    \lvert x \rvert \coloneqq \sqrt{(x|x)} = \sqrt{\sum^m_{j=1}\lvert x_j\rvert^2}
    \:\:\:\:\:\: x = (x_1,\ldots,x_m) \in \mathbb{K}^m 
\]
the \textbf{Euclidean norm}. In the real case, we write also \(x \cdot y\) for (\(x|y\)). 

We further define the norm 
\[
    \lvert x \rvert_1 \coloneqq \sum^m_{j=1} \lvert x_j \rvert , \:\:\:\:\: x = (x_1,\ldots,x_m)\in \mathbb{K}^m  
\]
\begin{proposition}
    Let \(m \in \mathbb{N}^\times\). Then 
    \[
        \lvert x \rvert_\infty \leq \lvert x \rvert \leq \sqrt{m} \lvert x \rvert_\infty
        , \:\:\:\:\:\:\:\:\:\: \frac{1}{\sqrt{m}}\lvert x \rvert_1 \leq \lvert x \rvert
        \leq \lvert x \rvert_1, \:\:\:\:\:\: x \in \mathbb{K}^m   
    \]
\end{proposition}

\subsection*{Equivalent Norm}

Let \(E\) be a vector space. Two norms \(\lVert \cdot \rVert_1\) and \(\lVert \cdot \rVert_2\)
on \(E\) are \textbf{equivalent} if there is some \(K \geq 1\) such that 
\[
    \frac{1}{K} \lVert x \rVert_1 \leq \lVert x \rVert_2 \leq K \lVert x \rVert_1 , \:\:\:\: x\in E
\]

In this case we write \(\lVert \cdot \rVert_1 ~ \lVert \cdot \rVert_2\). 

\begin{remark}
    \begin{enumerate}[label=(\alph*)]
        \item \(~\) is an equivalence relation on the set of all norms of a fixed vector space. 
        \item \(\lVert \cdot \rVert_1 ~ \lVert \cdot \rVert ~ \lVert \cdot \rVert_\infty\) 
        on \(\mathbb{K}^m\). 
        \item We write \(\mathbb{B}^m\) for the \textbf{real open Euclidean unit ball},
        that is, \(\mathbb{B}^m \coloneqq \mathbb{B}_{\mathbb{R}^m}\) and 
        \(\mathbb{B}^m_1\) and \(\mathbb{B}^m_\infty\) for the unit balls in 
        (\(\mathbb{R}^m, \lvert \cdot \rvert_1\)) and in (\(\mathbb{R}^m, \lvert \cdot \rvert_\infty\))
        respectively. We have
        \[
          \mathbb{B}^m \subseteq \mathbb{B}^m_\infty \subseteq \sqrt{m}\mathbb{B}^m
          \:\:\:\:\:\:\:
          \mathbb{B}^m_1 \subseteq \mathbb{B}^m \subseteq \sqrt{m}\mathbb{B}^m_1
        \]

        \item Let \(E = \left(E, \lVert \cdot \rVert \right)\) be a normed vector space
        and \(\lVert \cdot \rVert_1\) a norm on \(E\) which is equivalent to
        \(\lVert \cdot \rVert\). Set \(E_1 \coloneqq \left(E, \lVert \cdot \rVert_1\right)\). 
        Then 
        \[
            \mathcal{U}_E(a) = \mathcal{U}_{E_1}(a), \:\:\:\:\:\:\: a \in E  
        \]
        that is, the set of neighborhoods of \(a\) depends only on the equivalence class
        of the norm. Equivalent norms produce the same set of neighborhoods. 
    \end{enumerate}
\end{remark}

\subsection*{Convergence in Product Spaces}

\begin{proposition}
    Let \(m \in \mathbb{N}^\times\) and \(x_n = \left(x^1_n, \ldots, x^m_n\right) \in \mathbb{K}^m\)
    for \(n \in \mathbb{N}\). Then the followings are equivalent:
    \begin{enumerate}
        \item The sequence \(\left(x_n\right)_{n\in \mathbb{N}}\) converges to 
        \(x = \left(x^1, \ldots, x^m\right)\) in \(\mathbb{K}^m\). 
        \item For each \(k \in \{1, \ldots, m\}\), the sequence \(\left(x^k_n\right)_{n\in\mathbb{N}}\)
        converges to \(x^k\) in \(\mathbb{K}\). 
    \end{enumerate}
\end{proposition}

