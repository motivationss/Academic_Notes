\documentclass[a4paper]{report}
% basics
\usepackage[utf8]{inputenc}
\usepackage[T1]{fontenc}
\usepackage{textcomp}
\usepackage[hyphens]{url}
\usepackage[style=alphabetic,maxcitenames=1]{biblatex}
\usepackage[colorlinks=true,linkcolor=cyan,urlcolor=magenta,citecolor=violet]{hyperref}
\usepackage{graphicx}
\usepackage{float}
\usepackage{booktabs}
\usepackage[inline, shortlabels]{enumitem}
\usepackage{emptypage}
\usepackage{subcaption}
\usepackage{multicol}
\usepackage[usenames,dvipsnames]{xcolor}
% quiver style
\usepackage{tikz-cd}
% `calc` is necessary to draw curved arrows.
\usetikzlibrary{calc}
% `pathmorphing` is necessary to draw squiggly arrows.
\usetikzlibrary{decorations.pathmorphing}

% A TikZ style for curved arrows of a fixed height, due to AndréC.
\tikzset{curve/.style={settings={#1},to path={(\tikztostart)
					.. controls ($(\tikztostart)!\pv{pos}!(\tikztotarget)!\pv{height}!270:(\tikztotarget)$)
					and ($(\tikztostart)!1-\pv{pos}!(\tikztotarget)!\pv{height}!270:(\tikztotarget)$)
					.. (\tikztotarget)\tikztonodes}},
	settings/.code={\tikzset{quiver/.cd,#1}
			\def\pv##1{\pgfkeysvalueof{/tikz/quiver/##1}}},
	quiver/.cd,pos/.initial=0.35,height/.initial=0}

% TikZ arrowhead/tail styles.
\tikzset{tail reversed/.code={\pgfsetarrowsstart{tikzcd to}}}
\tikzset{2tail/.code={\pgfsetarrowsstart{Implies[reversed]}}}
\tikzset{2tail reversed/.code={\pgfsetarrowsstart{Implies}}}
% TikZ arrow styles.
\tikzset{no body/.style={/tikz/dash pattern=on 0 off 1mm}}

% useful macro for class
\newcommand{\probability}[2]{\mathbb{\MakeUppercase{P}}_{#1} \left(#2\right)}
\newcommand{\variance}[2]{\mathrm{Var}_{#1} \left[ #2 \right]}
\newcommand{\expectation}[2]{\mathbb{\MakeUppercase{E}}_{#1} \left[#2\right]}
\newcommand{\at}[3]{\left.#1\right\vert_{#2}^{#3}}
\newcommand\quotient[2]{
	\mathchoice
	{% \displaystyle
		\text{\raise1ex\hbox{$#1$}\Big/\lower1ex\hbox{$#2$}}%
	}
	{% \textstyle
		#1\,/\,#2
	}
	{% \scriptstyle
		#1\,/\,#2
	}
	{% \scriptscriptstyle  
		#1\,/\,#2
	}
}
\newcommand{\identity}{\mathrm{id}}
\newcommand{\sinc}{\mathop{\mathrm{sinc}}}
\newcommand{\rect}{\mathop{\mathrm{rect}}}
\newcommand{\tri}{\mathop{\mathrm{tri}}}
\newcommand{\Real}{\mathop{\mathrm{Re}}}

\newcommand{\Homomorphism}{\mathrm{Hom}}
\newcommand{\Morphism}{\mathrm{Mor}}
\newcommand{\Object}{\mathrm{Ob}}

\usepackage{physics}

\DeclareMathOperator{\im}{Im}
\DeclareMathOperator{\sgn}{sgn}
\DeclareMathOperator{\Int}{Int}
\DeclareMathOperator{\diag}{diag}

\let\implies\Rightarrow
\let\impliedby\Leftarrow
\let\iff\Leftrightarrow

\usepackage{stmaryrd} % for \lightning
\newcommand\conta{\scalebox{1.1}{\(\lightning\)}}

\usepackage{bm}
\usepackage{bbm}


% figure support
\usepackage{import}
\usepackage{xifthen}
\pdfminorversion=7
\usepackage{pdfpages}
\usepackage{transparent}
\newcommand{\incfig}[1]{%
	\def\svgwidth{\columnwidth}
	\import{./Figures/}{#1.pdf_tex}
}


\usepackage{amsmath, amsfonts, mathtools, amsthm, amssymb}
\usepackage{geometry}
\usepackage{mathrsfs}
\usepackage{cancel}
\usepackage{systeme}
\usepackage{caption}
\captionsetup{belowskip=0pt}

\geometry{a4paper,left=2.54cm,right=2.54cm,top=2.54cm,bottom=2.54cm}


% for the big braces
\usepackage{bigdelim}


% correct
\definecolor{correct}{HTML}{009900}
\newcommand\correct[2]{\ensuremath{\:}{\color{red}{#1}}\ensuremath{\to }{\color{correct}{#2}}\ensuremath{\:}}
\newcommand\green[1]{{\color{correct}{#1}}}



% horizontal rule
\newcommand\hr{
	\noindent\rule[0.5ex]{\linewidth}{0.5pt}
}


% hide parts
\newcommand\hide[1]{}



% si unitx
\usepackage{siunitx}
\sisetup{locale = FR}
% \renewcommand\vec[1]{\mathbf{#1}}
\newcommand\mat[1]{\mathbf{#1}}


% tikz
\usepackage{tikz}
\usetikzlibrary{intersections, angles, quotes, positioning}
\usetikzlibrary{arrows.meta}
\usepackage{pgfplots}
\pgfplotsset{compat=1.13}


\tikzset{
	force/.style={thick, {Circle[length=2pt]}-stealth, shorten <=-1pt}
}

% Algorithm Env
\usepackage[linesnumbered,lined,vlined,ruled,commentsnumbered]{algorithm2e}
\SetKwComment{Comment}{// }{}
\SetArgSty{textsl}

% theorems
\makeatother
\usepackage{thmtools}
\usepackage[framemethod=TikZ]{mdframed}

\mdfsetup{skipabove=1em,skipbelow=0em}

\theoremstyle{definition}

\declaretheoremstyle[
	headfont=\bfseries\sffamily\color{ForestGreen!70!black}, bodyfont=\normalfont,
	mdframed={
			linewidth=2pt,
			rightline=false, topline=false, bottomline=false,
			linecolor=ForestGreen, backgroundcolor=ForestGreen!5,
			nobreak=false
		},
]{thmgreenbox}

\declaretheoremstyle[
	headfont=\bfseries\sffamily\color{ForestGreen!70!black}, bodyfont=\normalfont,
	mdframed={
			linewidth=2pt,
			rightline=false, topline=false, bottomline=false,
			linecolor=ForestGreen, backgroundcolor=ForestGreen!8,
			nobreak=false
		},
]{thmgreen2box}

\declaretheoremstyle[
	headfont=\bfseries\sffamily\color{NavyBlue!70!black}, bodyfont=\normalfont,
	mdframed={
			linewidth=2pt,
			rightline=false, topline=false, bottomline=false,
			linecolor=NavyBlue, backgroundcolor=NavyBlue!5,
			nobreak=false
		}
]{thmbluebox}

\declaretheoremstyle[
	headfont=\bfseries\sffamily\color{TealBlue!70!black}, bodyfont=\normalfont,
	mdframed={
			linewidth=2pt,
			rightline=false, topline=false, bottomline=false,
			linecolor=TealBlue,
			nobreak=false
		}
]{thmblueline}

\declaretheoremstyle[
	headfont=\bfseries\sffamily\color{RawSienna!70!black}, bodyfont=\normalfont,
	mdframed={
			linewidth=2pt,
			rightline=false, topline=false, bottomline=false,
			linecolor=RawSienna, backgroundcolor=RawSienna!5,
			nobreak=false
		}
]{thmredbox}

\declaretheoremstyle[
	headfont=\bfseries\sffamily\color{RawSienna!70!black}, bodyfont=\normalfont,
	mdframed={
			linewidth=2pt,
			rightline=false, topline=false, bottomline=false,
			linecolor=RawSienna, backgroundcolor=RawSienna!8,
			nobreak=false
		}
]{thmred2box}

\declaretheoremstyle[
	headfont=\bfseries\sffamily\color{SeaGreen!70!black}, bodyfont=\normalfont,
	mdframed={
			linewidth=2pt,
			rightline=false, topline=false, bottomline=false,
			linecolor=SeaGreen, backgroundcolor=SeaGreen!2,
			nobreak=false
		}
]{thmgreen3box}

\declaretheoremstyle[
	headfont=\bfseries\sffamily\color{WildStrawberry!70!black}, bodyfont=\normalfont,
	numbered=no,
	mdframed={
			linewidth=2pt,
			rightline=false, topline=false, bottomline=false,
			linecolor=WildStrawberry, backgroundcolor=WildStrawberry!5,
			nobreak=false
		},
]{thmpinkbox}

\declaretheoremstyle[
	headfont=\bfseries\sffamily\color{MidnightBlue!70!black}, bodyfont=\normalfont,
	mdframed={
			linewidth=2pt,
			rightline=false, topline=false, bottomline=false,
			linecolor=MidnightBlue, backgroundcolor=MidnightBlue!5,
			nobreak=false
		}
]{thmblue2box}

\declaretheoremstyle[
	headfont=\bfseries\sffamily\color{Gray!70!black}, bodyfont=\normalfont,
	mdframed={
			linewidth=2pt,
			rightline=false, topline=false, bottomline=false,
			linecolor=Gray, backgroundcolor=Gray!5,
			nobreak=false
		}
]{notgraybox}

\declaretheoremstyle[
	headfont=\bfseries\sffamily\color{Gray!70!black}, bodyfont=\normalfont,
	mdframed={
			linewidth=2pt,
			rightline=false, topline=false, bottomline=false,
			linecolor=Gray,
			nobreak=false
		}
]{notgrayline}

% \declaretheoremstyle[
% 	headfont=\bfseries\sffamily\color{RawSienna!70!black}, bodyfont=\normalfont,
% 	numbered=no,
% 	mdframed={
% 			linewidth=2pt,
% 			rightline=false, topline=false, bottomline=false,
% 			linecolor=RawSienna, backgroundcolor=RawSienna!1,
% 		},
% 	qed=\qedsymbol
% ]{thmproofbox}

\declaretheoremstyle[
	headfont=\bfseries\sffamily\color{NavyBlue!70!black}, bodyfont=\normalfont,
	numbered=no,
	mdframed={
			linewidth=2pt,
			rightline=false, topline=false, bottomline=false,
			linecolor=NavyBlue, backgroundcolor=NavyBlue!1,
			nobreak=false
		}
]{thmexplanationbox}

\declaretheoremstyle[
	headfont=\bfseries\sffamily\color{WildStrawberry!70!black}, bodyfont=\normalfont,
	numbered=no,
	mdframed={
			linewidth=2pt,
			rightline=false, topline=false, bottomline=false,
			linecolor=WildStrawberry, backgroundcolor=WildStrawberry!1,
			nobreak=false
		}
]{thmanswerbox}

\declaretheorem[style=thmgreenbox, name=Definition, numberwithin=section]{definition}
\declaretheorem[style=thmgreen2box, name=Definition, numbered=no]{definition*}
\declaretheorem[style=thmredbox, name=Theorem, numberwithin=section]{theorem}
\declaretheorem[style=thmred2box, name=Theorem, numbered=no]{theorem*}
\declaretheorem[style=thmredbox, name=Lemma, numberwithin=section]{lemma}
\declaretheorem[style=thmredbox, name=Proposition, numberwithin=section]{proposition}
\declaretheorem[style=thmredbox, name=Corollary, numberwithin=section]{corollary}
\declaretheorem[style=thmblue2box, name=Claim, numbered=no]{claim}

% Redefine proof environment to get a full control. 
\makeatletter
\renewenvironment{proof}[1][\proofname]{\par
	\pushQED{\qed}%
	\normalfont \topsep-2\p@\@plus6\p@\relax
	\trivlist
	\item[\hskip\labelsep
	            \color{RawSienna!70!black}\sffamily\bfseries
	            #1\@addpunct{.}]\ignorespaces
	\begin{mdframed}[linewidth=2pt,rightline=false, topline=false, bottomline=false,linecolor=RawSienna, backgroundcolor=RawSienna!1]
		}{%
		\popQED\endtrivlist\@endpefalse
	\end{mdframed}
}
\makeatother

\declaretheorem[style=thmbluebox, numbered=no, name=Example]{eg}
\declaretheorem[style=thmexplanationbox, numbered=no, name=Proof]{tmpexplanation}
\newenvironment{explanation}[1][]{\vspace{-10pt}\pushQED{\(\circledast\)}\begin{tmpexplanation}}{\null\hfill\popQED\end{tmpexplanation}}

\declaretheorem[style=thmblueline, numbered=no, name=Remark]{remark}
\declaretheorem[style=thmblueline, numbered=no, name=Note]{note}
\declaretheorem[style=thmpinkbox, numbered=no, name=Exercise]{exercise}
\declaretheorem[style=notgrayline, numbered=no, name=As previously seen]{prev}
\declaretheorem[style=thmgreen3box, numbered=no, name=Intuition]{intuition}
\declaretheorem[style=notgraybox, numbered=no, name=Notation]{notation}
\declaretheorem[style=thmpinkbox, numbered=no, name=Problem]{problem}
\declaretheorem[style=thmanswerbox, numbered=no, name=Answer]{tmpanswer}
\newenvironment{answer}[1][]{\vspace{-10pt}\pushQED{\(\circledast\)}\begin{tmpanswer}}{\null\hfill\popQED\end{tmpanswer}}


\usepackage{etoolbox}
\renewcommand{\qed}{\null\hfill\(\blacksquare\)}

\makeatletter

\def\testdateparts#1{\dateparts#1\relax}
\def\dateparts#1 #2 #3 #4 #5\relax{
	\marginpar{\small\textsf{\mbox{#1 #2 #3 #5}}}
}

\def\@lecture{}%
\newcommand{\lecture}[3]{
	\ifthenelse{\isempty{#3}}{%
		\def\@lecture{Lecture #1}%
	}{%
		\def\@lecture{Lecture #1: #3}%
	}%
	\section*{\@lecture}
	\marginpar{\small\textsf{\mbox{#2}}}
}
\usepackage{pgffor}%
\newcommand{\lec}[2]{%
	\foreach \c in {#1,...,#2}{%
			\IfFileExists{Lectures/lec_\c.tex} {%
				\input{Lectures/lec_\c.tex}%
			}{}%
		}%
}

\def\@topic{}%
\newcommand{\topic}[3]{
	\ifthenelse{\isempty{#3}}{%
		\def\@topic{Topic #1}%
	}{%
		\def\@topic{Topic #1: #3}%
	}%
	\section*{\@topic}
	\marginpar{\small\textsf{\mbox{#2}}}
}
\usepackage{pgffor}%
\newcommand{\tpc}[2]{%
	\foreach \c in {#1,...,#2}{%
			\IfFileExists{Topics/tpc_\c.tex} {%
				\input{Topics/tpc_\c.tex}%
			}{}%
		}%
}

% fancy headers
\usepackage{fancyhdr}
\pagestyle{fancy}

% LE: left even
% RO: right odd
% CE, CO: center even, center odd
\fancyhead[LE,RO]{Pingbang Hu}

\fancyhead[RO,LE]{\@lecture} % Right odd,  Left even
\fancyhead[RE,LO]{}          % Right even, Left odd
\fancyfoot[RO,LE]{\thepage}  % Right odd,  Left even
\fancyfoot[RE,LO]{}          % Right even, Left odd
\fancyfoot[C]{\leftmark}     % Center

\makeatother

% notes
\usepackage[color=pink]{todonotes}
\usepackage{marginnote}
\let\marginpar\marginnote

% Fix some stuff
% %http://tex.stackexchange.com/questions/76273/multiple-pdfs-with-page-group-included-in-a-single-page-warning
\pdfsuppresswarningpagegroup=1

% Appendix environment
\usepackage{appendix}
\def\chapterautorefname{Section}
\def\sectionautorefname{Section}
\def\appendixautorefname{Appendix}
\renewcommand\appendixname{Appendix}
\renewcommand\appendixtocname{Appendix}
\renewcommand\appendixpagename{Appendix}
% begin appendix autoref patch [\autoref subsections in appendix](https://tex.stackexchange.com/questions/149807/autoref-subsections-in-appendix)
\makeatletter
\patchcmd{\hyper@makecurrent}{%
	\ifx\Hy@param\Hy@chapterstring
		\let\Hy@param\Hy@chapapp
	\fi
}{%
	\iftoggle{inappendix}{%true-branch
		% list the names of all sectioning counters here
		\@checkappendixparam{chapter}%
		\@checkappendixparam{section}%
		\@checkappendixparam{subsection}%
		\@checkappendixparam{subsubsection}%
		\@checkappendixparam{paragraph}%
		\@checkappendixparam{subparagraph}%
	}{}%
}{}{\errmessage{failed to patch}}

\newcommand*{\@checkappendixparam}[1]{%
	\def\@checkappendixparamtmp{#1}%
	\ifx\Hy@param\@checkappendixparamtmp
		\let\Hy@param\Hy@appendixstring
	\fi
}
\makeatletter

\newtoggle{inappendix}
\togglefalse{inappendix}

\apptocmd{\appendix}{\toggletrue{inappendix}}{}{\errmessage{failed to patch}}
\apptocmd{\subappendices}{\toggletrue{inappendix}}{}{\errmessage{failed to patch}}
% end appendix autoref patch

\setcounter{tocdepth}{1}
\author{Pingbang Hu}
\title{MATH597\\Analysis II}

\thispagestyle{empty}
\addbibresource{ref.bib}

\usetikzlibrary{lindenmayersystems}
\pgfdeclarelindenmayersystem{cantor set}{
  \rule{F -> FfF}
  \rule{f -> fff}
}
\usetikzlibrary{arrows}
\newcommand{\nMAX}{63} 
\newcommand{\xGrSamp}{8} 

\usepackage{pgfplots}
\begin{document}

\maketitle

\begin{abstract}
	Notice that since in this course, the cross-referencing between theorems, lemmas, and propositions are quite complex and
	hard to keep track of, hence in this note, whenever you see a \textbf{!} over \(=\), like \(\overset{!}{=}\), then that
	\textbf{!} is \emph{clickable}! It will direct you to the corresponding theorem, lemma, or proposition we're using to deduce that particular
	equality.

	\par Notice that there are some proofs is \textbf{intended} left as assignments, and for completeness, I put them in \autoref{Apx:Additional-Proofs},
	use it in your \textbf{own risks}! You'll lose the chance to practice and really understand the materials.

	\par Additionally, we'll use Folland\cite{folland1999real} as our main text, while using Tao\cite{tao2013introduction} and Axler\cite{axler2019measure}
	as supplementary references.

	\par This course is taken in Winter 2022, and the date on the covering page is the last updated time.
\end{abstract}

\tableofcontents

\lec{1}{40}

\newpage
%─────Appendix────────────────────────────────────────────────────────────────────────────────────────────────────────────────────────────────────────
\appendix
\appendixpage

%────────────────────────────────────────────────────────────────────────────────────────────────────────────────────────────────────────────────────
\chapter{Additional Proofs}
%────────────────────────────────────────────────────────────────────────────────────────────────────────────────────────────────────────────────────
\section{Seifert-Van Kampen Theorem on Groupoid}\label{thm:Seifert-Van-Kampen-Theorem-on-groupoid}
\begin{theorem}[Seifert-Van Kampen Theorem on groupoid]
	Given \(X_0, X_1, X\) as topological spaces with \(X_0 \cup X_1 = X\). Then the functor \(\Pi \colon \underline{\mathrm{Top}}\to \underline{\mathrm{Gpd}}\) maps the
	\hyperref[def:cocartesian]{cocartesian} diagram in \(\underline{\mathrm{Top} _\ast}\) to a \hyperref[def:cocartesian]{cocartesian} diagram in \(\underline{\mathrm{Gp} }\)
	as follows.
	\[
		\begin{tikzcd}
			{(X_0\cap X_1, x_0)} & {(X_0, x_0)} \\
			{(X_1, x_0)} & {(X, x_0)}
			\arrow["{j_0}", from=1-1, to=1-2]
			\arrow["{i_0}", from=1-2, to=2-2]
			\arrow["{i_1}"', from=2-1, to=2-2]
			\arrow["{j_1}"', from=1-1, to=2-1]
		\end{tikzcd}\overset{\Pi}{\longmapsto}
		\begin{tikzcd}
			{\Pi(X_0\cap X_1)} & {\Pi(X_0)} \\
			{\Pi(X_1)} & {\Pi(X)}
			\arrow["{\Pi (j_0)}", from=1-1, to=1-2]
			\arrow["{\Pi (i_0)}", from=1-2, to=2-2]
			\arrow["{\Pi (i_1)}"', from=2-1, to=2-2]
			\arrow["{\Pi (j_1)}"', from=1-1, to=2-1]
		\end{tikzcd} \]
\end{theorem}
\begin{note}
	Notice that \(X_0, X_1, X\) don't need to be \hyperref[def:path]{path}-connected in particular.
\end{note}

Surprisingly, the proof of \autoref{thm:Seifert-Van-Kampen-Theorem-on-groupoid} is much more elegant with the elementary proof of \autoref{thm:Seifert-Van-Kampen-Theorem}, hence we give
the proof here.
\begin{proof}
	Let \(\mathscr{G} \in \Object (\underline{\mathrm{Gpd} })\) a \hyperref[def:groupoid]{groupoid}, and given \hyperref[def:functor]{functors}
	\[
		F\colon \Pi (X_0)\to \mathscr{G} ,\quad G\colon \Pi (X_1)\to \mathscr{G}
	\]
	such that
	\[
		F\circ \Pi (j_0) = G\circ \Pi (j_1).
	\]
	\[
		\begin{tikzcd}
			{\Pi(X_0\cap X_1)} & {\Pi_1(X_0)} \\
			{\Pi_1(X_1)} & {\Pi_1(X)} \\
			&& {\mathscr{G}}
			\arrow["{\Pi(j_0)}", from=1-1, to=1-2]
			\arrow["{\Pi(i_0)}", from=1-2, to=2-2]
			\arrow["{\Pi(i_1)}"', from=2-1, to=2-2]
			\arrow["{\Pi(j_1)}"', from=1-1, to=2-1]
			\arrow["{\exists!K}"{description}, dashed, from=2-2, to=3-3]
			\arrow["F", curve={height=-12pt}, from=1-2, to=3-3]
			\arrow["G"', curve={height=12pt}, from=2-1, to=3-3]
		\end{tikzcd}
	\]
	We now only need to prove that there exists a unique \hyperref[def:functor]{functor} \(K\colon \Pi (X)\to \mathscr{G} \)  such that the above diagram commutes.

	We can define \(K\) as
	\begin{itemize}
		\item on \hyperref[def:object]{objects}: For all \(x\in \Object (\Pi (X)) = X\),
		      \[
			      K(x) = \begin{dcases}
				      F(x), & \text{ if } x\in X_0 ; \\
				      G(x), & \text{ if } x\in X_1 .
			      \end{dcases}
		      \]
		      This is well-defined since the diagram (without \(K\)) commutes.
		\item on \hyperref[def:morphism]{morphisms}: For every \(p, q\in X\), \(\left< \gamma \right> \colon p\to q\) in \(\Homomorphism _{\Pi (X)}(p, q)\), we need to define
		      \(K(\left< \gamma  \right> )\in \Homomorphism _{\mathscr{G} }(K(p), K(q))\). Our strategy is for every path \(\gamma \) from \(p\) to \(q\), we define
		      \(\widetilde{K} (\gamma )\in \Homomorphism_{\mathscr{G} } (K(p), K(q))\).
		      Then if we also have \(\widetilde{K} (\gamma ) = \widetilde{K} (\gamma ^\prime )\) for \(\gamma \simeq \gamma ^\prime \ \mathrm{rel} \{0, 1\}\), then we can just let
		      \[
			      K(\left< \gamma  \right> ) \coloneqq \widetilde{K} (\gamma ).
		      \]
		      Now we start to construct \(\widetilde{K} \).

		      Given a path \(\gamma \colon [0, 1]\to X\), \(\gamma (0) = p, \gamma (1) = q\). Since \(\Int (X_0) \cup \Int (X_1) = X\), we see that
		      \[
			      \gamma ^{-1} (\Int (X_0)) \cup \gamma ^{-1} (\Int (X_1)) = [0, 1].
		      \]
		      From Lebesgue Lemma\footnote{\url{https://en.wikipedia.org/wiki/Lebesgue\%27s_number_lemma}}, there exists a finite partition
		      \[
			      0 = t_0 < t_1 < \ldots <t_{m-1} < t_{m} = 1
		      \]
		      such that for every \(i\),
		      \[
			      \gamma ([t_{i-1}, t_{i} ])\subset \Int (X_0) \text{ or } \Int (X_1) .
		      \]
		      \begin{figure}[H]
			      \centering
			      \incfig{pf:thm:Seifert-Van-Kampen-Theorem-on-groupoid}
			      \label{fig:pf:thm:Seifert-Van-Kampen-Theorem-on-groupoid}
		      \end{figure}
		      Now, let \(\gamma _{i} \colon [0, 1]\to X, t\mapsto \gamma ((1-t)t_{i-1}+t\cdot t_{i} )\), we see that \(\gamma _{i} \) is either a \hyperref[def:path]{path} in \(X_0\) or \(X_1\).
		      We then define \(\widetilde{K} (\gamma )\coloneqq \widetilde{K} (\gamma _{m} )\circ \widetilde{K} (\gamma _{m-1}) \circ \ldots \circ \widetilde{K} (\gamma _1)\in \Homomorphism _{\mathscr{G} }(K(P), K(q)) \)
		      such that
		      \[
			      \widetilde{K} (\gamma _{i} ) = \begin{dcases}
				      F(\left< \gamma _{i} \right> ), & \text{ if } \gamma _{i} \subset X_0 ; \\
				      G(\left< \gamma _{i} \right> ), & \text{ if } \gamma _{i} \subset X_1 .
			      \end{dcases}
		      \]
		      We need to prove that \(\widetilde{K} (\gamma )\) does not depend on the partition. It's sufficient to prove that for any partition
		      \[
			      0 = t_0 < t_1 < \ldots <t_{m-1} < t_{m} = 1,
		      \]
		      we consider any \textbf{finer} partition
		      \[
			      0 = t_0= t_{10}< t_{11} <\ldots < t_{1K_1}= t_1 = t_{20} <t_{21}<\ldots < t_{mK_{m} } = t_{m} = 1.
		      \]
		      As before, we denote \(\gamma _{ij}\colon [0, 1]\to X, t\mapsto \gamma ((1-t)t_{i j-1} + t\cdot t_{ij} )\). It's clear that as long as
		      \[
			      \widetilde{K} (\gamma _{i} ) = \widetilde{K} (\gamma _{i K_{i} })\circ \widetilde{K} (\gamma _{i K_{i}-1 })\circ \ldots \circ \widetilde{K} (\gamma _{i 0}),
		      \]
		      then our claim is proved. But this is immediate since \(F\) and \(G\) are \hyperref[def:functor]{functor} and for any \(i\), we only use either \(F\) or \(G\) all the time.

		      Now we prove \(\gamma \underset{H}{\simeq }\gamma ^\prime \  \mathrm{rel} \{0, 1\}\), then \(\widetilde{K} (\gamma ) = \widetilde{K} (\gamma ^\prime )\).
		      This is best shown by some diagram.
		      \begin{figure}[H]
			      \centering
			      \incfig{pf:thm:Seifert-Van-Kampen-Theorem-on-groupoid-2}
			      \label{fig:pf:thm:Seifert-Van-Kampen-Theorem-on-groupoid-2}
		      \end{figure}
		      The left-hand side represents a partition \(\mathcal{\MakeUppercase{p}} \) of \([0, 1]\times [0, 1]\) such that every small square's image in \(X\) under \(H\) is either entirely in \(X_0\)
		      or in \(x_1\). Consider all paths from \((0, 0)\) to \((1, 1)\) such that it only goes right or up. We see that for any such path \(L\), consider
		      \[
			      \gamma _{L} \colon [0, 1]\to L, \quad t\mapsto \gamma _{L} (t).
		      \]
		      We let \(\Gamma _{L} \colon \at{H}{L}{} \circ \gamma _{L} \colon [0, 1]\to X\), we see that \(\Gamma _{L} \) is a \hyperref[def:path]{path} from \(p\) to \(q\). Now, if
		      for two paths \(L_1\) and \(L_2\) such that they only differ from \underline{a square}.
		      \begin{figure}[H]
			      \centering
			      \incfig{pf:thm:Seifert-Van-Kampen-Theorem-on-groupoid-3}
			      \label{fig:pf:thm:Seifert-Van-Kampen-Theorem-on-groupoid-3}
		      \end{figure}
		      We claim that \(\gamma _{L_1}, \Gamma _{L_2}\) are two \hyperref[def:path]{paths} from \(p\) to \(q\), and \(\widetilde{K} (\Gamma _{L_1}) = \widetilde{K} (\Gamma _{L_2})\).
		      Now, we denote \(\Gamma _0\) and \(\Gamma _1\) as follows.
		      \begin{figure}[H]
			      \centering
			      \incfig{pf:thm:Seifert-Van-Kampen-Theorem-on-groupoid-4}
			      \caption{The definition of \(\Gamma _0\) and \(\Gamma _1\).}
			      \label{fig:pf:thm:Seifert-Van-Kampen-Theorem-on-groupoid-4}
		      \end{figure}
		      It's clearly that by only finitely many steps, we can transform \(\Gamma _0\) to \(\Gamma _1\), hence
		      \[
			      \widetilde{K} (\Gamma _0) = \widetilde{K} (\Gamma _1).
		      \]
		      Finally, we observe that
		      \[
			      \widetilde{K} (\gamma _0) = \widetilde{K} (\Gamma _0) = \widetilde{K} (\Gamma _1) = \widetilde{K} (\gamma _1).
		      \]

		      If we now define \(K(\left< \gamma  \right> ) = \widetilde{K} (\gamma )\), then \(K\colon \Morphism (\Pi (X))\to \Morphism (\mathscr{G} )\), then it's well-defined.
	\end{itemize}

	We now prove \(K\colon \Pi (X)\to \mathscr{G}\) is indeed a \hyperref[def:functor]{functor}. But this is immediate from the definition of \(K\), namely it'll send identity to identity and
	the composition associates.

	Also, we need to prove that the following diagram commutes.
	\[
		\begin{tikzcd}
			{\Pi(X_0\cap X_1)} & {\Pi_1(X_0)} \\
			{\Pi_1(X_1)} & {\Pi_1(X)} \\
			&& {\mathscr{G}}
			\arrow["{\Pi(j_0)}", from=1-1, to=1-2]
			\arrow["{\Pi(i_0)}", from=1-2, to=2-2]
			\arrow["{\Pi(i_1)}"', from=2-1, to=2-2]
			\arrow["{\Pi(j_1)}"', from=1-1, to=2-1]
			\arrow["{K}"{description}, from=2-2, to=3-3]
			\arrow["F", curve={height=-12pt}, from=1-2, to=3-3]
			\arrow["G"', curve={height=12pt}, from=2-1, to=3-3]
		\end{tikzcd}
	\]
	But this is again trivial.

	Finally, we need to show that such \(K\) is unique. This is the same as the proof of \autoref{lma:lec7}, hence the proof is done.
\end{proof}

%────────────────────────────────────────────────────────────────────────────────────────────────────────────────────────────────────────────────────
\section{An alternative proof of \hyperref[thm:Seifert-Van-Kampen-Theorem]{Seifert Van-Kampen Theorem}}\label{pf:an-alternative-proof-of-Seifert-Van-Kampen-thm}
\begin{theorem}
	We claim that the diagram
	\[
		\begin{tikzcd}
			{\pi_1(X_0\cap X_1, x_0)} & {\pi_1(X_0, x_0)} \\
			{\pi_1(X_1, x_0)} & {\pi_1(X, x_0)} \\
			\arrow["{(j_0)_\ast}", from=1-1, to=1-2]
			\arrow["{(i_0)_\ast}", from=1-2, to=2-2]
			\arrow["{(i_1)_\ast}"', from=2-1, to=2-2]
			\arrow["{(j_1)_\ast}"', from=1-1, to=2-1]
		\end{tikzcd}
	\]
	is \hyperref[def:cocartesian]{cocartesian}.
\end{theorem}
\begin{proof}
	The basic idea is that, for this diagram,
	\[
		\begin{tikzcd}
			{\Pi(X_0\cap X_1)} & {\Pi(X_0)} \\
			{\Pi(X_1)} & {\Pi(X)} \\
			\arrow[from=1-1, to=1-2]
			\arrow[from=1-2, to=2-2]
			\arrow[from=2-1, to=2-2]
			\arrow[from=1-1, to=2-1]
		\end{tikzcd}
	\]
	we want to construct a \hyperref[def:morphism]{morphism} \(r\colon \Pi (Z)\to \pi _1(Z, p)\) in \(\underline{\mathrm{Gpd}}\) such that
	\(Z = X_0 \cap X_1, X_0, X_1, X\). For every \(x\in Z\), we fix a \hyperref[def:path]{path} \(\gamma _{x} \) such that it connects \(p\) and \(x\) and satisfies
	\begin{enumerate}[(1)]
		\item If \(x\in X_0 \cap X_1\), then \(\im  (\gamma _{x} )\subset X_0 \cap X_1\)
		\item If \(x\in X_0\), then \(\im  (\gamma _{x} )\subset X_0\)
		\item If \(x\in X_1\), then \(\im  (\gamma _{x} )\subset X_1\)
		\item \(\gamma _p = c_p\)
	\end{enumerate}

	The proof is given in \url{https://www.bilibili.com/video/BV1P7411N7fW?p=38&spm_id_from=pageDriver}.
	\todo{If have time.}
\end{proof}

%────────────────────────────────────────────────────────────────────────────────────────────────────────────────────────────────────────────────────
\section{Cellular Boundary Formula in \autoref{def:cellular-chain-complex}}\label{pf:cellular-boundary-formula-is-well-defined}
\begin{theorem}
	For \(n>1\), the \hyperref[def:boundary-homomorphism]{boundary maps} \(\partial _n\) of \hyperref[def:cellular-chain-complex]{cellular chain complex}
	given by
	\[
		\partial _n(e_\alpha^n) = \sum_\beta \partial _{\alpha\beta} e_\beta^{n - 1}
	\]
	is well-defined.
\end{theorem}
\begin{proof}
	Here we are identifying the \hyperref[def:cell]{cells} \(e^n_\alpha \) and \(e^{n-1}_\beta\) with generators of the corresponding summands of the
	\hyperref[def:cellular-chain-complex]{cellular chain groups}, namely \(C_n(X)\). The summation in the formula contains only finitely many terms
	since the \hyperref[def:attaching-map]{attaching map} of \(e^n_\alpha \) has compact image, so this image meets only finitely many \hyperref[def:cell]{cells} \(e^{n-1}_\beta\).
	To derive the cellular boundary formula, consider the following commutative diagram.
	\[
		\begin{tikzcd}[column sep=tiny]
			{H_n(D^n_\alpha, \partial D^n_\alpha)} & {\widetilde{H}_{n-1}(\partial D^n_\alpha)} & {\widetilde{H}_{n-1}(S^{n-1}_\beta)} \\
			{H_n(X^n, X^{n-1})} & {\widetilde{H}_{n-1}(X^{n-1})} & {\widetilde{H}_{n-1}(\quotient{X^{n-1}}{X^{n-2}})} \\
			& {H_{n-1}(X^{n-1}, X^{n-2})} & {H_{n-1}(\quotient{X^{n-1}}{X^{n-2}}, \quotient{X^{n-2}}{X^{n-2}})}
			\arrow["{{\Phi_{\alpha}}_\ast}", from=1-1, to=2-1]
			\arrow["\partial", from=1-1, to=1-2]
			\arrow["{\partial_n}", from=2-1, to=2-2]
			\arrow["\cong"', draw=none, from=1-1, to=1-2]
			\arrow["{{\Delta_{\alpha \beta}}_{\ast}}", from=1-2, to=1-3]
			\arrow["{{q_{\beta}}_{\ast}}"', from=2-3, to=1-3]
			\arrow["\cong", from=2-3, to=3-3]
			\arrow["\cong", from=3-2, to=3-3]
			\arrow["{q_\ast}", from=2-2, to=2-3]
			\arrow["{{\varphi_\alpha}_\ast}", from=1-2, to=2-2]
			\arrow["{j_{n-1}}", from=2-2, to=3-2]
			\arrow["{d_n}"', from=2-1, to=3-2]
		\end{tikzcd}
	\]
	where
	\begin{itemize}
		\item \(\Phi _\alpha \) is the characteristic map of the \hyperref[def:cell]{cell} \(e^n_\alpha \) and \(\varphi _\alpha \) is its \hyperref[def:attaching-map]{attaching map}.
		\item \(q\colon X^{n-1} \to \quotient{X^{n-1} }{X^{n-2} } \) is the quotient map.
		\item \(q_\beta \colon \quotient{X^{n-1} }{X^{n-2} } \to S^{n-1}_\beta\) collapses the complement of the \hyperref[def:cell]{cell} \(e^{n-1}_\beta\) to a point, the resulting
		      \hyperref[CW-complex-quotient]{quotient} sphere being identified with \(S^{n-1}_\beta\)\footnote{Which is just \(\quotient{D^{n-1}_\beta}{\partial D^{n-1}_\beta}\).}
		      via the characteristic map \(\Phi _\beta \).
		\item \(\Delta _{\alpha \beta }\colon \partial D^n_\alpha \to S^{n-1}_\beta\) is the composition \(q_\beta q \varphi _\alpha \), i.e., the
		      \hyperref[def:attaching-map]{attaching map} of \(e^n_\alpha \) followed by the quotient map \(X^{n-1}\to S^{n-1}_\beta\) collapsing the complement of \(e^{n-1}_\beta\)
		      in \(X^{n-1}\) to a point.
	\end{itemize}

	The map \({\Phi _\alpha }_\ast\) takes a chosen generator \([D^n_\alpha ]\in H_n(D^n_\alpha , \partial D^n_\alpha )\) to a generator of the \(\mathbb{\MakeUppercase{z}} \)
	summand of \(H_n(X^n, X^{n-1})\) corresponding to \(e^n_\alpha \). Letting \(e^n_\alpha \) denote this generator, commutativity of the left half
	of the diagram then gives
	\[
		\partial _n(e^n_\alpha )= j_{n-1}{\varphi _\alpha }_\ast\partial [D^n_\alpha ].
	\]
	In terms of the basis for \(H_{n-1}(X^{n-1}, X^{n-2} )\) corresponding to the \hyperref[def:cell]{cells} \(e^{n-1}_\beta\), the map
	\({q_\beta }_\ast\) is the projection of \(\widetilde{H} _{n-1}(\quotient{X^{n-1} }{X^{n-2}})\) onto its \(\mathbb{\MakeUppercase{z}} \)
	summand corresponding to \(e^{n-1}_\beta\). Commutativity of the diagram then yields the formula for \(\partial _n\) given above.
\end{proof}

%────────────────────────────────────────────────────────────────────────────────────────────────────────────────────────────────────────────────────
\chapter{Algebra}
This section aims to give some reference about the required algebra content, including \hyperref[def:Abelian-group]{Abelian groups}, and
\hyperref[def:free-Abelian-group]{free Abelian group}, which is used heavily when discuss homology, and also some \hyperref[sec:homological-algebra]{homological algebra},
where we will focus on \hyperref[def:exact-sequence]{exact sequence} specifically.
%────────────────────────────────────────────────────────────────────────────────────────────────────────────────────────────────────────────────────
\section{Abelian Group}
\begin{definition}[Abelian group]\label{def:Abelian-group}
	A group \((G, \cdot)\) is an \emph{Abelian group} if for every \(a, b\in G\), we have
	\[
		a\cdot b = b\cdot a.
	\]
	We often denote \(\cdot\) as \(+\) if \((G, \cdot)\) is a \hyperref[def:Abelian-group]{Abelian group}.
\end{definition}

\begin{definition}[Product of groups]\label{def:product-of-groups}
	Given two groups \((G, \cdot), (H, \cdot)\), the \emph{product of \(G\) and \(H\)}, denoted by \(G\times H\) is defined as
	\[
		G\times H = \left\{(g, h)\mid g\in G, h\in H\right\}
	\]
	and
	\[
		(g_1, h_1)\cdot (g_2, h_2)\coloneqq (g_1\cdot g_2, h_1\cdot h_2).
	\]
\end{definition}

\begin{notation}
	For simplicity, given an index set \(I\), we'll denote the order pair \((g_{\alpha _1}, g_{\alpha _2}, \ldots)\) as \((g_\alpha )_{\alpha \in I}\). Note that
	the latter notation can handle the case that \(I\) is either countable or uncountable, while the former can only handle the countable case.
\end{notation}

\begin{definition}[Direct product]\label{def:direct-product}
	Given \((G_\alpha , +)\), \(\alpha \in I\) as a collection of \hyperref[def:Abelian-group]{Abelian group}, we define their \emph{direct product} as
	\[
		\left(\prod\limits_{\alpha \in I}G_\alpha , + \right),
	\]
	where
	\[
		\prod\limits_{\alpha \in I} G_\alpha = \left\{(g_\alpha )_{\alpha \in I}\mid g_\alpha \in G_\alpha \right\}
	\]
	and \(\forall (g_\alpha ), (h_\alpha )\in \prod\limits_{\alpha \in I} G_\alpha \)
	\[
		(g_\alpha )+(h_\alpha ) \coloneqq g_\alpha + h_\alpha
	\]
	for all \(\alpha \in I\).

	Specifically, if \(I\) is finite, namely there are only finely many \hyperref[def:Abelian-group]{Abelian groups} \((G_1, +),\ldots , (G_n, +)\), and
	\(\left(\prod\limits_{i=1}^{n} G_{i} , +\right)\) can be denoted as
	\[
		\left(G_1 \times \ldots \times G_n, + \right).
	\]
\end{definition}

\begin{definition}[External direct sum]\label{def:external-direct-sum}
	Given a collection of \hyperref[def:Abelian-group]{Abelian groups} \(\{G_\alpha \}_{\alpha \in I}\), the \emph{external direct sum} of them, denoted as \(\left(\bigoplus_{\alpha \in I} G_\alpha , +\right)\)
	as
	\[
		\bigoplus_{\alpha \in I}G_\alpha \coloneqq \left\{(g_\alpha )_{\alpha \in I}\mid \underset{\alpha \in I}{\forall }\ g_\alpha \in G_\alpha, \text{\# non-zero elements in \(g_{\alpha} < \infty\)}\right\}.
	\]
	And for every \((g_\alpha ), (h_\alpha )\in \bigoplus_{\alpha \in I}G_\alpha \),
	\[
		(g_\alpha )+(h_\alpha )\coloneqq g_\alpha +h_\alpha
	\]
	for all \(\alpha \in I\).\footnote{This may not be the best notation: What we're really trying to say is \((g_\alpha )_{\alpha \in I}+ (h_\alpha )_{\alpha \in I} \coloneqq g_i + h_i \) for all \(i \in I\).}
\end{definition}
\begin{note}
	We see that
	\[
		\bigoplus_{\alpha \in I}G_\alpha \subset \prod\limits_{\alpha \in I}G_\alpha.
	\]
	Additionally, we also have
	\[
		\left(\bigoplus_{\alpha \in I}G_\alpha , +\right)< \left(\prod\limits_{\alpha \in I}G_\alpha , +\right).
	\]
\end{note}
\begin{remark}
	We see that the operation \(+\) is indeed closed since the sum of \(g, g^\prime \in \bigoplus_{\alpha \in I} G_\alpha \) will have only finitely non-zero elements if
	\(g, g^\prime \) both have only finitely many non-zero elements.
\end{remark}

We see that if \(I\) is a finite index set, given a collection of \hyperref[def:Abelian-group]{Abelian group} \(\{G_\alpha \}_{\alpha \in I}\), then
\[
	G_1 \times \ldots \times G_n = G_{1}  \oplus  \ldots \oplus G_n.
\]

\begin{definition}[Internal direct sum]\label{def:internal-direct-sum}
	Given an \hyperref[def:Abelian-group]{Abelian group} \(G\), and a collection of the subgroups \(\{G_\alpha \}_{\alpha \in I}\) of \(G\), we say \(G\) is an \emph{internal direct sum}
	of \(\{G_\alpha \}_{\alpha \in I}\) if for any \(g\in G\), we can write
	\[
		g = \sum\limits_{\alpha \in I} g_\alpha
	\]
	\textbf{uniquely}, where \(g_\alpha \in G_\alpha \) has only finitely many non-zero elements. In this case, we denote
	\[
		G = \bigoplus_{\alpha \in I}G_\alpha .
	\]
\end{definition}
Intuitively, the \hyperref[def:external-direct-sum]{external direct sum} is to build a new group based on the given collection of groups \(\{G_\alpha \}_{\alpha \in I}\), while the internal direct sum is
to express an \textbf{already known} group \(G\) with an \textbf{already known} collection of groups \(\{G_\alpha \}_{\alpha \in I}\).

\begin{remark}[Relation between Internal and External direct sum]\label{rmk:relation-between-internal-and-externam-direct-sum}
	Given an \hyperref[def:Abelian-group]{Abelian group} \(G\) and its \hyperref[def:internal-direct-sum]{internal direct sum} decomposition \(\bigoplus_{\alpha \in I}G_\alpha \),
	\(G\) is isomorphic to the \hyperref[def:external-direct-sum]{external direct sum} of \(\{G_\alpha \}_{\alpha \in I}\). We see this from the following group homomorphism:
	\[
		\underset{g\in G}{\forall }\ g = \sum\limits_{\alpha \in I}g_\alpha \mapsto (g_\alpha )_{\alpha \in I}.
	\]

	Conversely, given a collection of \hyperref[def:Abelian-group]{Abelian group} \(\{G_\alpha \}_{\alpha \in I}\), and let \(G = \bigoplus_{\alpha \in I}G_\alpha \) as the
	\hyperref[def:external-direct-sum]{external direct sum} of \(\{G_\alpha \}\), denote
	\(i_{\alpha_0 }\colon G_{\alpha_0} \to \bigoplus_{\alpha \in I}G_\alpha\)
	as a canonical embedding
	\[
		g_{ \alpha_0 }\mapsto i_{ \alpha_0 } (g_{ \alpha_0 }) = (h_\alpha)_{\alpha\in I},
	\]
	where
	\[
		h_\alpha  = \begin{dcases}
			g_{ \alpha_0 }, & \text{ if } \alpha_0 = \alpha  ; \\
			0,              & \text{ if } \alpha_0 \neq \alpha
		\end{dcases}
	\]
	given \(\alpha_0\). Then
	\[
		G_{\alpha_0} ^\prime \coloneqq i_{\alpha_0} (G_{\alpha_0}) < \bigoplus _{\alpha \in I}G_\alpha
	\]
	and \(G\) is the \hyperref[def:internal-direct-sum]{internal direct sum} of \(G^\prime _{\alpha_0} \), \(\alpha_0\in I\). This is because
	\(\forall g= (g_\alpha )_{\alpha \in I}\in G(= \bigoplus_{\alpha \in I}G_\alpha )\), we have
	\[
		g = \sum\limits_{\alpha \in I} i_\alpha (g_\alpha ).
	\]
	Note that the above sum is well-defined since there are only finitely many non-zero elements for each \(g_\alpha \). And additionally, we can
	see the uniqueness of this decomposition by defining \(\pi _{\alpha_0} \) such that
	\[
		\pi _{\alpha_0} \colon \bigoplus_{\alpha \in I}G_\alpha \to G_{\alpha_0} , \quad (g_\alpha )_{\alpha \in I}\mapsto g_{\alpha_0} ,
	\]
	then \(\pi _\alpha \circ i_\alpha = \identity_{G_\alpha } \), \(\pi _\alpha \circ i_\beta = 0\) for all \(\beta \neq \alpha \) and
	\[
		\pi _\beta (g) = \pi _\beta \left(\sum\limits_{\alpha \in I}^{}i_\alpha (g_\alpha ) \right) = \sum\limits_{\alpha \in I}\pi _\beta \circ i_\alpha (g_\alpha ) = \pi _\beta \circ i_\beta (g_\beta ) = g_\beta
	\]
	for all \(\beta \in I\), where the second equality is because this summation is finite. Hence, we have
	\[
		g = \sum\limits_{\alpha \in I}i_\alpha (\pi _\alpha (g)).
	\]
\end{remark}

\begin{definition}
	Given two \hyperref[def:Abelian-group]{Abelian groups} \(G, H\), we define \(\Homomorphism (G, H)\) as
	\[
		\Homomorphism (G, H) \coloneqq \left\{f\colon G\to H \mid \text{\(f\) is a group homomorphism} \right\},
	\]
	then we can define
	\[
		\begin{split}
			+\colon \Homomorphism (G, H)\times \Homomorphism (G, H) &\to \Homomorphism (G, H)\\
			(\varphi , \psi )&\mapsto \varphi + \psi,
		\end{split}
	\]
	where
	\[
		(\varphi + \psi )(g)\coloneqq \varphi (g) + \psi (g).
	\]
\end{definition}

\begin{remark}[Relation between direct sum and direct product]\label{rmk:relation-between-direct-sum-and-direct-product}
	Given a collection of \hyperref[def:Abelian-group]{Abelian groups} \(\{G_\alpha \}_{\alpha \in I}\), and another \hyperref[def:Abelian-group]{Abelian group} \(H\), there exists a
	\(\varphi \) such that
	\[
		\begin{split}
			\varphi \colon \Homomorphism \left(\bigoplus_{\alpha \in I}G_\alpha , H\right)&\to \prod\limits_{\alpha \in I}\Homomorphism (G_\alpha , H)\\
			f&\mapsto \varphi (f)\coloneqq (f_\alpha )_{\alpha \in I}
		\end{split}
	\]
	where \(f_\alpha = f\circ i_\alpha \), where \(i_\alpha \) is the canonical embedding from \(G_\alpha \) to \(\bigoplus_{\alpha \in I}G _\alpha \). We claim
	that \(\varphi \) is an isomorphism.

	\begin{itemize}
		\item \(\varphi \) is injective. This is obvious since \(\ker  (\varphi ) = 0\) from the fact that if \(\varphi (f) = 0\), then \(f_\alpha  = 0\) for all \(\alpha\), hence
		      \(f\) is \(0\).
		\item \(\varphi \) is surjective. For every \((f_\alpha )_{\alpha \in I}\in \prod\limits_{\alpha \in I}\Homomorphism (G_\alpha , H) \), we define
		      \[
			      \begin{split}
				      f\colon \bigoplus_{\alpha \in I}G_\alpha &\to H\\
				      \sum\limits_{\alpha \in I}^{}g_\alpha &\mapsto \sum\limits_{\alpha \in I}f_\alpha (g_\alpha ).
			      \end{split}
		      \]
		      We see that \(f\in \Homomorphism \left(\bigoplus_{\alpha \in I}G_\alpha , H\right)\) and \(\varphi (f) = (f_\alpha )_{\alpha \in I}\).
	\end{itemize}

	This shows that
	\[
		\Homomorphism \left(\bigoplus_{\alpha \in I}G_\alpha , H\right)\cong \prod\limits_{\alpha \in I}\Homomorphism (G_\alpha , H).
	\]
\end{remark}

\begin{exercise}
	We can show that
	\[
		\Homomorphism \left(H, \prod\limits_{\alpha \in I}G_\alpha\right) \cong \prod\limits_{\alpha \in I}\Homomorphism (H, G_\alpha).
	\]
	Note the order in the \(\Homomorphism \) matters.
\end{exercise}

%────────────────────────────────────────────────────────────────────────────────────────────────────────────────────────────────────────────────────
\section{Free Abelian Group}
\begin{definition}[Free Abelian group]\label{def:free-Abelian-group}
	Given an  \hyperref[def:Abelian-group]{Abelian group} \((G, +)\), we say \(G\) is a \emph{free Abelian group} if there exists a collection of elements \(\{g_\alpha \}_{\alpha \in J}\) in \(G\) such that
	\(\{g_\alpha \}_{\alpha \in J}\) forms a \textbf{basis} of \(G\), i.e., for all \(g\in G\), \(\exists ! n _\alpha \in \mathbb{\MakeUppercase{z}} \) for all \(\alpha \in J\) such that
	\[
		g = \sum\limits_{\alpha \in J} n_\alpha g_\alpha
	\]
	with finitely many non-zero \(n_\alpha\).
\end{definition}
\begin{remark}
	If \(G\) is a \hyperref[def:free-Abelian-group]{free Abelian group}, and \(\{g_\alpha \}_{\alpha \in J}\) is a basis, then for every \(\alpha \in J\), \(\left< g_\alpha \right> \) is an
	infinite cyclic group since
	\[
		n\cdot g_\alpha = 0 = 0\cdot g_\alpha \implies n = 0.
	\]
	And from \autoref{def:free-Abelian-group}, we have
	\[
		G = \bigoplus_{\alpha \in J}\left< g_\alpha  \right>.
	\]

	Conversely, assume there are a collection of infinite cyclic group \(\left< g_\alpha  \right> \) for \(\alpha \in I\) in \(G\) such that
	\[
		G = \bigoplus_{\alpha \in I}\left< g_\alpha  \right>,
	\]
	then \(\{g_\alpha \}_{\alpha \in I}\) is a basis of \(G\), hence \(G\) is a \hyperref[def:free-Abelian-group]{free Abelian group}.
\end{remark}

\begin{proposition}
	If \(G\) is an  \hyperref[def:Abelian-group]{Abelian group}, then the following are equivalent.
	\begin{enumerate}[(1)]
		\item \(G\) is a \hyperref[def:free-Abelian-group]{free Abelian group}.
		\item \(G\) is an \hyperref[def:internal-direct-sum]{internal direct sum} of some infinite cyclic groups.
		\item \(G\) is isomorphic to the \hyperref[def:external-direct-sum]{external direct sum} of some additive groups of integers \(\mathbb{\MakeUppercase{z}} \).
	\end{enumerate}
\end{proposition}
\begin{proof}
	We see that \(1. \iff 2.\) is already proved. And for \(2. \iff 3.\), this follows directly from the \hyperref[rmk:relation-between-internal-and-externam-direct-sum]{relation between internal and external direct sum}.
\end{proof}

Now, consider \(G\) as a \hyperref[def:free-Abelian-group]{free Abelian group}, then
\[
	u\colon \begin{tikzcd}
		G & {\bigoplus\limits_{\alpha\in I} \mathbb{Z}}
		\arrow["\cong", from=1-1, to=1-2]
	\end{tikzcd}
\]
for some \(I\). Denote \(e_\alpha \coloneqq i_\alpha (1)\in \bigoplus_{\alpha \in I}\mathbb{\MakeUppercase{z}} \), where \(i_\alpha \colon \mathbb{\MakeUppercase{z}} \to \bigoplus_{\alpha \in I}\mathbb{\MakeUppercase{z}} \) is the
canonical embedding, i.e., \(e_\alpha = (g_\alpha )_{\alpha \in I}\in \bigoplus_{\alpha \in I}\mathbb{\MakeUppercase{z}} \), where
\[
	g_\beta = \begin{dcases}
		1, & \text{ if } \beta = \alpha;    \\
		0, & \text{ if } \beta \neq \alpha.
	\end{dcases}
\]
Moreover, denote \(\epsilon _\alpha \) as the image of \(e_\alpha \) under the isomorphism \(u\), namely \(\epsilon _\alpha = u^{-1} (e_\alpha )\), then \(\{\epsilon _\alpha \}_{\alpha \in I}\) is a basis of \(G\).

Now, for every \hyperref[def:Abelian-group]{Abelian group} \(H\), we have
\[
	\begin{tikzcd}[column sep=small]
		{\Homomorphism(G, H)} && {\Homomorphism\left(\bigoplus\limits_{\alpha\in I}\mathbb{Z}, H\right)} & f && {f\circ u^{-1}} \\
		&& {\prod\limits_{\alpha\in I}\Homomorphism\left(\mathbb{Z}, H\right)} &&& {(f\circ u^{-1}\circ i_{\alpha})_{\alpha\in I}} \\
		&& {\prod\limits_{\alpha\in I}H} &&& {\left(f\circ u^{-1}\circ i_\alpha(1)\right)_{\alpha\in I}}
		\arrow[maps to, from=1-6, to=2-6]
		\arrow[maps to, from=2-6, to=3-6]
		\arrow[dashed, maps to, from=1-4, to=3-6]
		\arrow[maps to, from=1-4, to=1-6]
		\arrow["\varphi", from=1-3, to=2-3]
		\arrow["\cong", from=2-3, to=3-3]
		\arrow["{\circ u}"', from=1-3, to=1-1]
		\arrow["\cong"', dashed, from=1-1, to=3-3]
		\arrow["\cong"', draw=none, from=1-3, to=2-3]
		\arrow["\cong"', draw=none, from=1-1, to=1-3]
	\end{tikzcd}
\]
where \(\varphi\) is the homeomorphism defined in \hyperref[rmk:relation-between-direct-sum-and-direct-product]{here}, and the homeomorphism
\[
	\prod\limits_{\alpha \in I}\Homomorphism (\mathbb{\MakeUppercase{z}} , H) \overset{\cong}{\longrightarrow} \prod\limits_{\alpha \in I}H
\]
is trivial since every \(f\in \prod\limits_{\alpha \in I}\Homomorphism (\mathbb{\MakeUppercase{z}} , H)\) corresponds to \(f(1)\in H\) uniquely. We see that
\[
	f\circ u^{-1} \circ i_\alpha (1) = f\circ u^{-1} (e_\alpha )= f(\epsilon _\alpha ).
\]
In other words, for all \hyperref[def:Abelian-group]{Abelian group} \(H\), a morphism from the set \(\{\epsilon _\alpha \}_{\alpha \in I}\) to \(H\) can be uniquely extended to the group
a homomorphism from \(G\) to \(H\).

\begin{remark}
	This means, to determine \(\Homomorphism (G, H)\), we only need to determine where each base element in \(G\) will map to in \(H\), and this is why it's \emph{free}.
\end{remark}

We now want to generate \hyperref[def:free-Abelian-group]{free Abelian group} by a set. Roughly speaking, given a set \(S\), we can generate a \hyperref[def:free-Abelian-group]{free Abelian group} \(Z\)
by defining
\[
	Z\coloneqq \left\{\sum\limits_{x\in S}^{} n_{x} x\mid n_{x} \in \mathbb{\MakeUppercase{z}} , \text{\# non-zero elements in \(n_x < \infty\)}\right\}
\]
with the naturally defined \(+\). Formally, we have the following.

\begin{definition}[Free Abelian group generated by a set]\label{def:free-Abelian-group-generated-by-sets}
	Given a set \(S\), the \emph{\hyperref[def:free-Abelian-group]{free Abelian group} generated by \(S\)} \((Z, +)\) is defined as
	\[
		Z\coloneqq \left\{f\colon S\to \mathbb{\MakeUppercase{z}} \mid \text{only finitely many \(x\in S\) such that \(f(x)\neq 0\)}\right\},
	\]
	with
	\[
		\begin{split}
			+\colon Z\times Z&\to Z\\
			(f, g)&\mapsto f + g.
		\end{split}
	\]
\end{definition}

\begin{remark}
	\(\{\phi _{x} \mid x\in S\}\) forms a basis of \(Z\), where \(\phi _{x} \colon S\to \mathbb{\MakeUppercase{z}} \) such that
	\[
		y\mapsto \phi _{x} (y) = \begin{dcases}
			1, & \text{ if } y=x ;    \\
			0, & \text{ if } y \neq x
		\end{dcases}
	\]
	is the characteristic function at \(x\). We see this by for all \(f\in S\), \(f = \sum\limits_{x\in S}^{} f(x)\phi _{x} \), which is uniquely defined. Hence, \((Z, +)\) is a \hyperref[def:free-Abelian-group]{free Abelian group}.
\end{remark}

\begin{note}
	Note that
	\[
		\begin{split}
			S &\overset{1:1}{\longleftrightarrow} \{\phi _{x} \mid x\in S\}\\
			x &\mapsto \phi _{x}.
		\end{split}
	\]
	Hence, we often denote the element \(\sum\limits_{x\in S}^{} \underbrace{n_{x}}_{f(x)} \phi _{x} \) in \(Z\) as
	\[
		\sum\limits_{x\in S}n_{x} \cdot x.
	\]
\end{note}
\begin{theorem}[The universal property of free Abelian group generated by a set]\label{thm:universal-property-of-free-Abelian-group-generated-by-a-set}
	Denote a canonical embedding \(i\colon S\to Z\), \(x\mapsto \phi _{x} \). Then for all \hyperref[def:Abelian-group]{Abelian group} \(H\) and \(f\colon S\to H\), there exists a unique group homomorphism
	\[
		\widetilde{f} \colon Z\to H
	\]
	such that \(\widetilde{f} \circ i = f\).
\end{theorem}
\begin{proof}
	We define
	\[
		\widetilde{f} \left(\sum\limits_{x\in S}^{} n_{x} \cdot x\right)\coloneqq \sum\limits_{x\in S}n_{x} f(x),
	\]
	and the uniqueness is obvious.
\end{proof}

Note that we can use the above \hyperref[thm:universal-property-of-free-Abelian-group-generated-by-a-set]{universal property} to describe a
\hyperref[def:free-Abelian-group]{free Abelian group} since we have the following.
\begin{proposition}
	Given \(Z^\prime \) as another \hyperref[def:Abelian-group]{Abelian group} and \(i^\prime \colon S\to Z^\prime \) as another canonical embedding such that for all \hyperref[def:Abelian-group]{Abelian group} \(H\)
	and \(f\colon S\to H\), there exists a unique group homomorphism \(\widetilde{f} \colon Z^\prime \to H\) such that \(\widetilde{f} \circ i^\prime = f\), then
	\[
		Z^\prime \cong Z.
	\]
\end{proposition}

Namely, we can describe a \hyperref[def:free-Abelian-group]{free Abelian group} by its \hyperref[thm:universal-property-of-free-Abelian-group-generated-by-a-set]{universal property}
uniquely up to isomorphism.

\begin{theorem}
	Assume \(G\) is a \hyperref[def:free-Abelian-group]{free Abelian group}. Assume there exists a finite basis \(\{g_1, \ldots , g_{n}  \}\) of \(G\), and also assume that there exists another basis
	\(\{h_\alpha \}_{\alpha \in I}\). Then we have
	\[
		\mathrm{card}(I)< \infty,
	\]
	specifically, we have
	\[
		\mathrm{card} (I)= n.
	\]
\end{theorem}
\begin{proof}
	Firstly, we observe that if we can show
	\[
		\mathrm{card} (I) \leq n,
	\]
	then by swapping \(\{h_\alpha \}_{\alpha \in I}\) and \(\{g_\alpha \}_{\alpha \in I}\), we will have \(\mathrm{card}(I) = n \).

	Suppose \(I\) is an infinite set, then we can find \(h_{\alpha _1}, \ldots , h_{\alpha _m} \) such that \(m>n\) and \(h_{\alpha _i} \neq h_{\alpha _j}\) for \(i \neq j\).
	Then since \(\{g_\alpha \}_{\alpha \in I}\) is a basis, we have
	\[
		h_{\alpha _{i} } = \sum\limits_{j=1}^{n} k_{i} ^j g_{j} , \forall i = 1, \ldots , m.
	\]
	Specifically, we have
	\[
		\begin{pmatrix}
			h_{\alpha _1} \\
			\vdots        \\
			h_{\alpha _m} \\
		\end{pmatrix} = \underbrace{\begin{pmatrix}
				k_1^1  & k_1^2 & \ldots & k_1^{n} \\
				\vdots &       & \ddots & \vdots  \\
				k_m^1  & k_m^2 & \ldots & k_m^{n} \\
			\end{pmatrix}}_{K\in M_{m\times n}(\mathbb{\MakeUppercase{z}})\subset M_{m\times n}(\mathbb{\MakeUppercase{q}} )}\begin{pmatrix}
			g_1    \\
			\vdots \\
			g_{n}  \\
		\end{pmatrix},
	\]
	where \(k_{i} ^j\in \mathbb{\MakeUppercase{z}} \). From linear algebra, we know that there exists \((r_1, \ldots, r _{m})\in \mathbb{\MakeUppercase{q}} ^m\setminus \{\vec{0} \}\) such that
	\[
		(r_1, \ldots , r_{m}) K = (0, \ldots , 0 ).
	\]
	Multiplying both sides with the common multiple of the denominator of \(r_{i}\), we see that there exists \((\ell _1, \ldots \ell _{m})\in \mathbb{\MakeUppercase{z}} ^m\setminus \{\vec{0} \}\) such that
	\[
		\begin{split}
			&(\ell _1, \ldots \ell _{m})K = (0, \ldots , 0 )\\
			\implies &(\ell _1, \ldots , \ell _{m}  )\begin{pmatrix}
				h_{\alpha _1} \\
				\vdots        \\
				h_{\alpha _m} \\
			\end{pmatrix} = (\ell _1, \ldots , \ell _m )K \begin{pmatrix}
				g_1    \\
				\vdots \\
				g_n    \\
			\end{pmatrix} = (0, \ldots , 0)\\
			\implies &\sum\limits_{i=1}^{m} \ell _{i} h_{\alpha _{i} }= \vec{0} \text{ for \((\ell_1, \ldots, \ell_m)\in \mathbb{Z}^m\setminus \{\vec{0}\}\)} \conta\\
			\implies &\mathrm{card} (I) < \infty.
		\end{split}
	\]
	From the same argument, we see that \(\mathrm{card} (I) \leq n\implies \mathrm{card} (I) = n\).
\end{proof}

\begin{remark}
	Furthermore, one can prove that if \(G\) is a \hyperref[def:free-Abelian-group]{free Abelian group}, then we can prove that any two bases of \(G\) are equinumerous, which
	handle the case that the basis is an infinite set.
\end{remark}
This induces the following definition.

\begin{definition}[Rank]\label{def:rank}
	Let \(G\) ba a \hyperref[def:free-Abelian-group]{free Abelian group}, the \emph{rank} of \(G\) is the cardinality of any basis of \(G\).
\end{definition}

%────────────────────────────────────────────────────────────────────────────────────────────────────────────────────────────────────────────────────
\section{Finitely Generated Abelian Group}\label{apx:ssc:finitely-generated-Abelian-group}
Since we're going to encounter some group as
\[
	\mathbb{\MakeUppercase{z}} \oplus \quotient{\mathbb{\MakeUppercase{z}}}{2\mathbb{\MakeUppercase{z}} },
\]
so it's useful to look into those finitely generated \hyperref[def:Abelian-group]{Abelian group}.

Let's start with a definition.

\begin{definition}[Torsion subgroup]\label{def:torsion-subgroup}
	Given an \hyperref[def:Abelian-group]{Abelian group} \(G\), we say that \(g\in G\) has finite order if \(\exists n\in \mathbb{\MakeUppercase{z}} \) such that
	\(n\cdot g = 0\). Specifically, we say that
	\[
		T\coloneqq \left\{g\in G\mid g\text{ has finite order}\right\}
	\]
	is a \emph{torsion subgroup}.

	If \(T= 0\) given \(G\), we say that \(G\) is \emph{torsion free}.\label{def:torsion-free}
\end{definition}
\begin{note}
	Note that \(T\) is indeed a subgroup, since for any \(g_1, g_2\in T\), \(g_1 + g_2\in T\) from the fact that it still has finite order.
\end{note}

\begin{remark}
	If \(G\) is a \hyperref[def:free-Abelian-group]{free Abelian group}, then \(G\) is \hyperref[def:torsion-free]{torsion free}. Conversely, if \(G\) is
	\hyperref[def:torsion-free]{torsion free}, we can't deduce \(G\) is a \hyperref[def:free-Abelian-group]{free Abelian group}. We see this from
	\((\mathbb{\MakeUppercase{q}} , +)\). Firstly, we see that \(\mathbb{\MakeUppercase{q}} \) is \hyperref[def:torsion-free]{torsion free},
	Now, suppose \(\mathbb{\MakeUppercase{q}} \) is a \hyperref[def:free-Abelian-group]{free Abelian group}, then there exists a basis \(\{r_\alpha \}_{\alpha \in I}\)
	of \(\mathbb{\MakeUppercase{q}} \) such that \(\left\vert I \right\vert > 1\). Now, consider \(\alpha _1, \alpha _2\in I\) such that \(\alpha _1, \alpha _2\in I\), for
	\(r_{\alpha _1}, r_{\alpha _2}\), there exists \(n, m\in \mathbb{\MakeUppercase{z}} \) and \(n, m\neq 0\) such that
	\[
		n r_{\alpha _1} + m r_{\alpha _2} = 0 \implies n = m = 0\conta
	\]
\end{remark}

%────────────────────────────────────────────────────────────────────────────────────────────────────────────────────────────────────────────────────
\subsection{Classification of Finitely generated Abelian Group}
Given a finitely generated \hyperref[def:Abelian-group]{Abelian group} \(G\), we may assume its generators are \(g_1, \ldots , g_n \). Let \(F\) be
\[
	F\coloneqq \underbrace{\mathbb{\MakeUppercase{z}} \oplus \ldots \oplus \mathbb{\MakeUppercase{z}}  }_{n \text{ times}},
\]
then there are a natural surjective homomorphism
\[
	\varphi \colon F\to G,\quad e_{i} \mapsto g_{i}
\]
where \(e_{i} = (0, \ldots , 0, \underset{i^{th} }{1}, 0, \ldots , 0  )\). Now, let \(K\coloneqq \ker \varphi  \), we have
\[
	G \cong \quotient{F}{K}.
\]
Then we have the following lemma.
\begin{lemma}
	\(K\) is a finitely generated \hyperref[def:Abelian-group]{Abelian group}.
\end{lemma}
\begin{proof}
	\[
		\begin{split}
			&\mathbb{\MakeUppercase{z}} \text{ is Noetherian}, F \text{ is a finitely generated \(\mathbb{Z}\)-module }  \\
			\implies &F \text{ is Noetherian module}\\
			\implies &\text{\(K\) as a submodule of \(F\) is a finitely generated \(\mathbb{Z}\)-module } \\
			\implies &\text{\(K\) is a finitely generated \hyperref[def:Abelian-group]{Abelian group}}.
		\end{split}
	\]
	Please refer all the concepts above from \cite{atiyah1994introduction}.
\end{proof}

Hence, we may assume the generators of \(K\) as \(b_1, \ldots , b_{m}\). From the definition of \(K\), we can further express \(b_{i}\) as
\[
	b_{i} = (b_{i1}, b_{i2}, \ldots , b_{in})\begin{pmatrix}
		e_{1}  \\
		\vdots \\
		e_{n}  \\
	\end{pmatrix}_{n \times n}
\]
for all \(i = 1, \ldots , m \). Denote all such row vectors \(b_{i} \) in a matrix \(B\), namely
\[
	B \coloneqq \begin{pmatrix}
		b_{11} & \ldots & b_{1n} \\
		\vdots & \ddots & \vdots \\
		b_{m1} & \ldots & b_{mn} \\
	\end{pmatrix}\in M_{m\times n} ({\mathbb{\MakeUppercase{z}} }),
\]
then we have
\[
	\begin{pmatrix}
		b_1    \\
		\vdots \\
		b_m    \\
	\end{pmatrix} = B\begin{pmatrix}
		e_{1}  \\
		\vdots \\
		e_{n}  \\
	\end{pmatrix}.
\]

\paragraph{Multiply a matrix on the right-hand side.} Now, consider a \(p\in \mathrm{GL}(n; \mathbb{\MakeUppercase{z}} ) \), then
\[
	\begin{pmatrix}
		b_1    \\
		\vdots \\
		b_{m}  \\
	\end{pmatrix} = B \cdot P \underbrace{P^{-1} \begin{pmatrix}
			e_{1}  \\
			\vdots \\
			e_{n}  \\
		\end{pmatrix}}_{\text{new basis} } = (BP)\cdot \begin{pmatrix}
		e_{1}^\prime \\
		\vdots       \\
		e_{n}^\prime \\
	\end{pmatrix},
\]
where
\[
	P^{-1} \begin{pmatrix}
		e_{1}  \\
		\vdots \\
		e_{n}  \\
	\end{pmatrix} \eqqcolon \begin{pmatrix}
		e_{1}^\prime \\
		\vdots       \\
		e_{n}^\prime \\
	\end{pmatrix}.
\]
We see that \(B\cdot P\) is the coefficient matrix of generators \(b_{1}, \ldots , b_{m}   \) under the new basis \(e_1^\prime , \ldots , e_{n} ^\prime  \).

\paragraph{Multiply a matrix on the left-hand side.} For a \(A\in \mathrm{GL}(m; \mathbb{\MakeUppercase{z}} ) \), then
\[
	\begin{pmatrix}
		b_1^\prime \\
		\vdots     \\
		b_m^\prime \\
	\end{pmatrix} = Q \begin{pmatrix}
		b_1    \\
		\vdots \\
		b_m    \\
	\end{pmatrix} = QB\begin{pmatrix}
		e_{1}  \\
		\vdots \\
		e_{n}  \\
	\end{pmatrix},
\]
since \(Q\) is invertible, hence \(b_1^\prime , \ldots , b_{m} ^\prime\) are also generators of \(K\). We see that \(QB\) is the coefficient matrix of new generators
\(b_1^\prime , \ldots , b_{m} ^\prime  \) under basis \(e_1, \ldots , e_{n}  \).

\paragraph{Generally} \(Q\cdot B\cdot P\) is the matrix representation of a particular set of \(F\)'s generators under a particular basis.

\begin{proposition}\label{prop:appendix-1}
	There exists \(P\in \mathrm{GL}(n; \mathbb{\MakeUppercase{z}} ) \) and \(Q\in \mathrm{GL}(m; \mathbb{\MakeUppercase{z}} ) \) such that
	\[
		Q\cdot B\cdot P = \begin{pmatrix}
			d_1 &        &       &   &        \\
			    & \ddots &       &   &        \\
			    &        & d_{k} &   &        \\
			    &        &       & 0 &        \\
			    &        &       &   & \ddots \\
		\end{pmatrix},
	\]
	where \(d_{i} \in \mathbb{\MakeUppercase{z}} ^+\) and \(d_1 \mid d_2 \mid \ldots \mid d_{k}  \).
\end{proposition}
\begin{proof}
	In fact, \(P, Q\) can be taken as the multiplication of the following three types of square matrices:
	\begin{itemize}
		\item \(P_{ij} \):
		      \[
			      P_{ij} = \begin{pmatrix}
				      1 &        &           &        &           &        &   \\
				        & \ddots &           &        &           &        &   \\
				        &        & 0         &        & 1_{(i j)} &        &   \\
				        &        &           & \ddots &           &        &   \\
				        &        & 1_{(j i)} &        & 0         &        &   \\
				        &        &           &        &           & \ddots &   \\
				        &        &           &        &           &        & 1 \\
			      \end{pmatrix},
		      \]
		      where the effect of multiplying \(P_{ij} \) from the right is \emph{swapping column \(i, j\)}.
		\item \(P_{i}(c) \), where \(c\) is the identity in \(\mathbb{\MakeUppercase{z}} \), i.e., \(c = \pm 1\):
		      \[
			      P_{i}(c) = \begin{pmatrix}
				      1 &        &          &        &   \\
				        & \ddots &          &        &   \\
				        &        & c_{(ii)} &        &   \\
				        &        &          & \ddots &   \\
				        &        &          &        & 1 \\
			      \end{pmatrix},
		      \]
		      where the effect of multiplying \(P_{i}(c)\) from the right is \emph{multiplying \(c\) to column \(i\)}.
		\item \(P_{ij}(a) \), \(a\in \mathbb{\MakeUppercase{z}} \):
		      \[
			      P_{ij} = \begin{pmatrix}
				      1 &        &           &   \\
				        & \ddots & a_{(i j)} &   \\
				        &        & \ddots    &   \\
				        &        &           & 1 \\
			      \end{pmatrix},
		      \]
		      where the effect of multiplying \(P_{ij}(a) \) from the right is \emph{adding \(a\) times column \(i\) to column \(j\)}.
	\end{itemize}
	We see that these are \emph{elementary column transformations} in linear algebra. In particular, if we multiply these matrices from the left, then it's called \emph{elementary row transformations}.

	That is to say, we're going to show
	\[
		B = \begin{pmatrix}
			b_{11} & \ldots & b_{1n} \\
			\vdots & \ddots & \vdots \\
			b_{m1} & \ldots & b_{mn} \\
		\end{pmatrix}
	\]
	can become
	\[
		\begin{pmatrix}
			d_1 &        &       &   &        \\
			    & \ddots &       &   &        \\
			    &        & d_{k} &   &        \\
			    &        &       & 0 &        \\
			    &        &       &   & \ddots \\
		\end{pmatrix},
	\]
	\(d_{i} \in \mathbb{\MakeUppercase{z}} ^+\), \(d_1\mid d_2 \mid \ldots \mid d_k \) from \emph{elementary column/row transformations}.

	We now show the steps to make this happens.
	\begin{itemize}
		\item \textbf{Step 1.} Using elementary transformations, we make \(b_{11} >0 \).
		\item \textbf{Step 2.} Using elementary transformations, we make \(b_{11}\) become a divisor of all elements in the first column and row.

		      \par We see that if \(b_{11}\nmid b_{1i}\) for \(i \neq 1\), we have \(b_{1i} = r\cdot b_{11} + s\) where \(0 < s < b_{11}\). Then we add \((-r)\) times the
		      \(1^{th} \) column to the \(i^{th} \) column and swapping the \(1^{th} \) and the \(i^{th} \) column, which makes B becomes
		      \[
			      \begin{pmatrix}
				      s      & \ldots \\
				      \vdots & \ddots \\
			      \end{pmatrix},
		      \]
		      for \(0 < s< b_{11} \). Since \(\mathrm{card}(\left\{n\in \mathbb{\MakeUppercase{z}} \mid 0 < n <b_{11} \right\}) < \infty \), hence in finitely many steps we can make
		      \(B\) becomes
		      \[
			      \begin{pmatrix}
				      d_1    & \ldots \\
				      \vdots & \ddots \\
			      \end{pmatrix},
		      \]
		      where \(d_1\) is a divisor of all other elements in the first column and row.
		\item \textbf{Step 3.} Using elementary transformations, we can multiply the first row by a proper integer and add it to the other rows, do the same but for columns also, then we can
		      make \(B\) becomes
		      \[
			      \begin{pmatrix}
				      d_1    & 0 & \ldots & 0 \\
				      0      &   &        &   \\
				      \vdots &   & B_1    &   \\
				      0      &   &        &   \\
			      \end{pmatrix}.
		      \]
		\item \textbf{Step 4.} We iteratively apply Step 1. to step 3., we make \(B\) into
		      \[
			      \begin{pmatrix}
				      d_1 &        &       &   &        \\
				          & \ddots &       &   &        \\
				          &        & d_{k} &   &        \\
				          &        &       & 0 &        \\
				          &        &       &   & \ddots \\
			      \end{pmatrix},
		      \]
		      where \(d_{i} \in \mathbb{\MakeUppercase{z}} ^+\).
		\item \textbf{Step 5.} Using elementary transformations, by swapping columns and rows, we may assume \(d_1 \leq d_2 \leq \ldots \leq  d_{k}  \).
		\item \textbf{Step 6.} Using elementary transformations, we can make \(B\) into
		      \[
			      \begin{pmatrix}
				      d_1^\prime &        &                  &   &        \\
				                 & \ddots &                  &   &        \\
				                 &        & d_{\ell }^\prime &   &        \\
				                 &        &                  & 0 &        \\
				                 &        &                  &   & \ddots \\
			      \end{pmatrix}
		      \]
		      such that \(0< d_1^\prime \leq \ldots \leq d_{\ell }^\prime\), \(d_1^\prime \mid d_2^\prime \mid \ldots \mid d_{\ell }^\prime\) since if \(d_1\nmid d_{i} \) for some \(i\in \{2, \ldots , k \}\), then
		      \[
			      \begin{pmatrix}
				      d_1 &        &       &   &        \\
				          & \ddots &       &   &        \\
				          &        & d_{k} &   &        \\
				          &        &       & 0 &        \\
				          &        &       &   & \ddots \\
			      \end{pmatrix}\to \begin{pmatrix}
				      d_1 & d_{i}  &       &   &        \\
				          & \ddots &       &   &        \\
				          &        & d_{k} &   &        \\
				          &        &       & 0 &        \\
				          &        &       &   & \ddots \\
			      \end{pmatrix},
		      \]
		      then from Step 2., we have
		      \[
			      \begin{pmatrix}
				      s      & \ldots & \\
				      \vdots & \ddots   \\
			      \end{pmatrix}
		      \]
		      where \(0 < s < d_1\) and \(s\) is a divisor of all other elements in the first row and column. Now, we repeat Step 3. to Step 5., we obtain
		      \[
			      \begin{pmatrix}
				      \widetilde{d} _1 &        &                    &   &        \\
				                       & \ddots &                    &   &        \\
				                       &        & \widetilde{d} _{j} &   &        \\
				                       &        &                    & 0 &        \\
				                       &        &                    &   & \ddots \\
			      \end{pmatrix}
		      \]
		      where \(\widetilde{d} _1 \leq \ldots \leq \widetilde{d} _{j}  \) such that \(\widetilde{d} _1 < d_1\). Since there are only finitely many integers which is smaller than \(d_1\), we
		      see that by repeating these steps, we can always make
		      \[
			      \begin{pmatrix}
				      d_1 &        &       &   &        \\
				          & \ddots &       &   &        \\
				          &        & d_{k} &   &        \\
				          &        &       & 0 &        \\
				          &        &       &   & \ddots \\
			      \end{pmatrix}
		      \]
		      into
		      \[
			      \begin{pmatrix}
				      \widetilde{d} _1 &        &                    &   &        \\
				                       & \ddots &                    &   &        \\
				                       &        & \widetilde{d} _{p} &   &        \\
				                       &        &                    & 0 &        \\
				                       &        &                    &   & \ddots \\
			      \end{pmatrix}
		      \]
		      such that \(d_1^\prime \mid d_i^\prime \) for all \(i \neq  1\) and \(d_1^\prime \leq d_2^\prime \leq \ldots \leq d_p^\prime  \). By the same idea of Step 3., we have the desired matrix.
	\end{itemize}
	Since all operations are elementary and there are only finitely many of them, hence the result follows.
\end{proof}

From the definition of \(Q\cdot B\cdot P\) and \autoref{prop:appendix-1}, there exists a basis \(e_1^\prime, \ldots , e_{n} ^\prime  \) of \(F\) such
that \(K\) has finitely many generators \(d_1 e_1^\prime , \ldots , d_{k} e_{k} ^\prime\), hence
\[
	G\cong \quotient{\mathbb{\MakeUppercase{z}} }{d_1 \mathbb{\MakeUppercase{z}} } \oplus \quotient{\mathbb{\MakeUppercase{z}} }{d_2 \mathbb{\MakeUppercase{z}} } \oplus \ldots \oplus \quotient{\mathbb{\MakeUppercase{z}} }{d_k \mathbb{\MakeUppercase{z}} }  \oplus \underbrace{\mathbb{\MakeUppercase{z}} \oplus \ldots \oplus \mathbb{\MakeUppercase{z}}  }_{n-k \text{ times}}.
\]

This leads to the following important theorem.

\begin{theorem}[Fundamental theorem of finitely generated Abelian group]\label{thm:fundamental-theorem-of-finitely-generalted-Abelian-group}
	Given a finitely generated \hyperref[def:Abelian-group]{Abelian group}, either \(G\) is a \hyperref[def:free-Abelian-group]{free Abelian group}, or there exists
	a unique set of \(\{m_{i}\in \mathbb{\MakeUppercase{z}} \mid  m_{i} > 1, i = 1, \ldots , t \}\) such that \(m_1 \mid m_2 \mid \ldots \mid m_t \) and a unique non-negative integer \(s\)
	such that
	\[
		G\cong \quotient{\mathbb{\MakeUppercase{z}} }{m_1 \mathbb{\MakeUppercase{z}} } \oplus \quotient{\mathbb{\MakeUppercase{z}} }{m_2 \mathbb{\MakeUppercase{z}} } \oplus \ldots \oplus \quotient{\mathbb{\MakeUppercase{z}} }{m_t \mathbb{\MakeUppercase{z}} }  \oplus \underbrace{\mathbb{\MakeUppercase{z}} \oplus \ldots \oplus \mathbb{\MakeUppercase{z}}  }_{s \text{ times}}.
	\]
\end{theorem}
\begin{proof}
	We need to show both uniqueness and existence.
	\paragraph{Existence.} From \autoref{prop:appendix-1}, we obtain a basis \(e_1^\prime , \ldots , e_{n} ^\prime\) of \(F\) and a basis \(d_1 e_1^\prime , \ldots , d_{k} e_{k} ^\prime\) in \(K\)
	such that \(d_1\mid \ldots \mid d_k \). Let
	\[
		(d_1, \ldots , d_{k}) = (1, \ldots , 1, m_1, \ldots , m_{t}),
	\]
	which implies
	\[
		\begin{split}
			G&\cong \quotient{F}{K} \\
			&\cong \quotient{\mathbb{\MakeUppercase{z}} }{d_1 \mathbb{\MakeUppercase{z}} } \oplus \quotient{\mathbb{\MakeUppercase{z}} }{d_2 \mathbb{\MakeUppercase{z}} } \oplus \ldots \oplus \quotient{\mathbb{\MakeUppercase{z}} }{d_k \mathbb{\MakeUppercase{z}} }  \oplus \mathbb{\MakeUppercase{z}} \oplus \ldots \oplus \mathbb{\MakeUppercase{z}}\\
			&= \underline{\quotient{\mathbb{\MakeUppercase{z}} }{1 \mathbb{\MakeUppercase{z}} } \oplus \ldots \oplus \quotient{\mathbb{\MakeUppercase{z}} }{1 \mathbb{\MakeUppercase{z}} }} \oplus\quotient{\mathbb{\MakeUppercase{z}} }{m_1 \mathbb{\MakeUppercase{z}} } \oplus \ldots \oplus \quotient{\mathbb{\MakeUppercase{z}} }{m_{t}  \mathbb{\MakeUppercase{z}} }  \oplus \mathbb{\MakeUppercase{z}} \oplus \ldots \oplus \mathbb{\MakeUppercase{z}}\\
			&= \quotient{\mathbb{\MakeUppercase{z}} }{m_1 \mathbb{\MakeUppercase{z}} } \oplus \ldots \oplus \quotient{\mathbb{\MakeUppercase{z}} }{m_t \mathbb{\MakeUppercase{z}} }  \oplus \underbrace{\mathbb{\MakeUppercase{z}} \oplus \ldots \oplus \mathbb{\MakeUppercase{z}}  }_{\exists ! s \text{ times}}.
		\end{split}
	\]

	\paragraph{Uniqueness.} Under the isomorphism \(\quotient{\mathbb{\MakeUppercase{z}} }{m_1 \mathbb{\MakeUppercase{z}} } \oplus \ldots \oplus \quotient{\mathbb{\MakeUppercase{z}} }{m_t \mathbb{\MakeUppercase{z}} }  \oplus \underbrace{\mathbb{\MakeUppercase{z}} \oplus \ldots \oplus \mathbb{\MakeUppercase{z}}  }_{s \text{ times}}\),
	we see that
	\[
		\quotient{\mathbb{\MakeUppercase{z}} }{m_1 \mathbb{\MakeUppercase{z}} } \oplus \ldots \oplus \quotient{\mathbb{\MakeUppercase{z}} }{m_t \mathbb{\MakeUppercase{z}} }
	\]
	corresponds to \(G\)'s \hyperref[def:torsion-subgroup]{torsion subgroup} \(T\), which implies
	\[
		\quotient{G}{T} \cong \underbrace{\mathbb{\MakeUppercase{z}} \oplus \ldots \oplus \mathbb{\MakeUppercase{z}}  }_{\text{\(s\) times} },
	\]
	which further implies \(\quotient{G}{T} \) is a \hyperref[def:free-Abelian-group]{free Abelian group} with
	\[
		\mathrm{rk} \left(\quotient{G}{T} \right) = s,
	\]
	which proves the uniqueness of \(s\).

	The proof of the uniqueness of \(m_{i} \) are long and tedious, we refer to \cite{armstrong2013basic}.
\end{proof}

\begin{definition}[Invariant factor]\label{def:invariant-factor}
	We call \(m_1, \ldots , m_t \) obtained from \autoref{thm:fundamental-theorem-of-finitely-generalted-Abelian-group} the \emph{invariant factor}.
\end{definition}

\begin{lemma}\label{lma:Chinese-remainder-theorem}
	Given a positive integer \(m\) such that
	\[
		m = p_1^{n_1}\cdot \ldots \cdot p_s^{n_s}
	\]
	where \(p\in \mathcal{\MakeUppercase{p}} \) are all prime and \(p_{i} \neq p_{j} \) for \(i\neq j\), with \(n_{i} \in \mathbb{\MakeUppercase{z}} ^+\) for all \(i\).
	Then
	\[
		\quotient{\mathbb{\MakeUppercase{z}}}{m\mathbb{\MakeUppercase{z}} } \cong \quotient{\mathbb{\MakeUppercase{z}} }{p_1^{n_1}\mathbb{\MakeUppercase{z}} } \oplus \ldots \oplus \quotient{\mathbb{\MakeUppercase{z}} }{p_s^{n_s}\mathbb{\MakeUppercase{z}} }.
	\]
\end{lemma}
\begin{proof}
	We define \(\phi\) as
	\[
		\begin{split}
			\phi \colon \quotient{\mathbb{\MakeUppercase{z}} }{m\mathbb{\MakeUppercase{z}} } &\to \quotient{\mathbb{\MakeUppercase{z}} }{p_1^{n_1}\mathbb{\MakeUppercase{z}} } \oplus \ldots \oplus \quotient{\mathbb{\MakeUppercase{z}} }{p_s^{n_s}\mathbb{\MakeUppercase{z}} }\\
			\overline{n} &\mapsto (n + \left< p_1^{n_1} \right>, \ldots , n + \left< p_s^{n_s} \right>  ).
		\end{split}
	\]
	Then \(\overline{n} \in \ker  \phi \iff \underset{i}{\forall }\ p_i^{n_{i} }\mid n \iff m\mid n \iff \overline{n} = \overline{0}\). This means \(\ker  \phi = 0\), hence \(\phi \) is an injection.

	We now prove \(\phi \) is a surjection. It's sufficient to prove that for all \(i\),
	\[
		(0, \ldots , 0, 1+ \left< p_i ^{n_{i} } \right>, 0, \ldots , 0)\in \quotient{\mathbb{\MakeUppercase{z}} }{p_1^{n_1}\mathbb{\MakeUppercase{z}} } \oplus \ldots \oplus \quotient{\mathbb{\MakeUppercase{z}} }{p_{s} ^{n_{s} }\mathbb{\MakeUppercase{z}} },
	\]
	there exists an \(\overline{n} \) such that
	\[
		\phi (\overline{n} ) = (0, \ldots , 0, 1+\left< p_{i} ^{n_{i}}\right>, 0, \ldots , 0).
	\]
	Notice that for all \(i\neq j\), \(\left< p_{i} ^{n_{i} } \right> + \left< p_{j} ^{n_{j} } \right> \in \mathbb{\MakeUppercase{z}}  \), hence there exists \(u_{j} \in \left< p_{i} ^{n_{i} } \right> \) and
	\(v_{j} \in \left< p_{j} ^{n_{j} } \right> \) such that \(u_{j} + v_{j} = 1\). Let \(n\) as
	\[
		n = \prod\limits_{i\neq j}(1 - u_{j} ),
	\]
	then
	\[
		n + \left< p_{i} ^{n_{i} } \right> = 1 + \left<  p_{i} ^{n_{i} }\right>,\quad n + \left< p_{j} ^{n_{j} } \right> = 0 + \left< p_{j} ^{n_{j} } \right>.
	\]
	Above implies
	\[
		\phi (\overline{n} ) = (0, \ldots , 0, 1 + \left< p_{i} ^{n_{i} } \right> , 0, \ldots , 0),
	\]
	hence \(\phi \) surjects, so
	\[
		\quotient{\mathbb{\MakeUppercase{z}}}{m\mathbb{\MakeUppercase{z}} } \cong \quotient{\mathbb{\MakeUppercase{z}} }{p_1^{n_1}\mathbb{\MakeUppercase{z}} } \oplus \ldots \oplus \quotient{\mathbb{\MakeUppercase{z}} }{p_s^{n_s}\mathbb{\MakeUppercase{z}} }.
	\]
\end{proof}

Combine \autoref{thm:fundamental-theorem-of-finitely-generalted-Abelian-group} and \autoref{lma:Chinese-remainder-theorem}, we see that we now only have
\[
	G\cong \quotient{\mathbb{\MakeUppercase{z}} }{m_1 \mathbb{\MakeUppercase{z}} } \oplus \quotient{\mathbb{\MakeUppercase{z}} }{m_2 \mathbb{\MakeUppercase{z}} } \oplus \ldots \oplus \quotient{\mathbb{\MakeUppercase{z}} }{m_t \mathbb{\MakeUppercase{z}} }  \oplus \underbrace{\mathbb{\MakeUppercase{z}} \oplus \ldots \oplus \mathbb{\MakeUppercase{z}}  }_{s \text{ times}},
\]
we can further decompose \(G\) into
\[
	G\cong \quotient{\mathbb{\MakeUppercase{z}} }{p_1^{s_1}\mathbb{\MakeUppercase{z}} }\oplus \ldots \oplus \quotient{\mathbb{\MakeUppercase{z}} }{p_k^{s_k}\mathbb{\MakeUppercase{z}} }   \oplus \underbrace{\mathbb{\MakeUppercase{z}} \oplus \ldots \oplus \mathbb{\MakeUppercase{z}}  }_{s \text{ times}},
\]
where \(p_1, \ldots , p_{k}  \) are primes (which may includes repeated terms), \(s_{i} \in \mathbb{\MakeUppercase{z}} ^+\) for all \(i\).

\begin{definition}[Elementary divisors]\label{def:elementary-divisors}
	The set
	\[
		\left\{p_1^{s_1}, \ldots , p_{k}^{s_k} \right\}
	\]
	are called \emph{elementary divisors} of \(G\).
\end{definition}

\begin{theorem}[Uniqueness of elementary divisors]\label{thm:uniqueness-of-elementary-divisors}
	\hyperref[def:elementary-divisors]{Elementary divisors} of a group \(G\) is unique.
\end{theorem}
\begin{proof}
	Please refer to \cite{armstrong2013basic}.
\end{proof}

%────────────────────────────────────────────────────────────────────────────────────────────────────────────────────────────────────────────────────
\section{Homological Algebra}\label{sec:homological-algebra}
\subsection{Exact Sequence}
\begin{prev}
	Given two \hyperref[def:Abelian-group]{Abelian groups} \(A, B\) and the group homomorphism \(\varphi\colon A\to B\), then we have
	\begin{itemize}
		\item \(\ker  \varphi = \left\{x\in A\mid \varphi (x) = 0\right\}\)
		\item \(\im  \varphi = \left\{\varphi (x)\mid x\in A\right\} \)
		\item \(\mathrm{coker}\varphi \coloneqq \quotient{B}{\im  \varphi}\)
		\item \(\mathrm{coIm} \varphi \coloneqq \quotient{A}{\ker  \varphi}\)
	\end{itemize}
\end{prev}

Consider a sequence of \hyperref[def:Abelian-group]{Abelian} group homomorphism
\[
	\begin{tikzcd}
		\ldots & {A_{i-1}} & {A_i} & {A_{i+1}} & \ldots
		\arrow[from=1-1, to=1-2]
		\arrow["{\phi_{i-1}}", from=1-2, to=1-3]
		\arrow["{\phi_i}", from=1-3, to=1-4]
		\arrow[from=1-4, to=1-5]
	\end{tikzcd}
\]
We denote this sequence as \(S\).

\begin{definition}[Exact]\label{def:apx:exact}
	We say \(S\) is \emph{exact} at \(A_{i} \) if
	\[
		\im  \phi _{i-1} = \ker \phi _{i} .
	\]
\end{definition}
\begin{remark}
	\autoref{def:apx:exact} is same as \autoref{def:exact}.
\end{remark}

\begin{definition}[Exact sequence]\label{def:exact-sequence}
	We call \(S\) is an \emph{exact sequence} if it's \hyperref[def:apx:exact]{exact} at \(A_{i} \) for all \(i\).
\end{definition}

\begin{remark}
	Specifically, consider the following two situations.
	\begin{itemize}
		\item We say
		      \[
			      \begin{tikzcd}
				      {A_0} & {A_{1}} & {A_2} & \ldots
				      \arrow[from=1-1, to=1-2]
				      \arrow[from=1-2, to=1-3]
				      \arrow[from=1-3, to=1-4]
			      \end{tikzcd}
		      \]
		      is an \hyperref[def:exact-sequence]{exact sequence} if it's \hyperref[def:apx:exact]{exact} at \(A_{i} \) for all \(i\geq 1\).
		\item We say
		      \[
			      \begin{tikzcd}
				      \ldots & {A_{-2}} & {A_{-1}} & {A_0}
				      \arrow[from=1-3, to=1-4]
				      \arrow[from=1-1, to=1-2]
				      \arrow[from=1-2, to=1-3]
			      \end{tikzcd}
		      \]
		      is an \hyperref[def:exact-sequence]{exact sequence} if it's \hyperref[def:apx:exact]{exact} at \(A_{i} \) for all \(i\leq -1\).
	\end{itemize}
\end{remark}

\begin{remark}
	Denote \(\circ \) as a trivial \hyperref[def:Abelian-group]{Abelian group}, then
	\[
		\begin{tikzcd}
			A & B & \circ
			\arrow["\phi", from=1-1, to=1-2]
			\arrow[from=1-2, to=1-3]
		\end{tikzcd}\text{ is an \hyperref[def:exact-sequence]{exact sequence}} \iff \phi \text{ is a surjective homomorphism};
	\]
	conversely,
	\[
		\begin{tikzcd}
			\circ & B & A
			\arrow[from=1-1, to=1-2]
			\arrow["\phi",from=1-2, to=1-3]
		\end{tikzcd}\text{ is an \hyperref[def:exact-sequence]{exact sequence}} \iff \phi \text{ is an injective homomorphism}.
	\]
\end{remark}

\begin{definition}[Short exact sequence]\label{def:short-exact-sequence}
	A \emph{short exact sequence} is an \hyperref[def:exact-sequence]{exact sequence} such that it has the following form
	\[
		\begin{tikzcd}
			\circ & A & B & C & \circ
			\arrow["\phi", from=1-2, to=1-3]
			\arrow["\psi", from=1-3, to=1-4]
			\arrow[from=1-1, to=1-2]
			\arrow[from=1-4, to=1-5]
		\end{tikzcd}.
	\]
\end{definition}

\begin{remark}
	Let \(B \overset{\psi}{\longrightarrow} C\) as a surjective homomorphism and \(K = \ker  \psi \), and we denote \(K \overset{i}{\longrightarrow}B \) as an injection. Then
	\[
		\begin{tikzcd}
			\circ & K & B & C & \circ
			\arrow["i", from=1-2, to=1-3]
			\arrow["\psi", from=1-3, to=1-4]
			\arrow[from=1-1, to=1-2]
			\arrow[from=1-4, to=1-5]
		\end{tikzcd}
	\]
	is a \hyperref[def:short-exact-sequence]{short exact sequence}. Conversely, if
	\[
		\begin{tikzcd}
			\circ & A & B & C & \circ
			\arrow["\phi ", from=1-2, to=1-3]
			\arrow["\psi", from=1-3, to=1-4]
			\arrow[from=1-1, to=1-2]
			\arrow[from=1-4, to=1-5]
		\end{tikzcd}
	\]
	is a \hyperref[def:short-exact-sequence]{short exact sequence}, then \(\phi \) is an injective homomorphism since it is \hyperref[def:apx:exact]{exact} at \(A\),
	and \(\psi \) is a surjective homomorphism since it is \hyperref[def:apx:exact]{exact} at \(C\), and \(\phi (A) = \ker \psi\) since it is \hyperref[def:apx:exact]{exact} at \(B\).
	This implies \(\phi \colon A\to \phi (A) = \ker  \psi \) is a group homeomorphism.
\end{remark}
\begin{eg}
	Given \(A, B\) as \hyperref[def:Abelian-group]{Abelian groups}, then
	\[
		\begin{tikzcd}[row sep=tiny]
			\circ & A & {A\oplus B} & B & \circ \\
			& a & {(a, 0)} \\
			&& {(a, b)} & b
			\arrow["i", from=1-2, to=1-3]
			\arrow["{\mathrm{Proj}_2}", from=1-3, to=1-4]
			\arrow[from=1-1, to=1-2]
			\arrow[from=1-4, to=1-5]
			\arrow["i", maps to, from=2-2, to=2-3]
			\arrow["{\mathrm{Proj}_2}", maps to, from=3-3, to=3-4]
		\end{tikzcd}
	\]
	is a \hyperref[def:short-exact-sequence]{short exact sequence}.
\end{eg}
\begin{eg}
	We see that
	\[
		\begin{tikzcd}[row sep=tiny]
			\circ & \mathbb{\MakeUppercase{z}} & {\mathbb{\MakeUppercase{z}} } & \quotient{\mathbb{\MakeUppercase{z}} }{n\mathbb{\MakeUppercase{z}} }  & \circ \\
			& k & {k\cdot n} \\
			\arrow["i", from=1-2, to=1-3]
			\arrow["{\mathrm{Proj}_2}", from=1-3, to=1-4]
			\arrow[from=1-1, to=1-2]
			\arrow[from=1-4, to=1-5]
			\arrow[maps to, from=2-2, to=2-3]
		\end{tikzcd}
	\]
	for \(n\in \mathbb{\MakeUppercase{z}} _{\geq 1}\) is a \hyperref[def:short-exact-sequence]{short exact sequence}.
\end{eg}

\begin{definition}[Isomorphism between sequences]\label{def:isomorphism-between-sequences}
	Given \(A_\cdot\) and \(B_\cdot\) defined as two sequences of \hyperref[def:Abelian-group]{Abelian group} homomorphisms
	\[A_\bullet : \begin{tikzcd}
			\ldots & {A_i} & {A_{i+1}} & \ldots & {}
			\arrow["{\phi_i}", from=1-2, to=1-3]
			\arrow[from=1-3, to=1-4]
			\arrow[from=1-1, to=1-2]
		\end{tikzcd}
	\]
	and
	\[B_\bullet : \begin{tikzcd}
			\ldots & {B_i} & {B_{i+1}} & \ldots & {}
			\arrow["{\psi_i}", from=1-2, to=1-3]
			\arrow[from=1-3, to=1-4]
			\arrow[from=1-1, to=1-2]
		\end{tikzcd}
	\]

	And we say a morphism \(\alpha \) from \(A_\bullet\) to \(B_\bullet\) is a series of group homomorphisms \(\alpha _i\colon A_{i} \to B_{i} \) for all \(i\in \mathbb{\MakeUppercase{z}} \) such that
	the following diagram commutes.
	\[
		\begin{tikzcd}
			\ldots & {A_i} & {A_{i+1}} & \ldots \\
			\ldots & {B_i} & {B_{i+1}} & \ldots
			\arrow["{\phi_i}", from=1-2, to=1-3]
			\arrow[from=1-3, to=1-4]
			\arrow[from=1-1, to=1-2]
			\arrow[from=2-3, to=2-4]
			\arrow["{\psi_i}", from=2-2, to=2-3]
			\arrow[from=2-1, to=2-2]
			\arrow["{\alpha_i}", from=1-2, to=2-2]
			\arrow["{\alpha_{i+1}}", from=1-3, to=2-3]
		\end{tikzcd}
	\]

	Additionally, if for all \(i\), \(\alpha _i\) is a group homeomorphism, then we say \(\alpha \colon A_\bullet\to B_\bullet\) is a homeomorphism.
\end{definition}

\begin{definition}[Split short exact sequence]\label{def:split-short-exact-sequence}
	Given a \hyperref[def:short-exact-sequence]{short exact sequence}
	\[
		\begin{tikzcd}
			\circ & A & B & C & \circ
			\arrow["\phi ", from=1-2, to=1-3]
			\arrow["\psi", from=1-3, to=1-4]
			\arrow[from=1-1, to=1-2]
			\arrow[from=1-4, to=1-5]
		\end{tikzcd}
	\]
	we say it is \emph{split} if there exists a group homeomorphism \(\theta \colon B\to A\oplus C\) such that
	\[
		\begin{tikzcd}
			\circ & A & B & C & \circ\\
			\circ & A & {A\oplus C} & C & \circ
			\arrow["\phi", from=1-2, to=1-3]
			\arrow[from=1-1, to=1-2]
			\arrow[from=2-2, to=2-3]
			\arrow[from=2-1, to=2-2]
			\arrow["\identity", from=1-2, to=2-2]
			\arrow["\theta", from=1-3, to=2-3]
			\arrow[from=2-3, to=2-4]
			\arrow[from=2-4, to=2-5]
			\arrow[from=1-4, to=1-5]
			\arrow["\psi", from=1-3, to=1-4]
			\arrow["\identity", from=1-4, to=2-4]
		\end{tikzcd}
	\]
	is the \hyperref[def:isomorphism-between-sequences]{isomorphism} between these two \hyperref[def:short-exact-sequence]{short exact sequences}.
\end{definition}

\begin{remark}
	Given \hyperref[def:split-short-exact-sequence]{split short exact sequence}
	\[
		\begin{tikzcd}
			\circ & A & B & C & \circ
			\arrow["\phi ", from=1-2, to=1-3]
			\arrow["\psi", from=1-3, to=1-4]
			\arrow[from=1-1, to=1-2]
			\arrow[from=1-4, to=1-5]
		\end{tikzcd}
	\]
	and \(\theta \) defined in \autoref{def:split-short-exact-sequence}, let \(i\colon A\to A\oplus C\), \(a\mapsto (a, 0)\) and \(j\colon C\to A\oplus C\), \(c\mapsto (0, c)\) are two
	canonical embeddings, then we have
	\[
		A\hyperref[def:external-direct-sum]{\oplus }C = i(A)\hyperref[def:internal-direct-sum]{\oplus} j(C).
	\]
	Consider \(\theta ^{-1} \colon A\oplus C \overset{\cong}{\longrightarrow}B \), then
	\[
		B= \theta ^{-1} (i(A)) \hyperref[def:internal-direct-sum]{\oplus} \theta ^{-1} (j(C)).
	\]
	Since the diagram in \autoref{def:split-short-exact-sequence} commutes, hence
	\[
		\theta ^{-1} (i(A)) = \theta ^{-1} \circ i(A) = \phi (A),
	\]
	hence
	\[
		B = \phi (A) \hyperref[def:internal-direct-sum]{\oplus} \underbrace{\theta ^{-1} (j(C))}_{D},
	\]
	which implies \(\at{\psi }{D}{}\colon D\to C \) is a group homeomorphism. We see that
	\[
		\begin{tikzcd}
			\circ & A & B & C & \circ
			\arrow["\phi ", from=1-2, to=1-3]
			\arrow["\psi", from=1-3, to=1-4]
			\arrow[from=1-1, to=1-2]
			\arrow[from=1-4, to=1-5]
		\end{tikzcd}
	\]
	\hyperref[def:split-short-exact-sequence]{split} implies \(B = \phi (A)\hyperref[def:internal-direct-sum]{\oplus} D\) and \(\at{\psi }{D}{}\colon D \overset{\cong}{\longrightarrow} C \).

	Conversely, if \(B = \phi (A)\hyperref[def:internal-direct-sum]{\oplus} D\) and \(\at{\psi }{D}{}\colon D \overset{\cong}{\longrightarrow} C \), then there exists a \(\theta \)
	\[
		\begin{split}
			\theta \colon B&\to A\oplus C\\
			\phi (a) + d&\mapsto (a, \psi (d))
		\end{split}
	\]
	for \(a\in A, d\in D\) such that
	\[
		\begin{tikzcd}
			\circ & A & B & C & \circ & {} \\
			\circ & A & {A\oplus C} & C & \circ \\
			&& {\phi(a)+d} & {\psi(d)} \\
			&& {(a, \psi(d))} & {\psi(d)}
			\arrow["\phi", from=1-2, to=1-3]
			\arrow[from=1-1, to=1-2]
			\arrow[from=2-2, to=2-3]
			\arrow[from=2-1, to=2-2]
			\arrow["\identity", from=1-2, to=2-2]
			\arrow["\theta", from=1-3, to=2-3]
			\arrow[from=2-3, to=2-4]
			\arrow[from=2-4, to=2-5]
			\arrow[from=1-4, to=1-5]
			\arrow["\psi", from=1-3, to=1-4]
			\arrow["\identity", from=1-4, to=2-4]
			\arrow[maps to, from=3-3, to=3-4]
			\arrow[maps to, from=3-4, to=4-4]
			\arrow[maps to, from=3-3, to=4-3]
			\arrow[maps to, from=4-3, to=4-4]
		\end{tikzcd}
	\]
	commutes.

	Hence, for a \hyperref[def:short-exact-sequence]{short exact sequence} \(\begin{tikzcd}
		\circ & A & B & C & \circ
		\arrow["\phi ", from=1-2, to=1-3]
		\arrow["\psi", from=1-3, to=1-4]
		\arrow[from=1-1, to=1-2]
		\arrow[from=1-4, to=1-5]
	\end{tikzcd}\) is \hyperref[def:split-short-exact-sequence]{split} if and only if \(B = \phi (A)\hyperref[def:internal-direct-sum]{\oplus} D\) and \(\at{\psi }{D}{}\colon D \overset{\cong}{\longrightarrow} C \).

	Remarkably, let \(\begin{tikzcd}
		\circ & A & B & C & \circ
		\arrow["\phi ", from=1-2, to=1-3]
		\arrow["\psi", from=1-3, to=1-4]
		\arrow[from=1-1, to=1-2]
		\arrow[from=1-4, to=1-5]
	\end{tikzcd}\) is a \hyperref[def:split-short-exact-sequence]{split short exact sequence}, then \(D\) constructed above is not unique. To see this, consider
	\[
		\begin{tikzcd}[row sep=tiny]
			\circ & \mathbb{\MakeUppercase{z}} & {\mathbb{\MakeUppercase{z}}\oplus \mathbb{\MakeUppercase{z}}  } & \mathbb{\MakeUppercase{z}} & \circ \\
			& n & {(n, 0)} \\
			& & {(n, m)} & {m} \\
			\arrow["i", from=1-2, to=1-3]
			\arrow["{\mathrm{Proj}_2}", from=1-3, to=1-4]
			\arrow[from=1-1, to=1-2]
			\arrow[from=1-4, to=1-5]
			\arrow[maps to, from=2-2, to=2-3]
			\arrow[maps to, from=3-3, to=3-4]
		\end{tikzcd}
	\]
	We have \(\mathbb{\MakeUppercase{z}} \oplus \mathbb{\MakeUppercase{z}} = i(\mathbb{\MakeUppercase{z}} )\oplus j(\mathbb{\MakeUppercase{z}} )\) where \(j\colon \mathbb{\MakeUppercase{z}} \to \mathbb{\MakeUppercase{z}} \oplus \mathbb{\MakeUppercase{z}} \),
	\(n\mapsto (0,n)\). We see that we can let \(D\coloneqq j(\mathbb{\MakeUppercase{z}} )\). Meanwhile, we can also let
	\[
		D \coloneqq \left\{(n, n)\mid n\in \mathbb{\MakeUppercase{z}} \right\}< \mathbb{\MakeUppercase{z}} \oplus\mathbb{\MakeUppercase{z}}
	\]
	such that \(\mathbb{\MakeUppercase{z}} \oplus\mathbb{\MakeUppercase{z}}  = i(\mathbb{\MakeUppercase{z}} )\oplus D\).
\end{remark}

\begin{eg}[Non-split short exact sequence]
	We see that
	\[
		\begin{tikzcd}[row sep=tiny]
			\circ & \mathbb{\MakeUppercase{z}} & {\mathbb{\MakeUppercase{z}} } & \quotient{\mathbb{\MakeUppercase{z}} }{n\mathbb{\MakeUppercase{z}} }  & \circ \\
			& k & {k\cdot n} \\
			\arrow["i", from=1-2, to=1-3]
			\arrow["{\mathrm{Proj}_2}", from=1-3, to=1-4]
			\arrow[from=1-1, to=1-2]
			\arrow[from=1-4, to=1-5]
			\arrow[maps to, from=2-2, to=2-3]
		\end{tikzcd}
	\]
	is not a \hyperref[def:split-short-exact-sequence]{split short exact sequence}, since if it is, then
	\[
		\begin{split}
			\mathbb{\MakeUppercase{z}} \oplus\quotient{\mathbb{\MakeUppercase{z}} }{n\mathbb{\MakeUppercase{z}} } &\cong\mathbb{\MakeUppercase{z}}\\
			(0, 1)&\mapsto k,
		\end{split}
	\]
	which is a contradiction since \(\mathbb{\MakeUppercase{z}} \) is \hyperref[def:torsion-subgroup]{torsion}-free while \(\mathbb{\MakeUppercase{z}} \oplus \quotient{\mathbb{\MakeUppercase{z}} }{n\mathbb{\MakeUppercase{z}} } \)  is not.
\end{eg}

\begin{lemma}[Splitting lemma]\label{lma:splitting-lemma}
	If \(\begin{tikzcd}
		\circ & A & B & C & \circ
		\arrow["\phi ", from=1-2, to=1-3]
		\arrow["\psi", from=1-3, to=1-4]
		\arrow[from=1-1, to=1-2]
		\arrow[from=1-4, to=1-5]
	\end{tikzcd}\) is a \hyperref[def:short-exact-sequence]{short exact sequence}, then the following are equivalent.
	\begin{enumerate}[(1)]
		\item This \hyperref[def:short-exact-sequence]{short exact sequence} \hyperref[def:split-short-exact-sequence]{split}.
		\item \(\exists p\colon B\to A\) such that \(p\circ \phi = \identity_{A} \).
		\item \(\exists q\colon C\to B\) such that \(\psi \circ q = \identity_{C}\).
	\end{enumerate}
\end{lemma}
\begin{proof}
	\begin{itemize}
		\item \(1. \implies 2.\) Let \(\theta \colon B \overset{\cong}{\longrightarrow}A\oplus C \) such that it's the \hyperref[def:isomorphism-between-sequences]{isomorphism} which makes the following diagram commutes.
		      \[
			      \begin{tikzcd}
				      \circ & A & B & C & \circ & {} \\
				      \circ & A & {A\oplus C} & C & \circ
				      \arrow["\phi", from=1-2, to=1-3]
				      \arrow[from=1-1, to=1-2]
				      \arrow["i", from=2-2, to=2-3]
				      \arrow[from=2-1, to=2-2]
				      \arrow["\identity", from=1-2, to=2-2]
				      \arrow["\theta", from=1-3, to=2-3]
				      \arrow[from=2-3, to=2-4]
				      \arrow[from=2-4, to=2-5]
				      \arrow[from=1-4, to=1-5]
				      \arrow["\psi", from=1-3, to=1-4]
				      \arrow["\identity", from=1-4, to=2-4]
				      \arrow["{\mathrm{Proj}_1}", curve={height=-12pt}, from=2-3, to=2-2]
			      \end{tikzcd}
		      \]
		      Then we let \(p\coloneqq \mathrm{Proj}_1 \circ \theta \), then
		      \[
			      p\circ \phi = \mathrm{Proj} _1 \circ \theta \circ \phi = \mathrm{Proj} _1\circ i = \identity_{A}.
		      \]
		\item \(1. \implies 3.\) Let \(\theta \colon B \overset{\cong}{\longrightarrow}A\oplus C \) such that it's the \hyperref[def:isomorphism-between-sequences]{isomorphism} which makes the following diagram commutes.
		      \[
			      \begin{tikzcd}
				      \circ & A & B & C & \circ & {} \\
				      \circ & A & {A\oplus C} & C & \circ
				      \arrow["\phi", from=1-2, to=1-3]
				      \arrow[from=1-1, to=1-2]
				      \arrow[from=2-2, to=2-3]
				      \arrow[from=2-1, to=2-2]
				      \arrow["\identity", from=1-2, to=2-2]
				      \arrow["\theta", from=1-3, to=2-3]
				      \arrow["{\mathrm{Proj}_2}", from=2-3, to=2-4]
				      \arrow[from=2-4, to=2-5]
				      \arrow[from=1-4, to=1-5]
				      \arrow["\psi", from=1-3, to=1-4]
				      \arrow["\identity", from=1-4, to=2-4]
				      \arrow["j", curve={height=-12pt}, from=2-4, to=2-3]
			      \end{tikzcd}
		      \]
		      Then we let \(q\coloneqq \theta ^{-1} \circ j\), then for all \(c\in C\), we have
		      \[
			      \psi \circ q(c) = \psi \left(\theta ^{-1} (j(c))\right) = \mathrm{Proj} _2 \circ \theta \left(\theta ^{-1} (j(c))\right) = \mathrm{Proj} _2(j(c)) = c,
		      \]
		      hence \(\psi \circ q = \identity_{C} \).
		\item \(2. \implies 1.\) We have
		      \[
			      \begin{tikzcd}
				      \circ & A & B & C & \circ & {}
				      \arrow["\phi", from=1-2, to=1-3]
				      \arrow[from=1-1, to=1-2]
				      \arrow[from=1-4, to=1-5]
				      \arrow["\psi", from=1-3, to=1-4]
				      \arrow["p", curve={height=-6pt}, from=1-3, to=1-2]
			      \end{tikzcd}
		      \]
		      where \(p\circ \phi = \identity_{A} \). We claim that \(B = \phi (A) \oplus \ker (p) \) since for every \(b\in B\), \(\phi (p(b))\in \phi (A)\), and
		      \[
			      b= \underbrace{\phi (p(b))}_{\in \phi (A)} + \underbrace{\left(b - \phi (p(b))\right)}_{\in \ker (p)}
		      \]
		      from the fact that
		      \[
			      p(b - \phi (p(b))) = p(b) - p\circ \phi (p(b)) = p(b) - p(b) = 0.
		      \]
		      We need to show the uniqueness also. Suppose \(b = \phi (a_1) + d_1 = \phi (a_2) + d_2\), \(a_1, a_2\in A\), \(d_1, d_2\in \ker (p) \). We see that
		      \[
			      \phi (a_1 - a_2) = d_2 - d_1 \implies p(\phi (a_1 - a_2)) = 0 \implies a_1 = a_2 \implies d_1 = d_2.
		      \]

		      Finally, we claim that
		      \[
			      \at{\psi }{\ker  (p)}{} \colon \ker  (p)\to C
		      \]
		      is a group homeomorphism. But it's obvious that \(\at{\psi }{\ker (p) }{} \) are both surjective and injective.
		\item \(3. \implies 1.\) We have
		      \[
			      \begin{tikzcd}
				      \circ & A & B & C & \circ & {}
				      \arrow["\phi", from=1-2, to=1-3]
				      \arrow[from=1-1, to=1-2]
				      \arrow[from=1-4, to=1-5]
				      \arrow["\psi", from=1-3, to=1-4]
				      \arrow["q", curve={height=-6pt}, from=1-4, to=1-3]
			      \end{tikzcd}
		      \]
		      where \(\psi \circ q = \identity_{C} \). We claim that \(B = \phi (A)\oplus q(C)\) since for every \(b\in B\),
		      \[
			      b = \underbrace{(b - q(\psi (b)))}_{\in \ker (\psi ) = \im (\phi )  } + \underbrace{q(\psi (b))}_{\in q(C)},
		      \]
		      which implies \(B = \phi (A) + q(C)\). We can also prove that
		      \[
			      B = \phi (A) \oplus q(C)
		      \]
		      similarly.
	\end{itemize}
\end{proof}

\newpage
%─────Reference──────────────────────────────────────────────────────────────────────────────────────────────────────────────────────────────────────
\printbibliography

\end{document}