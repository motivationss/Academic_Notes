\chapter{Normed Vector Space}

\section{Norms}

\begin{definition}[Norm]\label{def: norm}
    Let \(E\) be a vector space over \(\mathbb{K}\). A function \(|| \cdot || \colon E \to
    \mathbb{R}^+\) is called a \textbf{norm} if the following hold:
    \begin{itemize}
        \item \( || x || = 0 \Leftrightarrow x = 0\). 
        \item \( || \lambda x|| = |\lambda| || x || \), \(x \in E\), \(\lambda \in \mathbb{K}\)
        (positive homogeneity)
        \item \( || x + y || \leq || x || + || y ||\), \(x, y \in E\) (triangle inequality).
        A pair (\(E, ||\cdot||\)) consisting of a vector space \(E\) and a norm
        \(||\cdot||\) is called a \textbf{normed vector space}. If the norm is clear from context,
        we write \(E\) instead of (\(E, || \cdot||\)).  
    \end{itemize}
\end{definition}

\begin{remark}
    Let \(E \coloneqq \left(E, ||\cdot||\right)\) be a normed vector space. 
    \begin{enumerate}[label=(\alph*)]
        \item The function 
        \[
            d \colon E \times E \to \mathbb{R}^+ , \:\:\:\:\: (x, y) \mapsto ||x-y||  
        \]
        is a metric on \(E\), the \textbf{metric induced from the norm}. Hence any 
        normed vector space is also a metric space. 
        \item The \textbf{reversed triangle inequality} holds for the norm:
        \[
            || x - y || \geq \left |  || x || - || y ||  \right |, \:\:\:\: x,y \in E  
        \]

        \item All statements from previous chapter also hold in normed vector space. 
    \end{enumerate}
\end{remark}

\subsection*{Balls}

For \(a \in E\) and \(r > 0\), we define the \textbf{open} and \textbf{closed balls} with
center at \(a\) and radius \(r\) by 
\[
   \mathbb{B}_E (a, r) \coloneqq \mathbb{B} (a, r) \coloneqq {x \in E; || x - a || < r}  
\]
and 
\[
  \bar{\mathbb{B}}_E (a, r) \coloneqq \bar{\mathbb{B}}_E \coloneqq {x\in E ; || x - a|| \leq r}.  
\]

These definitions agree with those for the metric space (\(E, d\)) when \(d\) is induced
from norm. We also write
\[
  \mathbb{B} \coloneqq \mathbb{B}(0, 1) = {x\in E; ||x|| < 1} \:\:\:\:\: \text{and}
  \:\:\:\:\: \bar{\mathbb{B}} \coloneqq \bar{\mathbb{B}}(0, 1) = {x \in E; ||x|| \leq 1}  
\]
for the \textbf{open} and \textbf{closed unit balls} in \(E\). We have
\[
    r\mathbb{B} = \mathbb{B}(0, r), \:\:\: r\bar{\mathbb{B}} = \bar{\mathbb{B}}(0, r)  
\]

\subsection*{Bounded Sets}

A subset \(X\) of \(E\) is called \textbf{bounded in} \(E\) (or \textbf{norm bounded}) if 
it is bounded in the induced metric space.

\begin{remark}
    Let \(E \coloneqq \left(E, ||\cdot||\right)\) be a normed vector space 
    \begin{enumerate}[label=(\alph*)]
        \item \(X \subseteq E\) is bounded if and only if there is some \(r > 0\) such
        that \(X \subseteq r\mathbb{B}\), that is, \(||x|| < r\) for all \(x\in X\). 
        \item If \(X\) and \(Y\) are nonempty bounded subsets of \(E\), then so are
        \(X \cup Y \), \(X + Y\) and \(\lambda X\) with \(\lambda \in \mathbb{K}\). 
    \end{enumerate}
\end{remark}

\begin{eg}
    \begin{enumerate}[label=(\alph*)]
        \item The absolute value \(|\cdot|\) is a norm on the vector space \(\mathbb{K}\). 
        \item Let \(F\) be a subspace of a normed vector space 
        \(E \coloneqq \left(E, \lVert \cdot \rVert \right)\). Then the restriction 
        \(\lVert \cdot \rVert_F \coloneqq  \lVert \cdot \rVert | F\) of 
        \(\lVert \cdot \rVert\) to \(F\) is a norm on \(F\). Thus 
        \(F \coloneqq \left(F, \lVert \cdot \rVert_F\right)\) is a normed vector space 
        with this \textbf{induced norm}. 
        \item Let \(E_j, \lVert \cdot \rVert_j\), \(1 \leq j \leq m\), be normed vector 
        space over \(\mathbb{K}\). Then

        \[
            \lVert x \rVert_\infty \coloneqq \displaystyle{\text{max}_{1\leq j\leq m}}
            \lVert x_j \rVert_j, \:\:\:\:\:\:\:\: x = (x_1,\ldots,x_m) \in E \coloneqq
            E_1 \times \cdots \times E_m   
        \]
        defines a norm, called the \textbf{product norm}, on the product vector space 
        \(E\). The metric on \(E\) induced from this norm coincides with the product metric
        from 5.2 example (d), where \(d_j\) is the metric induced on \(E_j\) from 
        \(\lVert \cdot \rVert_j\). 
        \item For \(m \in \mathbb{N}^\times, \mathbb{K}^m \) is a normed vector space
        with the \textbf{maximum norm}
        \[
            |x|_\infty \coloneqq     
        \] 
    \end{enumerate}
\end{eg}



