\chapter{The Real Numbers}

Starting words: we seek an ordered \textbf{extension field} of \(\mathbb{Q}\) in which
the equation \(x^2 = a\) is solvable for each \(a > 0\).

\section{Order Completeness}

We say a totally ordered set \(X\) is \textbf{order complete} (or \(X\) satisfies the \textbf{completeness axiom}) if every nonempty subset
of \(X\) which is bounded above has a supremum.

\begin{proposition}
    Let \(X\) be a totally ordered set. Then the followings are equivalent:
    \begin{enumerate}[label= (\roman*)]
        \item \(X\) is order complete. 
        \item Every nonempty subset of \(X\) which is bounded below has an infimum.
        \item For all nonempty subsets \(A, B\) of \(X\) such that \(a \leq b\) for all 
        (\(a, b\)) \(\in A \times B\), there is some \(c \in X\) such that \(a \leq c \leq b\)
        for all \(\left(a, b\right) \in A \times B\)  (\textbf{Dedekind cut property})
    \end{enumerate}
\end{proposition}
\begin{note}
    A relation \(\leq\) on \(X\) is a \textbf{partial order} on \(X\) if it is reflexive,
    transitive and \textbf{anti-symmetric}, that is, 
    \[
        (x \leq y) \and (y \leq x) \implies x = y  
    \]

    If \(\leq\) is a partial order on \(X\), then the pair (\(X, \leq\)) is called a
    \textbf{partially ordered set}. If, in addition,
    \[
        \forall x,y \in X \colon (x \leq y) \lor (y \leq x)  
    \]
    then \(\leq \) is called a \textbf{total order} on \(X\) and (\(X, \leq\)) is a \textbf{totally ordered set}. 
\end{note}

\begin{corollary}
    A totally ordered set is order complete if and only if every nonempty bounded subset 
    has a supremum and an infimum. 
\end{corollary}

\begin{theorem}[Dedekind's Construction of the Real Numbers]
    There is, up to isomorphism, a unique order complete extension field \(\mathbb{R}\) of 
    \(\mathbb{Q}\). This extension is called \textbf{the field of real numbers}. 
\end{theorem}

\begin{proposition} [A Characterization of Supremum and Infimum] 

    \begin{enumerate}[label=(\roman*)]
        Followed from natural order defined by \(\mathbb{R}\). 
        \item If \(A \subseteq \mathbb{R}\) and \(x \in \mathbb{R}\), then
        \begin{enumerate}
            \item \(x < \text{sup} \left( A \right) \Leftrightarrow  \exists a \in A  \) such that \(x < a\).
            \item  \(x < \text{inf} \left( A \right) \Leftrightarrow  \exists a \in A  \) such that \(x > a\).
        \end{enumerate}
        \item Every subset \(A\) of \(\mathbb{R}\) has a supremum and an infimum in \(\mathbb{R}\) 
    \end{enumerate}
\end{proposition}

\section{The Consequence of Order Completeness}

\subsection*{The Archimedean Property}

\begin{proposition}[Archimedes]
    \(\mathbb{N}\) is not bounded above in \(\mathbb{R}\), that is, for each \(x \in \mathbb{R}\)
    there is some \(n \in \mathbb{N}\) such that \(n > x\). 
\end{proposition}

\begin{corollary} Equivalent statements as the above proposition
    \begin{enumerate}[label=(\alph*)]
        \item Let \(a \in \mathbb{R}\). If \(0 \leq a \leq 1/n\) for all \(n \in \mathbb{N}^\times\).
        \item For each \(a \in \mathbb{R}\) with \(a > 0\) there is some \(n \in \mathbb{N}^\times\) such that \(1/n < a\). 
    \end{enumerate}
\end{corollary}

\subsection*{The Density of the Rational/Irrational Numbers in \(\mathbb{R}\)}

\begin{proposition}
    For all \(a, b \in \mathbb{R}\) such that \(a < b\), there is some \(r \in \mathbb{Q}\)
    such that \(a < r < b\). 
\end{proposition}

\begin{proposition}[\(n^{\text{th}}\) Roots]
    For all \(a \in \mathbb{R}^{+}\) and \(n \in \mathbb{N}^\times\), there is a unique
    \(x \in \mathbb{R}^+\) such that \(x^n = a\)
\end{proposition}

\begin{proposition}
    For all \(a, b \in \mathbb{R}\) such that \(a < b\), there is some \(\xi  \in \mathbb{R}\backslash\mathbb{Q}\)
    such that \(a < \xi  < b\).
\end{proposition}

\subsection*{Intervals}

An \textbf{interval} is a subset \(J\) of \(\mathbb{R}\) such that 
\[
    (x, y \in J, x < y) \implies (z \in J \text{for} x < z < y)
\]

If \(J\) is a nonempty interval, then inf(\(J\)) \(\in \bar{\mathbb{R}}\) is the \textbf{left endpoint}
and sup(\(J\)) \(\in \bar{\mathbb{R}}\) is the \textbf{right endpoint} of \(J\). \(J\) is
\textbf{closed on the left} if \(a \coloneqq \) inf(\(J\)) is in \(J\), and otherwise it is 
\textbf{open on the left}. The same applies to the other side.  \par 

An interval is \textbf{perfect} if it contains at least two points. It is \textbf{bounded}
if both endpoints are in \(\textbf{R}\) and is \textbf{unbounded} otherwise. If \(J\) is 
a bounded interval, then the nonnegative number \(\left | J \right | \coloneqq \)
sup(\(J\)) - inf(\(J\)) is called the \textbf{length} of \(J\). 